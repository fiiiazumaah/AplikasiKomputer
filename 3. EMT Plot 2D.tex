% Options for packages loaded elsewhere
\PassOptionsToPackage{unicode}{hyperref}
\PassOptionsToPackage{hyphens}{url}
\documentclass[
]{book}
\usepackage{xcolor}
\usepackage{amsmath,amssymb}
\setcounter{secnumdepth}{-\maxdimen} % remove section numbering
\usepackage{iftex}
\ifPDFTeX
  \usepackage[T1]{fontenc}
  \usepackage[utf8]{inputenc}
  \usepackage{textcomp} % provide euro and other symbols
\else % if luatex or xetex
  \usepackage{unicode-math} % this also loads fontspec
  \defaultfontfeatures{Scale=MatchLowercase}
  \defaultfontfeatures[\rmfamily]{Ligatures=TeX,Scale=1}
\fi
\usepackage{lmodern}
\ifPDFTeX\else
  % xetex/luatex font selection
\fi
% Use upquote if available, for straight quotes in verbatim environments
\IfFileExists{upquote.sty}{\usepackage{upquote}}{}
\IfFileExists{microtype.sty}{% use microtype if available
  \usepackage[]{microtype}
  \UseMicrotypeSet[protrusion]{basicmath} % disable protrusion for tt fonts
}{}
\makeatletter
\@ifundefined{KOMAClassName}{% if non-KOMA class
  \IfFileExists{parskip.sty}{%
    \usepackage{parskip}
  }{% else
    \setlength{\parindent}{0pt}
    \setlength{\parskip}{6pt plus 2pt minus 1pt}}
}{% if KOMA class
  \KOMAoptions{parskip=half}}
\makeatother
\usepackage{graphicx}
\makeatletter
\newsavebox\pandoc@box
\newcommand*\pandocbounded[1]{% scales image to fit in text height/width
  \sbox\pandoc@box{#1}%
  \Gscale@div\@tempa{\textheight}{\dimexpr\ht\pandoc@box+\dp\pandoc@box\relax}%
  \Gscale@div\@tempb{\linewidth}{\wd\pandoc@box}%
  \ifdim\@tempb\p@<\@tempa\p@\let\@tempa\@tempb\fi% select the smaller of both
  \ifdim\@tempa\p@<\p@\scalebox{\@tempa}{\usebox\pandoc@box}%
  \else\usebox{\pandoc@box}%
  \fi%
}
% Set default figure placement to htbp
\def\fps@figure{htbp}
\makeatother
\setlength{\emergencystretch}{3em} % prevent overfull lines
\providecommand{\tightlist}{%
  \setlength{\itemsep}{0pt}\setlength{\parskip}{0pt}}
\usepackage{bookmark}
\IfFileExists{xurl.sty}{\usepackage{xurl}}{} % add URL line breaks if available
\urlstyle{same}
\hypersetup{
  hidelinks,
  pdfcreator={LaTeX via pandoc}}

\author{}
\date{}

\begin{document}
\frontmatter

\mainmatter
\chapter{Menggambar Grafik 2D dengan EMT}\label{menggambar-grafik-2d-dengan-emt}

Notebook ini menjelaskan tentang cara menggambar berbagaikurva dan grafik 2D dengan software EMT. EMT menyediakan fungsi plot2d() untuk menggambar berbagai kurva dan grafik dua dimensi (2D).

\subsection{3.1 Plot Dasar}\label{plot-dasar}

Ada beberapa fungsi dasar plot. Ada koordinat layar, yang selalu berkisar dari 0 hingga 1024 di setiap sumbu, tidak peduli apakah layarnya persegi atau tidak. Selain itu, ada koordinat plot, yang dapat diatur dengan setplot(). Pemetaan antara koordinat bergantung pada jendela plot saat ini. Misalnya, shrinkwindow() default menyisakan ruang untuk label sumbu dan judul plot.

Dalam contoh, kita hanya menggambar beberapa garis acak dalam berbagai warna. Untuk detail tentang fungsi-fungsi ini, pelajari fungsi inti EMT.

\textgreater clg; // kosongkan layar

\textgreater window(0,0,1024,1024); // gunakan semua jendela

\textgreater setplot(0,1,0,1); // mengatur koordinat plot

\textgreater hold on; // mulai mode penimpaan

\textgreater n=100; X=random(n,2); Y=random(n,2); // membuat koordinat acak

\textgreater colors=rgb(random(n),random(n),random(n)); // dapatkan warna acak

\textgreater loop 1 to n; color(colors{[}\#{]}); plot(X{[}\#{]},Y{[}\#{]}); end; // plot

\textgreater hold off; // akhir mode penimpaan

\textgreater insimg; // masukan ke notebook

\begin{figure}
\centering
\pandocbounded{\includegraphics[keepaspectratio]{images/Alfi Nur Azumah-23030630002-MatB23-EMT Plot 2D-001.png}}
\caption{images/Alfi\%20Nur\%20Azumah-23030630002-MatB23-EMT\%20Plot\%202D-001.png}
\end{figure}

\textgreater reset;

Grafik perlu ditahan, karena perintah plot() akan membersihkan jendela plot.

Untuk membersihkan semua yang telah kita lakukan, kita menggunakan reset().

Untuk menampilkan gambar hasil plot di layar notebook, perintah plot2d() dapat diakhiri dengan titik dua (:). Cara lainnya adalah perintah plot2d() diakhiri dengan titik koma (;), kemudian menggunakan perintah insimg() untuk menampilkan gambar hasil plot.

Sebagai contoh lain, kita menggambar plot sebagai inset di plot lain. Ini dilakukan dengan mendefinisikan jendela plot yang lebih kecil. Perhatikan bahwa jendela ini tidak menyediakan ruang untuk label sumbu di luar jendela plot. Kita harus menambahkan beberapa margin untuk ini yang sesuai kebutuhan. Perhatikan bahwa kita menyimpan dan memulihkan jendela penuh, dan menahan plot saat ini saat kita memplot inset.

\textgreater plot2d(``x\^{}3-x'');

\textgreater xw=200; yw=100; ww=300; hw=300;

\textgreater ow=window();

\textgreater window(xw,yw,xw+ww,yw+hw);

\textgreater hold on;

\textgreater barclear(xw-50,yw-10,ww+60,ww+60);

\textgreater plot2d(``x\^{}4-x'',grid=6):

\begin{figure}
\centering
\pandocbounded{\includegraphics[keepaspectratio]{images/Alfi Nur Azumah-23030630002-MatB23-EMT Plot 2D-002.png}}
\caption{images/Alfi\%20Nur\%20Azumah-23030630002-MatB23-EMT\%20Plot\%202D-002.png}
\end{figure}

\textgreater hold off;

\textgreater window(ow);

Plot dengan beberapa gambar dibuat dengan cara yang sama. Ada fungsi utilitas figure() untuk ini.

\subsection{3.2 Aspek Plot}\label{aspek-plot}

Plot default menggunakan jendela plot persegi. Anda dapat mengubahnya dengan fungsi aspect(). Jangan lupa untuk mengatur ulang aspect nanti. Anda juga dapat mengubah default ini di menu dengan ``Set Aspect'' ke rasio aspek tertentu atau ke ukuran jendela grafik saat ini.

Namun, Anda juga dapat mengubahnya untuk satu plot. Untuk ini, ukuran area plot saat ini diubah, dan jendela diatur sehingga label memiliki cukup ruang.

\textgreater aspect(2); // rasio panjang dan lebar 2:1

\textgreater plot2d({[}``sin(x)'',``cos(x)''{]},0,2pi):

\begin{figure}
\centering
\pandocbounded{\includegraphics[keepaspectratio]{images/Alfi Nur Azumah-23030630002-MatB23-EMT Plot 2D-003.png}}
\caption{images/Alfi\%20Nur\%20Azumah-23030630002-MatB23-EMT\%20Plot\%202D-003.png}
\end{figure}

\textgreater aspect();

\textgreater reset;

Fungsi reset() mengembalikan pengaturan plot default termasuk rasio aspek.

\subsection{3.3 Plot 2D di Euler}\label{plot-2d-di-euler}

EMT Math Toolbox memiliki plot dalam 2D, baik untuk data maupun fungsi. EMT menggunakan fungsi plot2d. Fungsi ini dapat memplot fungsi dan data.

Anda dapat memplot dalam Maxima menggunakan Gnuplot atau dalam Python menggunakan Math Plot Lib.

Euler dapat memplot plot 2D dari 1. ekspresi fungsi, variabel, atau kurva berparameter, 2. vektor nilai x-y, 3. awan titik dalam bidang, 4. kurva implisit dengan level atau daerah level. 5. Fungsi kompleks

Gaya plot mencakup berbagai gaya untuk garis dan titik, plot batang, dan plot berbayang.

\subsection{3.4 Plot Ekspresi atau Variabel}\label{plot-ekspresi-atau-variabel}

Ekspresi tunggal dalam ``x'' (misalnya ``4*x\^{}2'') atau nama fungsi (misalnya ``f'') menghasilkan grafik fungsi.

Berikut adalah contoh paling dasar, yang menggunakan rentang default dan menetapkan rentang y yang tepat agar sesuai dengan plot fungsi.

Catatan: Jika Anda mengakhiri baris perintah dengan titik dua ``:'', plot akan disisipkan ke dalam jendela teks. Jika tidak, tekan TAB untuk melihat plot jika jendela plot tertutup.

\textgreater plot2d(``x\^{}2''):

\begin{figure}
\centering
\pandocbounded{\includegraphics[keepaspectratio]{images/Alfi Nur Azumah-23030630002-MatB23-EMT Plot 2D-004.png}}
\caption{images/Alfi\%20Nur\%20Azumah-23030630002-MatB23-EMT\%20Plot\%202D-004.png}
\end{figure}

\textgreater aspect(1.5); plot2d(``x\^{}3-x''):

\begin{figure}
\centering
\pandocbounded{\includegraphics[keepaspectratio]{images/Alfi Nur Azumah-23030630002-MatB23-EMT Plot 2D-005.png}}
\caption{images/Alfi\%20Nur\%20Azumah-23030630002-MatB23-EMT\%20Plot\%202D-005.png}
\end{figure}

\textgreater a:=5.6; plot2d(``exp(-a*x\^{}2)/a''); insimg(30); // menampilkan gambar hasil plot setinggi 25 baris

\begin{figure}
\centering
\pandocbounded{\includegraphics[keepaspectratio]{images/Alfi Nur Azumah-23030630002-MatB23-EMT Plot 2D-006.png}}
\caption{images/Alfi\%20Nur\%20Azumah-23030630002-MatB23-EMT\%20Plot\%202D-006.png}
\end{figure}

Dari beberapa contoh sebelumnya Anda dapat melihat bahwa aslinya gambar plot menggunakan sumbu X dengan rentang nilai dari -2 sampai dengan 2. Untuk mengubah rentang nilai X dan Y, Anda dapat menambahkan nilai-nilai batas X (dan Y) di belakang ekspresi yang digambar.

Rentang plot ditetapkan dengan parameter yang ditetapkan berikut

\begin{itemize}
\tightlist
\item
  a,b: rentang x (default -2,2)
\item
  c,d: rentang y (default: skala dengan nilai)
\item
  r: alternatifnya radius di sekitar pusat plot
\item
  cx,cy: koordinat pusat plot (default 0,0)
\item
  cx,cy: the coordinates of the plot center (default 0,0)
\end{itemize}

\textgreater plot2d(``x\^{}3-x'',-1,2):

\begin{figure}
\centering
\pandocbounded{\includegraphics[keepaspectratio]{images/Alfi Nur Azumah-23030630002-MatB23-EMT Plot 2D-007.png}}
\caption{images/Alfi\%20Nur\%20Azumah-23030630002-MatB23-EMT\%20Plot\%202D-007.png}
\end{figure}

\textgreater plot2d(``sin(x)'',-2*pi,2*pi): // plot sin(x) pada interval {[}-2pi, 2pi{]}

\begin{figure}
\centering
\pandocbounded{\includegraphics[keepaspectratio]{images/Alfi Nur Azumah-23030630002-MatB23-EMT Plot 2D-008.png}}
\caption{images/Alfi\%20Nur\%20Azumah-23030630002-MatB23-EMT\%20Plot\%202D-008.png}
\end{figure}

\textgreater plot2d(``cos(x)'',``sin(3*x)'',xmin=0,xmax=2pi):

\begin{figure}
\centering
\pandocbounded{\includegraphics[keepaspectratio]{images/Alfi Nur Azumah-23030630002-MatB23-EMT Plot 2D-009.png}}
\caption{images/Alfi\%20Nur\%20Azumah-23030630002-MatB23-EMT\%20Plot\%202D-009.png}
\end{figure}

Alternatif untuk titik dua adalah perintah insimg(lines), yang menyisipkan plot yang menempati sejumlah baris teks tertentu.

Dalam opsi, plot dapat diatur agar muncul + di jendela terpisah yang dapat diubah ukurannya, + di jendela buku catatan.

Lebih banyak gaya dapat dicapai dengan perintah plot tertentu.

Dalam kasus apa pun, tekan tombol tabulator untuk melihat plot, jika disembunyikan.

Untuk membagi jendela menjadi beberapa plot, gunakan perintah figure(). Dalam contoh, kita memplot x\^{}1 hingga x\^{}4 menjadi 4 bagian jendela. figure(0) mengatur ulang jendela default.

\textgreater reset;

\textgreater figure(2,2); \ldots{}\\
\textgreater{} for n=1 to 4; figure(n); plot2d(``x\^{}''+n); end; \ldots{}\\
\textgreater{} figure(0):

\begin{figure}
\centering
\pandocbounded{\includegraphics[keepaspectratio]{images/Alfi Nur Azumah-23030630002-MatB23-EMT Plot 2D-010.png}}
\caption{images/Alfi\%20Nur\%20Azumah-23030630002-MatB23-EMT\%20Plot\%202D-010.png}
\end{figure}

Dalam plot2d(), tersedia gaya alternatif dengan grid=x. Sebagai gambaran umum, kami menampilkan berbagai gaya grid dalam satu gambar (lihat di bawah untuk perintah figure()). Gaya grid=0 tidak disertakan. Gaya ini tidak menampilkan grid dan bingkai.

\textgreater figure(3,3); \ldots{}\\
\textgreater{} for k=1:9; figure(k); plot2d(``x\^{}3-x'',-2,1,grid=k); end; \ldots{}\\
\textgreater{} figure(0):

\begin{figure}
\centering
\pandocbounded{\includegraphics[keepaspectratio]{images/Alfi Nur Azumah-23030630002-MatB23-EMT Plot 2D-011.png}}
\caption{images/Alfi\%20Nur\%20Azumah-23030630002-MatB23-EMT\%20Plot\%202D-011.png}
\end{figure}

Jika argumen untuk plot2d() adalah ekspresi yang diikuti oleh empat angka, angka-angka ini adalah rentang x dan y untuk plot.

Atau, a, b, c, d dapat ditetapkan sebagai parameter yang ditetapkan sebagai a=\ldots{} dst.

Dalam contoh berikut, kami mengubah gaya kisi, menambahkan label, dan menggunakan label vertikal untuk sumbu y.

\textgreater aspect(1.5); plot2d(``sin(x)'',0,2pi,-1.2,1.2,grid=3,xl=``x'',yl=``sin(x)''):

\begin{figure}
\centering
\pandocbounded{\includegraphics[keepaspectratio]{images/Alfi Nur Azumah-23030630002-MatB23-EMT Plot 2D-012.png}}
\caption{images/Alfi\%20Nur\%20Azumah-23030630002-MatB23-EMT\%20Plot\%202D-012.png}
\end{figure}

\textgreater plot2d(``sin(x)+cos(2*x)'',0,4pi):

\begin{figure}
\centering
\pandocbounded{\includegraphics[keepaspectratio]{images/Alfi Nur Azumah-23030630002-MatB23-EMT Plot 2D-013.png}}
\caption{images/Alfi\%20Nur\%20Azumah-23030630002-MatB23-EMT\%20Plot\%202D-013.png}
\end{figure}

Gambar yang dihasilkan dengan memasukkan plot ke dalam jendela teks disimpan dalam direktori yang sama dengan buku catatan, secara default dalam subdirektori bernama ``images''. Gambar tersebut juga digunakan oleh ekspor HTML.

Anda cukup menandai gambar apa pun dan menyalinnya ke clipboard dengan Ctrl-C. Tentu saja, Anda juga dapat mengekspor grafik saat ini dengan fungsi-fungsi di menu File.

Fungsi atau ekspresi dalam plot2d dievaluasi secara adaptif. Untuk kecepatan lebih, matikan plot adaptif dengan \textless adaptive dan tentukan jumlah subinterval dengan n=\ldots{} Ini hanya diperlukan dalam kasus yang jarang terjadi.

\textgreater plot2d(``sign(x)*exp(-x\^{}2)'',-1,1,\textless adaptive,n=10000):

\begin{figure}
\centering
\pandocbounded{\includegraphics[keepaspectratio]{images/Alfi Nur Azumah-23030630002-MatB23-EMT Plot 2D-014.png}}
\caption{images/Alfi\%20Nur\%20Azumah-23030630002-MatB23-EMT\%20Plot\%202D-014.png}
\end{figure}

\textgreater plot2d(``x\^{}x'',r=1.2,cx=1,cy=1):

\begin{figure}
\centering
\pandocbounded{\includegraphics[keepaspectratio]{images/Alfi Nur Azumah-23030630002-MatB23-EMT Plot 2D-015.png}}
\caption{images/Alfi\%20Nur\%20Azumah-23030630002-MatB23-EMT\%20Plot\%202D-015.png}
\end{figure}

Perhatikan bahwa x\^{}x tidak didefinisikan untuk x\textless=0. Fungsi plot2d menangkap kesalahan ini, dan mulai memplot segera setelah fungsi didefinisikan. Ini berfungsi untuk semua fungsi yang mengembalikan NAN di luar rentang definisinya.

\textgreater plot2d(``log(x)'',-0.1,2):

\begin{figure}
\centering
\pandocbounded{\includegraphics[keepaspectratio]{images/Alfi Nur Azumah-23030630002-MatB23-EMT Plot 2D-016.png}}
\caption{images/Alfi\%20Nur\%20Azumah-23030630002-MatB23-EMT\%20Plot\%202D-016.png}
\end{figure}

Parameter square=true (atau \textgreater square) memilih rentang y secara otomatis sehingga hasilnya adalah jendela plot persegi. Perhatikan bahwa secara default, Euler menggunakan ruang persegi di dalam jendela plot.

\textgreater plot2d(``x\^{}3-x'',\textgreater square):

\begin{figure}
\centering
\pandocbounded{\includegraphics[keepaspectratio]{images/Alfi Nur Azumah-23030630002-MatB23-EMT Plot 2D-017.png}}
\caption{images/Alfi\%20Nur\%20Azumah-23030630002-MatB23-EMT\%20Plot\%202D-017.png}
\end{figure}

\textgreater plot2d(`'integrate(``sin(x)*exp(-x\^{}2)'',0,x)'\,',0,2): // plot integral

\begin{figure}
\centering
\pandocbounded{\includegraphics[keepaspectratio]{images/Alfi Nur Azumah-23030630002-MatB23-EMT Plot 2D-018.png}}
\caption{images/Alfi\%20Nur\%20Azumah-23030630002-MatB23-EMT\%20Plot\%202D-018.png}
\end{figure}

Jika Anda memerlukan lebih banyak ruang untuk label-y, panggil shrinkwindow() dengan parameter yang lebih kecil, atau tetapkan nilai positif untuk ``lebih kecil'' di plot2d().

\textgreater plot2d(``gamma(x)'',1,10,yl=``y-values'',smaller=6,\textless vertical):

\begin{figure}
\centering
\pandocbounded{\includegraphics[keepaspectratio]{images/Alfi Nur Azumah-23030630002-MatB23-EMT Plot 2D-019.png}}
\caption{images/Alfi\%20Nur\%20Azumah-23030630002-MatB23-EMT\%20Plot\%202D-019.png}
\end{figure}

Ekspresi simbolik juga dapat digunakan, karena disimpan sebagai ekspresi string sederhana.

\textgreater x=linspace(0,2pi,1000); plot2d(sin(5x),cos(7x)):

\begin{figure}
\centering
\pandocbounded{\includegraphics[keepaspectratio]{images/Alfi Nur Azumah-23030630002-MatB23-EMT Plot 2D-020.png}}
\caption{images/Alfi\%20Nur\%20Azumah-23030630002-MatB23-EMT\%20Plot\%202D-020.png}
\end{figure}

\textgreater a:=5.6; expr \&= exp(-a*x\^{}2)/a; // define expression

\textgreater plot2d(expr,-2,2): // plot from -2 to 2

\begin{figure}
\centering
\pandocbounded{\includegraphics[keepaspectratio]{images/Alfi Nur Azumah-23030630002-MatB23-EMT Plot 2D-021.png}}
\caption{images/Alfi\%20Nur\%20Azumah-23030630002-MatB23-EMT\%20Plot\%202D-021.png}
\end{figure}

\textgreater plot2d(expr,r=1,thickness=2): // plot in a square around (0,0)

\begin{figure}
\centering
\pandocbounded{\includegraphics[keepaspectratio]{images/Alfi Nur Azumah-23030630002-MatB23-EMT Plot 2D-022.png}}
\caption{images/Alfi\%20Nur\%20Azumah-23030630002-MatB23-EMT\%20Plot\%202D-022.png}
\end{figure}

\textgreater plot2d(\&diff(expr,x),\textgreater add,style=``--'',color=red): // add another plot

\begin{figure}
\centering
\pandocbounded{\includegraphics[keepaspectratio]{images/Alfi Nur Azumah-23030630002-MatB23-EMT Plot 2D-023.png}}
\caption{images/Alfi\%20Nur\%20Azumah-23030630002-MatB23-EMT\%20Plot\%202D-023.png}
\end{figure}

\textgreater plot2d(\&diff(expr,x,2),a=-2,b=2,c=-2,d=1): // plot in rectangle

\begin{figure}
\centering
\pandocbounded{\includegraphics[keepaspectratio]{images/Alfi Nur Azumah-23030630002-MatB23-EMT Plot 2D-024.png}}
\caption{images/Alfi\%20Nur\%20Azumah-23030630002-MatB23-EMT\%20Plot\%202D-024.png}
\end{figure}

\textgreater plot2d(\&diff(expr,x),a=-2,b=2,\textgreater square): // keep plot square

\begin{figure}
\centering
\pandocbounded{\includegraphics[keepaspectratio]{images/Alfi Nur Azumah-23030630002-MatB23-EMT Plot 2D-025.png}}
\caption{images/Alfi\%20Nur\%20Azumah-23030630002-MatB23-EMT\%20Plot\%202D-025.png}
\end{figure}

\textgreater plot2d(``x\^{}2'',0,1,steps=1,color=red,n=10):

\begin{figure}
\centering
\pandocbounded{\includegraphics[keepaspectratio]{images/Alfi Nur Azumah-23030630002-MatB23-EMT Plot 2D-026.png}}
\caption{images/Alfi\%20Nur\%20Azumah-23030630002-MatB23-EMT\%20Plot\%202D-026.png}
\end{figure}

\textgreater plot2d(``x\^{}2'',\textgreater add,steps=2,color=blue,n=10):

\begin{figure}
\centering
\pandocbounded{\includegraphics[keepaspectratio]{images/Alfi Nur Azumah-23030630002-MatB23-EMT Plot 2D-027.png}}
\caption{images/Alfi\%20Nur\%20Azumah-23030630002-MatB23-EMT\%20Plot\%202D-027.png}
\end{figure}

\subsection{3.5 Fungsi dalam satu Parameter}\label{fungsi-dalam-satu-parameter}

Fungsi plotting yang paling penting untuk plot planar adalah plot2d(). Fungsi ini diimplementasikan dalam bahasa Euler dalam file ``plot.e'', yang dimuat di awal program.

Berikut ini beberapa contoh penggunaan fungsi. Seperti biasa dalam EMT, fungsi yang berfungsi untuk fungsi atau ekspresi lain, Anda dapat meneruskan parameter tambahan (selain x) yang bukan variabel global ke fungsi dengan parameter titik koma atau dengan koleksi panggilan.

\textgreater function f(x,a) := x\textsuperscript{2/a+a*x}2-x; // define a function

\textgreater a=0.3; plot2d(``f'',0,1;a): // plot with a=0.3

\begin{figure}
\centering
\pandocbounded{\includegraphics[keepaspectratio]{images/Alfi Nur Azumah-23030630002-MatB23-EMT Plot 2D-028.png}}
\caption{images/Alfi\%20Nur\%20Azumah-23030630002-MatB23-EMT\%20Plot\%202D-028.png}
\end{figure}

\textgreater plot2d(``f'',0,1;0.4): // plot with a=0.4

\begin{figure}
\centering
\pandocbounded{\includegraphics[keepaspectratio]{images/Alfi Nur Azumah-23030630002-MatB23-EMT Plot 2D-029.png}}
\caption{images/Alfi\%20Nur\%20Azumah-23030630002-MatB23-EMT\%20Plot\%202D-029.png}
\end{figure}

\textgreater plot2d(\{\{``f'',0.2\}\},0,1): // plot with a=0.2

\begin{figure}
\centering
\pandocbounded{\includegraphics[keepaspectratio]{images/Alfi Nur Azumah-23030630002-MatB23-EMT Plot 2D-030.png}}
\caption{images/Alfi\%20Nur\%20Azumah-23030630002-MatB23-EMT\%20Plot\%202D-030.png}
\end{figure}

\textgreater plot2d(\{\{``f(x,b)'',b=0.1\}\},0,1): // plot with 0.1

\begin{figure}
\centering
\pandocbounded{\includegraphics[keepaspectratio]{images/Alfi Nur Azumah-23030630002-MatB23-EMT Plot 2D-031.png}}
\caption{images/Alfi\%20Nur\%20Azumah-23030630002-MatB23-EMT\%20Plot\%202D-031.png}
\end{figure}

\textgreater function f(x) := x\^{}3-x; \ldots{}\\
\textgreater{} plot2d(``f'',r=1):

\begin{figure}
\centering
\pandocbounded{\includegraphics[keepaspectratio]{images/Alfi Nur Azumah-23030630002-MatB23-EMT Plot 2D-032.png}}
\caption{images/Alfi\%20Nur\%20Azumah-23030630002-MatB23-EMT\%20Plot\%202D-032.png}
\end{figure}

Berikut ini adalah ringkasan fungsi yang diterima 1. ekspresi atau ekspresi simbolik dalam x 2. fungsi atau fungsi simbolik berdasarkan nama seperti ``f'' 3. fungsi simbolik hanya berdasarkan nama f

Fungsi plot2d() juga menerima fungsi simbolik. Untuk fungsi simbolik, hanya nama yang berfungsi.

\textgreater function f(x) \&= diff(x\^{}x,x)

\begin{verbatim}
                            x
                           x  (log(x) + 1)
\end{verbatim}

\textgreater plot2d(f (x),0,2):

\begin{verbatim}
log defined only for positive numbers!
Error in log
Try "trace errors" to inspect local variables after errors.
f:
    useglobal; return x^x*(log(x)+1) 
Error in:
plot2d(f (x),0,2): ...
            ^
\end{verbatim}

Tentu saja, untuk ekspresi atau ungkapan simbolik, nama variabel sudah cukup untuk memplotnya.

\textgreater expr \&= sin(x)*exp(-x)

\begin{verbatim}
                              - x
                             E    sin(x)
\end{verbatim}

\textgreater plot2d(expr,0,3pi):

\begin{figure}
\centering
\pandocbounded{\includegraphics[keepaspectratio]{images/Alfi Nur Azumah-23030630002-MatB23-EMT Plot 2D-033.png}}
\caption{images/Alfi\%20Nur\%20Azumah-23030630002-MatB23-EMT\%20Plot\%202D-033.png}
\end{figure}

\textgreater function f(x) \&= x\^{}x;

\textgreater plot2d(f,r=1,cx=1,cy=1,color=blue,thickness=2);

\textgreater plot2d(\&diff(f(x),x),\textgreater add,color=red,style=``-.-''):

\begin{figure}
\centering
\pandocbounded{\includegraphics[keepaspectratio]{images/Alfi Nur Azumah-23030630002-MatB23-EMT Plot 2D-034.png}}
\caption{images/Alfi\%20Nur\%20Azumah-23030630002-MatB23-EMT\%20Plot\%202D-034.png}
\end{figure}

Untuk gaya garis, ada berbagai pilihan. + style=``\ldots{}''. Pilih dari ``-'', ``--'', ``-.'', ``.'', ``.-.'', ``-.-''. + color: Lihat di bawah untuk warna.

thickness: Default adalah 1.

Warna dapat dipilih sebagai salah satu warna default, atau sebagai warna RGB. + 0..15: indeks warna default. + konstanta warna: putih, hitam, merah, hijau, biru, cyan, zaitun, abu-abu muda, abu-abu, abu-abu tua, oranye, hijau muda, biru kehijauan, biru muda, oranye muda, kuning + rgb(merah,hijau,biru): parameter adalah bilangan real dalam {[}0,1{]}.

\textgreater plot2d(``exp(-x\^{}2)'',r=2,color=red,thickness=3,style=``--''):

\begin{figure}
\centering
\pandocbounded{\includegraphics[keepaspectratio]{images/Alfi Nur Azumah-23030630002-MatB23-EMT Plot 2D-035.png}}
\caption{images/Alfi\%20Nur\%20Azumah-23030630002-MatB23-EMT\%20Plot\%202D-035.png}
\end{figure}

Berikut ini tampilan warna EMT yang telah ditetapkan sebelumnya.

\textgreater aspect(2); columnsplot(ones(1,16),lab=0:15,grid=0,color=0:15):

\begin{figure}
\centering
\pandocbounded{\includegraphics[keepaspectratio]{images/Alfi Nur Azumah-23030630002-MatB23-EMT Plot 2D-036.png}}
\caption{images/Alfi\%20Nur\%20Azumah-23030630002-MatB23-EMT\%20Plot\%202D-036.png}
\end{figure}

Namun Anda dapat menggunakan warna apa pun.

\textgreater columnsplot(ones(1,16),grid=0,color=rgb(0,0,linspace(0,1,15))):

\begin{figure}
\centering
\pandocbounded{\includegraphics[keepaspectratio]{images/Alfi Nur Azumah-23030630002-MatB23-EMT Plot 2D-037.png}}
\caption{images/Alfi\%20Nur\%20Azumah-23030630002-MatB23-EMT\%20Plot\%202D-037.png}
\end{figure}

\subsection{3.6 Menggambar Beberapa Kurva pada bidang koordinat yang sama}\label{menggambar-beberapa-kurva-pada-bidang-koordinat-yang-sama}

Plotting lebih dari satu fungsi (multifungsi) ke dalam satu jendela dapat dilakukan dengan berbagai cara. Salah satu metodenya adalah menggunakan \textgreater add untuk beberapa panggilan ke plot2d secara keseluruhan, kecuali panggilan pertama. Kami telah menggunakan fitur ini pada contoh di atas.

\textgreater aspect(); plot2d(``cos(x)'',r=2,grid=6); plot2d(``x'',style=``.'',\textgreater add):

\begin{figure}
\centering
\pandocbounded{\includegraphics[keepaspectratio]{images/Alfi Nur Azumah-23030630002-MatB23-EMT Plot 2D-038.png}}
\caption{images/Alfi\%20Nur\%20Azumah-23030630002-MatB23-EMT\%20Plot\%202D-038.png}
\end{figure}

\textgreater aspect(1.5); plot2d(``sin(x)'',0,2pi); plot2d(``cos(x)'',color=blue,style=``--'',\textgreater add):

\begin{figure}
\centering
\pandocbounded{\includegraphics[keepaspectratio]{images/Alfi Nur Azumah-23030630002-MatB23-EMT Plot 2D-039.png}}
\caption{images/Alfi\%20Nur\%20Azumah-23030630002-MatB23-EMT\%20Plot\%202D-039.png}
\end{figure}

Salah satu kegunaan \textgreater add adalah untuk menambahkan titik pada kurva.

\textgreater plot2d(``sin(x)'',0,pi); plot2d(2,sin(2),\textgreater points,\textgreater add):

\begin{figure}
\centering
\pandocbounded{\includegraphics[keepaspectratio]{images/Alfi Nur Azumah-23030630002-MatB23-EMT Plot 2D-040.png}}
\caption{images/Alfi\%20Nur\%20Azumah-23030630002-MatB23-EMT\%20Plot\%202D-040.png}
\end{figure}

Kami menambahkan titik potong dengan label (pada posisi ``cl'' untuk tengah kiri), dan memasukkan hasilnya ke dalam buku catatan. Kami juga menambahkan judul pada plot.

\textgreater plot2d({[}``cos(x)'',``x''{]},r=1.1,cx=0.5,cy=0.5, \ldots{}\\
\textgreater{} color={[}black,blue{]},style={[}``-'',``.''{]}, \ldots{}\\
\textgreater{} grid=1):

\begin{figure}
\centering
\pandocbounded{\includegraphics[keepaspectratio]{images/Alfi Nur Azumah-23030630002-MatB23-EMT Plot 2D-041.png}}
\caption{images/Alfi\%20Nur\%20Azumah-23030630002-MatB23-EMT\%20Plot\%202D-041.png}
\end{figure}

\textgreater x0=solve(``cos(x)-x'',1); \ldots{}\\
\textgreater{} plot2d(x0,x0,\textgreater points,\textgreater add,title=``Intersection Demo''); \ldots{}\\
\textgreater{} label(``cos(x) = x'',x0,x0,pos=``cl'',offset=20):

\begin{figure}
\centering
\pandocbounded{\includegraphics[keepaspectratio]{images/Alfi Nur Azumah-23030630002-MatB23-EMT Plot 2D-042.png}}
\caption{images/Alfi\%20Nur\%20Azumah-23030630002-MatB23-EMT\%20Plot\%202D-042.png}
\end{figure}

Dalam demo berikut, kami memplot fungsi sinc(x)=sin(x)/x dan ekspansi Taylor ke-8 dan ke-16. Kami menghitung ekspansi ini menggunakan Maxima melalui ekspresi simbolik.

Plot ini dilakukan dalam perintah multi-baris berikut dengan tiga panggilan ke plot2d(). Yang kedua dan ketiga memiliki set flag \textgreater add, yang membuat plot menggunakan rentang sebelumnya.

Kami menambahkan kotak label yang menjelaskan fungsi-fungsi tersebut.

\textgreater\$taylor(sin(x)/x,x,0,4)

\[\frac{x^4}{120}-\frac{x^2}{6}+1\]\textgreater plot2d(``sinc(x)'',0,4pi,color=green,thickness=2); \ldots{}\\
\textgreater{} plot2d(\&taylor(sin(x)/x,x,0,8),\textgreater add,color=blue,style=``--''); \ldots{}\\
\textgreater{} plot2d(\&taylor(sin(x)/x,x,0,16),\textgreater add,color=red,style=``-.-''); \ldots{}\\
\textgreater{} labelbox({[}``sinc'',``T8'',``T16''{]},styles={[}``-'',``--'',``-.-''{]}, \ldots{}\\
\textgreater{} colors={[}black,blue,red{]}):

\begin{figure}
\centering
\pandocbounded{\includegraphics[keepaspectratio]{images/Alfi Nur Azumah-23030630002-MatB23-EMT Plot 2D-044.png}}
\caption{images/Alfi\%20Nur\%20Azumah-23030630002-MatB23-EMT\%20Plot\%202D-044.png}
\end{figure}

Dalam contoh berikut, kami menghasilkan Polinomial Bernstein.

\textgreater plot2d(``(1-x)\^{}10'',0,1); // plot first function

\textgreater for i=1 to 10; plot2d(``bin(10,i)*x\textsuperscript{i*(1-x)}(10-i)'',\textgreater add); end;

\textgreater insimg;

\begin{figure}
\centering
\pandocbounded{\includegraphics[keepaspectratio]{images/Alfi Nur Azumah-23030630002-MatB23-EMT Plot 2D-045.png}}
\caption{images/Alfi\%20Nur\%20Azumah-23030630002-MatB23-EMT\%20Plot\%202D-045.png}
\end{figure}

Metode kedua menggunakan sepasang matriks nilai-x dan matriks nilai-y dengan ukuran yang sama.

Kita buat matriks nilai dengan satu Polinomial Bernstein di setiap baris. Untuk ini, kita cukup menggunakan vektor kolom i. Lihat pengantar tentang bahasa matriks untuk mempelajari lebih detail.

\textgreater x=linspace(0,1,500);

\textgreater n=10; k=(0:n)'; // n is row vector, k is column vector

\textgreater y=bin(n,k)*x\textsuperscript{k*(1-x)}(n-k); // y is a matrix then

\textgreater plot2d(x,y):

\begin{figure}
\centering
\pandocbounded{\includegraphics[keepaspectratio]{images/Alfi Nur Azumah-23030630002-MatB23-EMT Plot 2D-046.png}}
\caption{images/Alfi\%20Nur\%20Azumah-23030630002-MatB23-EMT\%20Plot\%202D-046.png}
\end{figure}

Perhatikan bahwa parameter warna dapat berupa vektor. Maka setiap warna digunakan untuk setiap baris matriks.

\textgreater x=linspace(0,1,200); y=x\^{}(1:10)'; plot2d(x,y,color=1:10):

\begin{figure}
\centering
\pandocbounded{\includegraphics[keepaspectratio]{images/Alfi Nur Azumah-23030630002-MatB23-EMT Plot 2D-047.png}}
\caption{images/Alfi\%20Nur\%20Azumah-23030630002-MatB23-EMT\%20Plot\%202D-047.png}
\end{figure}

Metode lain adalah menggunakan vektor ekspresi (string). Anda kemudian dapat menggunakan array warna, array gaya, dan array ketebalan dengan panjang yang sama.

\textgreater plot2d({[}``sin(x)'',``cos(x)''{]},0,2pi,color=4:5):

\begin{figure}
\centering
\pandocbounded{\includegraphics[keepaspectratio]{images/Alfi Nur Azumah-23030630002-MatB23-EMT Plot 2D-048.png}}
\caption{images/Alfi\%20Nur\%20Azumah-23030630002-MatB23-EMT\%20Plot\%202D-048.png}
\end{figure}

\textgreater plot2d({[}``sin(x)'',``cos(x)''{]},0,2pi): // plot vector of expressions

\begin{figure}
\centering
\pandocbounded{\includegraphics[keepaspectratio]{images/Alfi Nur Azumah-23030630002-MatB23-EMT Plot 2D-049.png}}
\caption{images/Alfi\%20Nur\%20Azumah-23030630002-MatB23-EMT\%20Plot\%202D-049.png}
\end{figure}

Kita bisa mendapatkan vektor tersebut dari Maxima menggunakan makelist() dan mxm2str().

\textgreater v \&= makelist(binomial(10,i)*x\textsuperscript{i*(1-x)}(10-i),i,0,10) // make list

\begin{verbatim}
               10            9              8  2             7  3
       [(1 - x)  , 10 (1 - x)  x, 45 (1 - x)  x , 120 (1 - x)  x , 
           6  4             5  5             4  6             3  7
210 (1 - x)  x , 252 (1 - x)  x , 210 (1 - x)  x , 120 (1 - x)  x , 
          2  8              9   10
45 (1 - x)  x , 10 (1 - x) x , x  ]
\end{verbatim}

\textgreater mxm2str(v) // get a vector of strings from the symbolic vector

\begin{verbatim}
(1-x)^10
10*(1-x)^9*x
45*(1-x)^8*x^2
120*(1-x)^7*x^3
210*(1-x)^6*x^4
252*(1-x)^5*x^5
210*(1-x)^4*x^6
120*(1-x)^3*x^7
45*(1-x)^2*x^8
10*(1-x)*x^9
x^10
\end{verbatim}

\textgreater plot2d(mxm2str(v),0,1): // plot functions

\begin{figure}
\centering
\pandocbounded{\includegraphics[keepaspectratio]{images/Alfi Nur Azumah-23030630002-MatB23-EMT Plot 2D-050.png}}
\caption{images/Alfi\%20Nur\%20Azumah-23030630002-MatB23-EMT\%20Plot\%202D-050.png}
\end{figure}

Alternatif lain adalah dengan menggunakan bahasa matriks Euler.

Jika suatu ekspresi menghasilkan matriks fungsi, dengan satu fungsi di setiap baris, semua fungsi ini akan diplot menjadi satu plot.

Untuk ini, gunakan vektor parameter dalam bentuk vektor kolom. Jika array warna ditambahkan, array tersebut akan digunakan untuk setiap baris plot.

\textgreater n=(1:10)'; plot2d(``x\^{}n'',0,1,color=1:10):

\begin{figure}
\centering
\pandocbounded{\includegraphics[keepaspectratio]{images/Alfi Nur Azumah-23030630002-MatB23-EMT Plot 2D-051.png}}
\caption{images/Alfi\%20Nur\%20Azumah-23030630002-MatB23-EMT\%20Plot\%202D-051.png}
\end{figure}

Ekspresi dan fungsi satu baris dapat melihat variabel global.

Jika Anda tidak dapat menggunakan variabel global, Anda perlu menggunakan fungsi dengan parameter tambahan, dan meneruskan parameter ini sebagai parameter titik koma.

Berhati-hatilah, untuk meletakkan semua parameter yang ditetapkan di akhir perintah plot2d. Dalam contoh ini, kami meneruskan a=5 ke fungsi f, yang kami plot dari -10 hingga 10.

\textgreater function f(x,a) := 1/a*exp(-x\^{}2/a); \ldots{}\\
\textgreater{} plot2d(``f'',-10,10;5,thickness=2,title=``a=5''):

\begin{figure}
\centering
\pandocbounded{\includegraphics[keepaspectratio]{images/Alfi Nur Azumah-23030630002-MatB23-EMT Plot 2D-052.png}}
\caption{images/Alfi\%20Nur\%20Azumah-23030630002-MatB23-EMT\%20Plot\%202D-052.png}
\end{figure}

Atau, gunakan koleksi dengan nama fungsi dan semua parameter tambahan. Daftar khusus ini disebut koleksi panggilan, dan itu adalah cara yang lebih disukai untuk meneruskan argumen ke suatu fungsi yang diteruskan sebagai argumen ke fungsi lain.

Dalam contoh berikut, kami menggunakan loop untuk memplot beberapa fungsi (lihat tutorial tentang pemrograman untuk loop).

\textgreater plot2d(\{\{``f'',1\}\},-10,10); \ldots{}\\
\textgreater{} for a=2:10; plot2d(\{\{``f'',a\}\},\textgreater add); end:

\begin{figure}
\centering
\pandocbounded{\includegraphics[keepaspectratio]{images/Alfi Nur Azumah-23030630002-MatB23-EMT Plot 2D-053.png}}
\caption{images/Alfi\%20Nur\%20Azumah-23030630002-MatB23-EMT\%20Plot\%202D-053.png}
\end{figure}

Kita dapat memperoleh hasil yang sama dengan cara berikut menggunakan bahasa matriks EMT. Setiap baris matriks f(x,a) adalah satu fungsi. Selain itu, kita dapat mengatur warna untuk setiap baris matriks. Klik dua kali pada fungsi getspectral() untuk penjelasannya.

\textgreater x=-10:0.01:10; a=(1:10)'; plot2d(x,f(x,a),color=getspectral(a/10)):

\begin{figure}
\centering
\pandocbounded{\includegraphics[keepaspectratio]{images/Alfi Nur Azumah-23030630002-MatB23-EMT Plot 2D-054.png}}
\caption{images/Alfi\%20Nur\%20Azumah-23030630002-MatB23-EMT\%20Plot\%202D-054.png}
\end{figure}

\subsection{3.7 Label Teks}\label{label-teks}

Dekorasi sederhana dapat berupa - judul dengan title=``\ldots{}'' - label x dan y dengan xl=``\ldots{}'', yl=``\ldots{}'' - label teks lain dengan label(``\ldots{}'',x,y)

Perintah label akan memplot ke dalam plot saat ini pada koordinat plot (x,y). Perintah ini dapat mengambil argumen posisi.

\textgreater plot2d(``x\textsuperscript{3-x'',-1,2,title=''y=x}3-x'',yl=``y'',xl=``x''):

\begin{figure}
\centering
\pandocbounded{\includegraphics[keepaspectratio]{images/Alfi Nur Azumah-23030630002-MatB23-EMT Plot 2D-055.png}}
\caption{images/Alfi\%20Nur\%20Azumah-23030630002-MatB23-EMT\%20Plot\%202D-055.png}
\end{figure}

\textgreater expr := ``log(x)/x''; \ldots{}\\
\textgreater{} plot2d(expr,0.5,5,title=``y=''+expr,xl=``x'',yl=``y''); \ldots{}\\
\textgreater{} label(``(1,0)'',1,0); label(``Max'',E,expr(E),pos=``lc''):

\begin{figure}
\centering
\pandocbounded{\includegraphics[keepaspectratio]{images/Alfi Nur Azumah-23030630002-MatB23-EMT Plot 2D-056.png}}
\caption{images/Alfi\%20Nur\%20Azumah-23030630002-MatB23-EMT\%20Plot\%202D-056.png}
\end{figure}

Ada juga fungsi labelbox(), yang dapat menampilkan fungsi dan teks. Fungsi ini mengambil vektor string dan warna, satu item untuk setiap fungsi.

\textgreater function f(x) \&= x\textsuperscript{2*exp(-x}2); \ldots{}\\
\textgreater{} plot2d(\&f(x),a=-3,b=3,c=-1,d=1); \ldots{}\\
\textgreater{} plot2d(\&diff(f(x),x),\textgreater add,color=blue,style=``--''); \ldots{}\\
\textgreater{} labelbox({[}``function'',``derivative''{]},styles={[}``-'',``--''{]}, \ldots{}\\
\textgreater{} colors={[}black,blue{]},w=0.4):

\begin{figure}
\centering
\pandocbounded{\includegraphics[keepaspectratio]{images/Alfi Nur Azumah-23030630002-MatB23-EMT Plot 2D-057.png}}
\caption{images/Alfi\%20Nur\%20Azumah-23030630002-MatB23-EMT\%20Plot\%202D-057.png}
\end{figure}

Kotak tersebut ditambatkan di kanan atas secara default, tetapi \textgreater left menambatkannya di kiri atas. Anda dapat memindahkannya ke tempat mana pun yang Anda suka. Posisi jangkar adalah sudut kanan atas kotak, dan angka-angkanya adalah pecahan dari ukuran jendela grafik. Lebarnya otomatis.

Untuk plot titik, kotak label juga berfungsi. Tambahkan parameter \textgreater points, atau vektor bendera, satu untuk setiap label.

Dalam contoh berikut, hanya ada satu fungsi. Jadi, kita dapat menggunakan string alih-alih vektor string. Kita tetapkan warna teks menjadi hitam untuk contoh ini.

\textgreater n=10; plot2d(0:n,bin(n,0:n),\textgreater addpoints); \ldots{}\\
\textgreater{} labelbox(``Binomials'',styles=``{[}{]}'',\textgreater points,x=0.1,y=0.1, \ldots{}\\
\textgreater{} tcolor=black,\textgreater left):

\begin{figure}
\centering
\pandocbounded{\includegraphics[keepaspectratio]{images/Alfi Nur Azumah-23030630002-MatB23-EMT Plot 2D-058.png}}
\caption{images/Alfi\%20Nur\%20Azumah-23030630002-MatB23-EMT\%20Plot\%202D-058.png}
\end{figure}

Gaya plot ini juga tersedia di statplot(). Seperti di plot2d(), warna dapat diatur untuk setiap baris plot. Ada plot yang lebih khusus untuk keperluan statistik (lihat tutorial tentang statistik).

\textgreater statplot(1:10,random(3,10),color={[}red,blue,black{]}):

\begin{figure}
\centering
\pandocbounded{\includegraphics[keepaspectratio]{images/Alfi Nur Azumah-23030630002-MatB23-EMT Plot 2D-059.png}}
\caption{images/Alfi\%20Nur\%20Azumah-23030630002-MatB23-EMT\%20Plot\%202D-059.png}
\end{figure}

Fitur serupa adalah fungsi textbox().

Lebar secara default adalah lebar maksimal baris teks. Namun, pengguna juga dapat mengaturnya.

\textgreater function f(x) \&= exp(-x)*sin(2*pi*x); \ldots{}\\
\textgreater{} plot2d(``f(x)'',0,2pi); \ldots{}\\
\textgreater{} textbox(latex(``\textbackslash text\{Example of a damped oscillation\}\textbackslash{} f(x)=e\^{}\{-x\}sin(2\textbackslash pi x)''),w=0.85):

\begin{figure}
\centering
\pandocbounded{\includegraphics[keepaspectratio]{images/Alfi Nur Azumah-23030630002-MatB23-EMT Plot 2D-060.png}}
\caption{images/Alfi\%20Nur\%20Azumah-23030630002-MatB23-EMT\%20Plot\%202D-060.png}
\end{figure}

Label teks, judul, kotak label, dan teks lainnya dapat berisi string Unicode (lihat sintaksis EMT untuk informasi lebih lanjut tentang string Unicode).

\textgreater plot2d(``x\^{}3-x'',title=u''x → x³ - x''):

\begin{figure}
\centering
\pandocbounded{\includegraphics[keepaspectratio]{images/Alfi Nur Azumah-23030630002-MatB23-EMT Plot 2D-061.png}}
\caption{images/Alfi\%20Nur\%20Azumah-23030630002-MatB23-EMT\%20Plot\%202D-061.png}
\end{figure}

Label pada sumbu x dan y dapat vertikal, begitu pula sumbunya.

\textgreater plot2d(``sinc(x)'',0,2pi,xl=``x'',yl=u''x → sinc(x)``,\textgreater vertical):

\begin{figure}
\centering
\pandocbounded{\includegraphics[keepaspectratio]{images/Alfi Nur Azumah-23030630002-MatB23-EMT Plot 2D-062.png}}
\caption{images/Alfi\%20Nur\%20Azumah-23030630002-MatB23-EMT\%20Plot\%202D-062.png}
\end{figure}

\subsection{3.8 LaTeX}\label{latex}

Anda juga dapat memplot rumus LaTeX jika Anda telah menginstal sistem LaTeX. Saya merekomendasikan MiKTeX. Jalur ke biner ``latex'' dan ``dvipng'' harus berada di jalur sistem, atau Anda harus mengatur LaTeX di menu opsi.

Perlu dicatat, bahwa penguraian LaTeX lambat. Jika Anda ingin menggunakan LaTeX dalam plot animasi, Anda harus memanggil latex() sebelum loop sekali dan menggunakan hasilnya (gambar dalam matriks RGB).

Dalam plot berikut, kami menggunakan LaTeX untuk label x dan y, label, kotak label, dan judul plot.

\textgreater plot2d(``exp(-x)*sin(x)/x'',a=0,b=2pi,c=0,d=1,grid=6,color=blue, \ldots{}\\
\textgreater{} title=latex(``\textbackslash text\{Function \(\\Phi\)\}''), \ldots{}\\
\textgreater{} xl=latex(``\textbackslash phi''),yl=latex(``\textbackslash Phi(\textbackslash phi)'')); \ldots{}\\
\textgreater{} textbox( \ldots{}\\
\textgreater{} latex(``\textbackslash Phi(\textbackslash phi) = e\^{}\{-\textbackslash phi\} \textbackslash frac\{\textbackslash sin(\textbackslash phi)\}\{\textbackslash phi\}''),x=0.8,y=0.5); \ldots{}\\
\textgreater{} label(latex(``\textbackslash Phi'',color=blue),1,0.4):

\begin{figure}
\centering
\pandocbounded{\includegraphics[keepaspectratio]{images/Alfi Nur Azumah-23030630002-MatB23-EMT Plot 2D-063.png}}
\caption{images/Alfi\%20Nur\%20Azumah-23030630002-MatB23-EMT\%20Plot\%202D-063.png}
\end{figure}

Sering kali, kita menginginkan spasi nonkonformal dan label teks pada sumbu x. Kita dapat menggunakan xaxis() dan yaxis() seperti yang akan kita tunjukkan nanti.

Cara termudah adalah membuat plot kosong dengan bingkai menggunakan grid=4, lalu menambahkan grid dengan ygrid() dan xgrid(). Dalam contoh berikut, kita menggunakan tiga string LaTeX untuk label pada sumbu x dengan xtick().

\textgreater plot2d(``sinc(x)'',0,2pi,grid=4,\textless ticks); \ldots{}\\
\textgreater{} ygrid(-2:0.5:2,grid=6); \ldots{}\\
\textgreater{} xgrid({[}0:2{]}*pi,\textless ticks,grid=6); \ldots{}\\
\textgreater{} xtick({[}0,pi,2pi{]},{[}``0'',``\textbackslash pi'',``2\textbackslash pi''{]},\textgreater latex):

\begin{figure}
\centering
\pandocbounded{\includegraphics[keepaspectratio]{images/Alfi Nur Azumah-23030630002-MatB23-EMT Plot 2D-064.png}}
\caption{images/Alfi\%20Nur\%20Azumah-23030630002-MatB23-EMT\%20Plot\%202D-064.png}
\end{figure}

Tentu saja, fungsi juga dapat digunakan.

\textgreater function map f(x) \ldots{}

\begin{verbatim}
if x>0 then return x^4
else return x^2
endif
endfunction
\end{verbatim}

Parameter ``peta'' membantu menggunakan fungsi untuk vektor. Untuk plot, parameter tersebut tidak diperlukan. Namun, untuk menunjukkan bahwa vektorisasi berguna, kami menambahkan beberapa poin penting ke plot pada x=-1, x=0, dan x=1.

Dalam plot berikut, kami juga memasukkan beberapa kode LaTeX. Kami menggunakannya untuk dua label dan kotak teks. Tentu saja, Anda hanya dapat menggunakan LaTeX jika Anda telah menginstal LaTeX dengan benar.

\textgreater plot2d(``f'',-1,1,xl=``x'',yl=``f(x)'',grid=6); \ldots{}\\
\textgreater{} plot2d({[}-1,0,1{]},f({[}-1,0,1{]}),\textgreater points,\textgreater add); \ldots{}\\
\textgreater{} label(latex(``x\^{}3''),0.72,f(0.72)); \ldots{}\\
\textgreater{} label(latex(``x\^{}2''),-0.52,f(-0.52),pos=``ll''); \ldots{}\\
\textgreater{} textbox( \ldots{}\\
\textgreater{} latex(``f(x)=\textbackslash begin\{cases\} x\^{}3 \& x\textgreater0 \textbackslash\textbackslash{} x\^{}2 \& x \textbackslash le 0\textbackslash end\{cases\}''), \ldots{}\\
\textgreater{} x=0.7,y=0.2):

\begin{figure}
\centering
\pandocbounded{\includegraphics[keepaspectratio]{images/Alfi Nur Azumah-23030630002-MatB23-EMT Plot 2D-065.png}}
\caption{images/Alfi\%20Nur\%20Azumah-23030630002-MatB23-EMT\%20Plot\%202D-065.png}
\end{figure}

\subsection{3.9 Interaksi Pengguna}\label{interaksi-pengguna}

Saat memplot fungsi atau ekspresi, parameter \textgreater user memungkinkan pengguna untuk memperbesar dan menggeser plot dengan tombol kursor atau tetikus. Pengguna dapat

\begin{enumerate}
\def\labelenumi{\arabic{enumi}.}
\tightlist
\item
  memperbesar dengan + atau -
\item
  memindahkan plot dengan tombol kursor
\item
  memilih jendela plot dengan tetikus
\item
  mengatur ulang tampilan dengan spasi
\item
  keluar dengan kembali
\end{enumerate}

Tombol spasi akan mengatur ulang plot ke jendela plot asli.

Saat memplot data, tanda \textgreater user akan menunggu penekanan tombol.

\textgreater plot2d(\{\{``x\^{}3-a*x'',a=1\}\},\textgreater user,title=``Press any key!''):

\begin{figure}
\centering
\pandocbounded{\includegraphics[keepaspectratio]{images/Alfi Nur Azumah-23030630002-MatB23-EMT Plot 2D-066.png}}
\caption{images/Alfi\%20Nur\%20Azumah-23030630002-MatB23-EMT\%20Plot\%202D-066.png}
\end{figure}

\textgreater plot2d(``exp(x)*sin(x)'',user=true, \ldots{}\\
\textgreater{} title=``+/- or cursor keys (return to exit)''):

\begin{figure}
\centering
\pandocbounded{\includegraphics[keepaspectratio]{images/Alfi Nur Azumah-23030630002-MatB23-EMT Plot 2D-067.png}}
\caption{images/Alfi\%20Nur\%20Azumah-23030630002-MatB23-EMT\%20Plot\%202D-067.png}
\end{figure}

Berikut ini menunjukkan cara interaksi pengguna tingkat lanjut (lihat tutorial tentang pemrograman untuk detailnya).

Fungsi bawaan mousedrag() menunggu kejadian tetikus atau papan ketik. Fungsi ini melaporkan gerakan tetikus ke bawah, gerakan tetikus ke atas, dan penekanan tombol. Fungsi dragpoints() memanfaatkan fungsi ini, dan memungkinkan pengguna menyeret titik mana pun dalam plot.

Pertama-tama, kita memerlukan fungsi plot. Misalnya, kita melakukan interpolasi pada 5 titik dengan polinomial. Fungsi ini harus memplot ke dalam area plot yang tetap.

\textgreater function plotf(xp,yp,select) \ldots{}

\begin{verbatim}
  d=interp(xp,yp);
  plot2d("interpval(xp,d,x)";d,xp,r=2);
  plot2d(xp,yp,>points,>add);
  if select>0 then
    plot2d(xp[select],yp[select],color=red,>points,>add);
  endif;
  title("Drag one point, or press space or return!");
endfunction
\end{verbatim}

Perhatikan parameter titik koma di plot2d (d dan xp), yang diteruskan ke evaluasi fungsi interp(). Tanpa ini, kita harus menulis fungsi plotinterp() terlebih dahulu, mengakses nilai secara global.

Sekarang kita buat beberapa nilai acak, dan biarkan pengguna menyeret titik-titiknya.

\textgreater t=-1:0.5:1; dragpoints(``plotf'',t,random(size(t))-0.5):

\begin{figure}
\centering
\pandocbounded{\includegraphics[keepaspectratio]{images/Alfi Nur Azumah-23030630002-MatB23-EMT Plot 2D-068.png}}
\caption{images/Alfi\%20Nur\%20Azumah-23030630002-MatB23-EMT\%20Plot\%202D-068.png}
\end{figure}

Ada juga fungsi yang memplot fungsi lain tergantung pada vektor parameter dan memungkinkan pengguna menyesuaikan parameter tersebut.

Pertama, kita perlu fungsi plot.

\textgreater function plotf({[}a,b{]}) := plot2d(``exp(a*x)*cos(2pi*b*x)'',0,2pi;a,b);

Kemudian kita perlu nama untuk parameter, nilai awal, dan matriks rentang nx2, secara opsional baris judul.

Ada slider interaktif, yang dapat mengatur nilai oleh pengguna. Fungsi dragvalues() menyediakan ini.

\textgreater dragvalues(``plotf'',{[}``a'',``b''{]},{[}-1,2{]},{[}{[}-2,2{]};{[}1,10{]}{]}, \ldots{}\\
\textgreater{} heading=``Drag these values:'',hcolor=black):

\begin{figure}
\centering
\pandocbounded{\includegraphics[keepaspectratio]{images/Alfi Nur Azumah-23030630002-MatB23-EMT Plot 2D-069.png}}
\caption{images/Alfi\%20Nur\%20Azumah-23030630002-MatB23-EMT\%20Plot\%202D-069.png}
\end{figure}

Dimungkinkan untuk membatasi nilai yang diseret ke bilangan bulat. Misalnya, kita menulis fungsi plot, yang memplot polinomial Taylor berderajat n ke fungsi kosinus.

\textgreater function plotf(n) \ldots{}

\begin{verbatim}
plot2d("cos(x)",0,2pi,>square,grid=6);
plot2d(&"taylor(cos(x),x,0,@n)",color=blue,>add);
textbox("Taylor polynomial of degree "+n,0.1,0.02,style="t",>left);
endfunction
\end{verbatim}

Sekarang kita biarkan derajat n bervariasi dari 0 hingga 20 dalam 20 stop. Hasil dragvalues() digunakan untuk memplot sketsa dengan n ini, dan untuk memasukkan plot ke dalam notebook.

\textgreater nd=dragvalues(``plotf'',``degree'',2,{[}0,20{]},20,y=0.8, \ldots{}\\
\textgreater{} heading=``Drag the value:''); \ldots{}\\
\textgreater{} plotf(nd):

\begin{figure}
\centering
\pandocbounded{\includegraphics[keepaspectratio]{images/Alfi Nur Azumah-23030630002-MatB23-EMT Plot 2D-070.png}}
\caption{images/Alfi\%20Nur\%20Azumah-23030630002-MatB23-EMT\%20Plot\%202D-070.png}
\end{figure}

Berikut ini adalah demonstrasi sederhana dari fungsi tersebut. Pengguna dapat menggambar di atas jendela plot, meninggalkan jejak titik-titik.

\textgreater function dragtest \ldots{}

\begin{verbatim}
  plot2d(none,r=1,title="Drag with the mouse, or press any key!");
  start=0;
  repeat
    {flag,m,time}=mousedrag();
    if flag==0 then return; endif;
    if flag==2 then
      hold on; mark(m[1],m[2]); hold off;
    endif;
  end
endfunction
\end{verbatim}

\textgreater dragtest // lihat hasilnya dan cobalah lakukan!

\subsection{3.10 Gaya Plot 2D}\label{gaya-plot-2d}

Secara default, EMT menghitung tanda centang sumbu otomatis dan menambahkan label ke setiap tanda centang. Ini dapat diubah dengan parameter grid. Gaya default sumbu dan label dapat dimodifikasi. Selain itu, label dan judul dapat ditambahkan secara manual. Untuk mengatur ulang ke gaya default, gunakan reset().

\textgreater aspect();

\textgreater figure(3,4); \ldots{}\\
\textgreater{} figure(1); plot2d(``x\^{}3-x'',grid=0); \ldots{} // tidak ada tampilan, bingkai dan sumbu

\textgreater{} figure(2); plot2d(``x\^{}3-x'',grid=1); \ldots{} // terdapat sumbu x-y

\textgreater{} figure(3); plot2d(``x\^{}3-x'',grid=2); \ldots{} // terdapat penanda kecil otomatis

\textgreater{} figure(4); plot2d(``x\^{}3-x'',grid=3); \ldots{} // sumbu x-y dengan label di dalamnya

\textgreater{} figure(5); plot2d(``x\^{}3-x'',grid=4); \ldots{} // tidak ada penanda hanya label

\textgreater{} figure(6); plot2d(``x\^{}3-x'',grid=5); \ldots{} // default tapi tidak ada margin

\textgreater{} figure(7); plot2d(``x\^{}3-x'',grid=6); \ldots{} // hanya sumbu dan penanda kecil

\textgreater{} figure(8); plot2d(``x\^{}3-x'',grid=7); \ldots{} // hanya sumbu dan penanda kecil pada sumbu

\textgreater{} figure(9); plot2d(``x\^{}3-x'',grid=8); \ldots{} // hanya sumbu dan penanda kecil terperinci pada sumbu tertentu

\textgreater{} figure(10); plot2d(``x\^{}3-x'',grid=9); \ldots{} // default dengan penanda penanda kecil di dalamnya

\textgreater{} figure(11); plot2d(``x\^{}3-x'',grid=10); \ldots// tidak ada penanda kecil, hanya sumbu

\textgreater{} figure(0):

\begin{figure}
\centering
\pandocbounded{\includegraphics[keepaspectratio]{images/Alfi Nur Azumah-23030630002-MatB23-EMT Plot 2D-071.png}}
\caption{images/Alfi\%20Nur\%20Azumah-23030630002-MatB23-EMT\%20Plot\%202D-071.png}
\end{figure}

Parameter \textless frame menonaktifkan bingkai, dan framecolor=blue menyetel bingkai ke warna biru.

Jika Anda menginginkan tanda centang Anda sendiri, Anda dapat menggunakan style=0, dan menambahkan semuanya nanti.

\textgreater aspect(1.5);

\textgreater plot2d(``x\^{}3-x'',grid=0); // plot

\textgreater frame; xgrid({[}-1,0,1{]}); ygrid(0): // add frame and grid

\begin{figure}
\centering
\pandocbounded{\includegraphics[keepaspectratio]{images/Alfi Nur Azumah-23030630002-MatB23-EMT Plot 2D-072.png}}
\caption{images/Alfi\%20Nur\%20Azumah-23030630002-MatB23-EMT\%20Plot\%202D-072.png}
\end{figure}

Untuk judul plot dan label sumbu, lihat contoh berikut.

\textgreater plot2d(``exp(x)'',-1,1);

\textgreater textcolor(black); // set the text color to black

\textgreater title(latex(``y=e\^{}x'')); // title above the plot

\textgreater xlabel(latex(``x'')); // ``x'' for x-axis

\textgreater ylabel(latex(``y''),\textgreater vertical); // vertical ``y'' for y-axis

\textgreater label(latex(``(0,1)''),0,1,color=blue): // label a point

\begin{figure}
\centering
\pandocbounded{\includegraphics[keepaspectratio]{images/Alfi Nur Azumah-23030630002-MatB23-EMT Plot 2D-073.png}}
\caption{images/Alfi\%20Nur\%20Azumah-23030630002-MatB23-EMT\%20Plot\%202D-073.png}
\end{figure}

Sumbu dapat digambar secara terpisah dengan xaxis() dan yaxis().

\textgreater plot2d(``x\^{}3-x'',\textless grid,\textless frame);

\textgreater xaxis(0,xx=-2:1,style=``-\textgreater{}''); yaxis(0,yy=-5:5,style=``-\textgreater{}''):

\begin{figure}
\centering
\pandocbounded{\includegraphics[keepaspectratio]{images/Alfi Nur Azumah-23030630002-MatB23-EMT Plot 2D-074.png}}
\caption{images/Alfi\%20Nur\%20Azumah-23030630002-MatB23-EMT\%20Plot\%202D-074.png}
\end{figure}

Teks pada plot dapat diatur dengan label(). Dalam contoh berikut, ``lc'' berarti tengah bawah. Ini mengatur posisi label relatif terhadap koordinat plot.

\textgreater function f(x) \&= x\^{}3-x

\begin{verbatim}
                                 3
                                x  - x
\end{verbatim}

\textgreater plot2d(f,-1,1,\textgreater square);

\textgreater x0=fmin(f,0,1); // compute point of minimum

\textgreater label(``Rel. Min.'',x0,f(x0),pos=``lc''): // add a label there

\begin{figure}
\centering
\pandocbounded{\includegraphics[keepaspectratio]{images/Alfi Nur Azumah-23030630002-MatB23-EMT Plot 2D-075.png}}
\caption{images/Alfi\%20Nur\%20Azumah-23030630002-MatB23-EMT\%20Plot\%202D-075.png}
\end{figure}

Ada juga teks box.

\textgreater plot2d(\&f(x),-1,1,-2,2); // function

\textgreater plot2d(\&diff(f(x),x),\textgreater add,style=``--'',color=red); // derivative

\textgreater labelbox({[}``f'',``f'''{]},{[}``-'',``--''{]},{[}black,red{]}): // label box

\begin{figure}
\centering
\pandocbounded{\includegraphics[keepaspectratio]{images/Alfi Nur Azumah-23030630002-MatB23-EMT Plot 2D-076.png}}
\caption{images/Alfi\%20Nur\%20Azumah-23030630002-MatB23-EMT\%20Plot\%202D-076.png}
\end{figure}

\textgreater plot2d({[}``exp(x)'',``1+x''{]},color={[}black,blue{]},style={[}``-'',``-.-''{]}):

\begin{figure}
\centering
\pandocbounded{\includegraphics[keepaspectratio]{images/Alfi Nur Azumah-23030630002-MatB23-EMT Plot 2D-077.png}}
\caption{images/Alfi\%20Nur\%20Azumah-23030630002-MatB23-EMT\%20Plot\%202D-077.png}
\end{figure}

\textgreater gridstyle(``-\textgreater{}'',color=gray,textcolor=gray,framecolor=gray); \ldots{}\\
\textgreater{} plot2d(``x\^{}3-x'',grid=1); \ldots{}\\
\textgreater{} settitle(``y=x\^{}3-x'',color=black); \ldots{}\\
\textgreater{} label(``x'',2,0,pos=``bc'',color=gray); \ldots{}\\
\textgreater{} label(``y'',0,6,pos=``cl'',color=gray); \ldots{}\\
\textgreater{} reset():

\begin{figure}
\centering
\pandocbounded{\includegraphics[keepaspectratio]{images/Alfi Nur Azumah-23030630002-MatB23-EMT Plot 2D-078.png}}
\caption{images/Alfi\%20Nur\%20Azumah-23030630002-MatB23-EMT\%20Plot\%202D-078.png}
\end{figure}

Untuk kontrol yang lebih baik, sumbu x dan sumbu y dapat dilakukan secara manual.

Perintah fullwindow() memperluas jendela plot karena kita tidak lagi memerlukan tempat untuk label di luar jendela plot. Gunakan shrinkwindow() atau reset() untuk mengatur ulang ke default.

\textgreater fullwindow; \ldots{}\\
\textgreater{} gridstyle(color=darkgray,textcolor=darkgray); \ldots{}\\
\textgreater{} plot2d({[}``2\textsuperscript{x'',''1'',''2}(-x)''{]},a=-2,b=2,c=0,d=4,\textless grid,color=4:6,\textless frame); \ldots{}\\
\textgreater{} xaxis(0,-2:1,style=``-\textgreater{}''); xaxis(0,2,``x'',\textless axis); \ldots{}\\
\textgreater{} yaxis(0,4,``y'',style=``-\textgreater{}''); \ldots{}\\
\textgreater{} yaxis(-2,1:4,\textgreater left); \ldots{}\\
\textgreater{} yaxis(2,2\^{}(-2:2),style=``.'',\textless left); \ldots{}\\
\textgreater{} labelbox({[}``2\textsuperscript{x'',''1'',''2}-x''{]},colors=4:6,x=0.8,y=0.2); \ldots{}\\
\textgreater{} reset:

\begin{figure}
\centering
\pandocbounded{\includegraphics[keepaspectratio]{images/Alfi Nur Azumah-23030630002-MatB23-EMT Plot 2D-079.png}}
\caption{images/Alfi\%20Nur\%20Azumah-23030630002-MatB23-EMT\%20Plot\%202D-079.png}
\end{figure}

Berikut contoh lain, di mana string Unicode digunakan dan sumbu berada di luar area plot.

\textgreater aspect(1.5);

\textgreater plot2d({[}``sin(x)'',``cos(x)''{]},0,2pi,color={[}red,green{]},\textless grid,\textless frame); \ldots{}\\
\textgreater{} xaxis(-1.1,(0:2)*pi,xt={[}``0'',u''π``,u''2π''{]},style=``-'',\textgreater ticks,\textgreater zero); \ldots{}\\
\textgreater{} xgrid((0:0.5:2)*pi,\textless ticks); \ldots{}\\
\textgreater{} yaxis(-0.1*pi,-1:0.2:1,style=``-'',\textgreater zero,\textgreater grid); \ldots{}\\
\textgreater{} labelbox({[}``sin'',``cos''{]},colors={[}red,green{]},x=0.5,y=0.2,\textgreater left); \ldots{}\\
\textgreater{} xlabel(u''φ``); ylabel(u''f(φ)``):

\begin{figure}
\centering
\pandocbounded{\includegraphics[keepaspectratio]{images/Alfi Nur Azumah-23030630002-MatB23-EMT Plot 2D-080.png}}
\caption{images/Alfi\%20Nur\%20Azumah-23030630002-MatB23-EMT\%20Plot\%202D-080.png}
\end{figure}

\subsection{3.11 Plotting Data 2D}\label{plotting-data-2d}

Jika x dan y adalah vektor data, data ini akan digunakan sebagai koordinat x dan y dari sebuah kurva. Dalam kasus ini, a, b, c, dan d, atau radius r dapat ditentukan, atau jendela plot akan menyesuaikan secara otomatis dengan data. Atau, \textgreater square dapat diatur untuk mempertahankan rasio aspek persegi.

Plotting ekspresi hanyalah singkatan untuk plot data. Untuk plot data, Anda memerlukan satu atau beberapa baris nilai-x, dan satu atau beberapa baris nilai-y. Dari rentang dan nilai-x, fungsi plot2d akan menghitung data yang akan diplot, secara default dengan evaluasi adaptif fungsi. Untuk plot titik gunakan ``\textgreater points'', untuk garis dan titik campuran gunakan ``\textgreater addpoints''.

Namun, Anda dapat memasukkan data secara langsung.

\begin{itemize}
\tightlist
\item
  Gunakan vektor baris untuk x dan y untuk satu fungsi.
\item
  Matriks untuk x dan y diplot baris demi baris.
\end{itemize}

Berikut adalah contoh dengan satu baris untuk x dan y.

\textgreater x=-10:0.1:10; y=exp(-x\^{}2)*x; plot2d(x,y):

\begin{figure}
\centering
\pandocbounded{\includegraphics[keepaspectratio]{images/Alfi Nur Azumah-23030630002-MatB23-EMT Plot 2D-081.png}}
\caption{images/Alfi\%20Nur\%20Azumah-23030630002-MatB23-EMT\%20Plot\%202D-081.png}
\end{figure}

Data juga dapat diplot sebagai titik. Gunakan points=true untuk ini. Plot bekerja seperti poligon, tetapi hanya menggambar sudutnya.

\begin{itemize}
\tightlist
\item
  style=``\ldots{}'': Pilih dari ``{[}{]}'', ``\textless\textgreater{}'', ``o'', ``.'', ``..'', ``+'', ``*``,''{[}{]}\#``,
\item
  ``\textless\textgreater\#'', ``o\#'', ``..\#'', ``\#'', ``\textbar{}''.
\end{itemize}

Untuk memplot kumpulan titik, gunakan \textgreater points. Jika warnanya adalah vektor warna, setiap titik

mendapatkan warna yang berbeda. Untuk matriks koordinat dan vektor kolom, warna berlaku untuk baris matriks.

Parameter \textgreater addpoints menambahkan titik ke segmen garis untuk plot data.

\textgreater xdata={[}1,1.5,2.5,3,4{]}; ydata={[}3,3.1,2.8,2.9,2.7{]}; // data

\textgreater plot2d(xdata,ydata,a=0.5,b=4.5,c=2.5,d=3.5,style=``.''); // lines

\textgreater plot2d(xdata,ydata,\textgreater points,\textgreater add,style=``o''): // add points

\begin{figure}
\centering
\pandocbounded{\includegraphics[keepaspectratio]{images/Alfi Nur Azumah-23030630002-MatB23-EMT Plot 2D-082.png}}
\caption{images/Alfi\%20Nur\%20Azumah-23030630002-MatB23-EMT\%20Plot\%202D-082.png}
\end{figure}

\textgreater p=polyfit(xdata,ydata,1); // get regression line

\textgreater plot2d(``polyval(p,x)'',\textgreater add,color=red): // add plot of line

\begin{figure}
\centering
\pandocbounded{\includegraphics[keepaspectratio]{images/Alfi Nur Azumah-23030630002-MatB23-EMT Plot 2D-083.png}}
\caption{images/Alfi\%20Nur\%20Azumah-23030630002-MatB23-EMT\%20Plot\%202D-083.png}
\end{figure}

\subsection{3.12 Menggambar Daerah Yang Berbatasan Kurva}\label{menggambar-daerah-yang-berbatasan-kurva}

Plot data sebenarnya adalah poligon. Kita juga dapat memplot kurva atau kurva terisi.

\begin{itemize}
\tightlist
\item
  fillex=benar mengisi plot.
\item
  style=``\ldots{}'': Pilih dari ``\#'', ``/'', ``",''/``.
\item
  fillcolor: Lihat di atas untuk warna yang tersedia.
\end{itemize}

Warna isian ditentukan oleh argumen ``fillcolor'', dan pada \textless outline opsional mencegah penggambaran batas untuk semua gaya kecuali yang default.

\textgreater t=linspace(0,2pi,1000); // parameter for curve

\textgreater x=sin(t)*exp(t/pi); y=cos(t)*exp(t/pi); // x(t) and y(t)

\textgreater figure(1,2); aspect(16/9)

\textgreater figure(1); plot2d(x,y,r=10); // plot curve

\textgreater figure(2); plot2d(x,y,r=10,\textgreater filled,style=``/'',fillcolor=red); // fill curve

\textgreater figure(0):

\begin{figure}
\centering
\pandocbounded{\includegraphics[keepaspectratio]{images/Alfi Nur Azumah-23030630002-MatB23-EMT Plot 2D-084.png}}
\caption{images/Alfi\%20Nur\%20Azumah-23030630002-MatB23-EMT\%20Plot\%202D-084.png}
\end{figure}

Pada contoh berikut ini kami memplot elips yang terisi dan dua segi enam yang terisi menggunakan kurva tertutup dengan 6 titik dengan gaya isian yang berbeda.

\textgreater x=linspace(0,2pi,1000); plot2d(sin(x),cos(x)*0.5,r=1,\textgreater filled,style=``/''):

\begin{figure}
\centering
\pandocbounded{\includegraphics[keepaspectratio]{images/Alfi Nur Azumah-23030630002-MatB23-EMT Plot 2D-085.png}}
\caption{images/Alfi\%20Nur\%20Azumah-23030630002-MatB23-EMT\%20Plot\%202D-085.png}
\end{figure}

\textgreater t=linspace(0,2pi,6); \ldots{}\\
\textgreater{} plot2d(cos(t),sin(t),\textgreater filled,style=``/'',fillcolor=red,r=1.2):

\begin{figure}
\centering
\pandocbounded{\includegraphics[keepaspectratio]{images/Alfi Nur Azumah-23030630002-MatB23-EMT Plot 2D-086.png}}
\caption{images/Alfi\%20Nur\%20Azumah-23030630002-MatB23-EMT\%20Plot\%202D-086.png}
\end{figure}

\textgreater t=linspace(0,2pi,6); plot2d(cos(t),sin(t),\textgreater filled,style=``\#''):

\begin{figure}
\centering
\pandocbounded{\includegraphics[keepaspectratio]{images/Alfi Nur Azumah-23030630002-MatB23-EMT Plot 2D-087.png}}
\caption{images/Alfi\%20Nur\%20Azumah-23030630002-MatB23-EMT\%20Plot\%202D-087.png}
\end{figure}

Contoh lain adalah septagon, yang kita buat dengan 7 titik pada lingkaran satuan.

\textgreater t=linspace(0,2pi,7); \ldots{}\\
\textgreater{} plot2d(cos(t),sin(t),r=1,\textgreater filled,style=``/'',fillcolor=red):

\begin{figure}
\centering
\pandocbounded{\includegraphics[keepaspectratio]{images/Alfi Nur Azumah-23030630002-MatB23-EMT Plot 2D-088.png}}
\caption{images/Alfi\%20Nur\%20Azumah-23030630002-MatB23-EMT\%20Plot\%202D-088.png}
\end{figure}

Berikut ini adalah himpunan nilai maksimal dari empat kondisi linier yang kurang dari atau sama dengan 3. Ini adalah A{[}k{]}.v\textless=3 untuk semua baris A. Untuk mendapatkan sudut yang bagus, kita menggunakan n yang relatif besar.

\textgreater A={[}2,1;1,2;-1,0;0,-1{]}

\begin{verbatim}
            2             1 
            1             2 
           -1             0 
            0            -1 
\end{verbatim}

\textgreater function f(x,y) := max({[}x,y{]}.A');

\textgreater plot2d(``f'',r=4,level={[}0;3{]},color=green,n=111):

\begin{figure}
\centering
\pandocbounded{\includegraphics[keepaspectratio]{images/Alfi Nur Azumah-23030630002-MatB23-EMT Plot 2D-089.png}}
\caption{images/Alfi\%20Nur\%20Azumah-23030630002-MatB23-EMT\%20Plot\%202D-089.png}
\end{figure}

Poin utama dari bahasa matriks adalah memungkinkan pembuatan tabel fungsi dengan mudah.

\textgreater t=linspace(0,2pi,1000); x=cos(3*t); y=sin(4*t);

Sekarang kita memiliki vektor nilai x dan y. plot2d() dapat memplot nilai-nilai ini sebagai kurva yang menghubungkan titik-titik. Plot dapat diisi. Dalam kasus ini, hasilnya akan bagus karena aturan lilitan, yang digunakan untuk pengisian.

\textgreater plot2d(x,y,\textless grid,\textless frame,\textgreater filled):

\begin{figure}
\centering
\pandocbounded{\includegraphics[keepaspectratio]{images/Alfi Nur Azumah-23030630002-MatB23-EMT Plot 2D-090.png}}
\caption{images/Alfi\%20Nur\%20Azumah-23030630002-MatB23-EMT\%20Plot\%202D-090.png}
\end{figure}

Vektor interval diplot terhadap nilai x sebagai daerah terisi antara nilai interval yang lebih rendah dan lebih tinggi.

Hal ini dapat berguna untuk memplot kesalahan perhitungan. Namun, hal ini juga dapat digunakan untuk memplot kesalahan statistik.

\textgreater t=0:0.1:1; \ldots{}\\
\textgreater{} plot2d(t,interval(t-random(size(t)),t+random(size(t))),style=``\textbar{}''); \ldots{}\\
\textgreater{} plot2d(t,t,add=true):

\begin{figure}
\centering
\pandocbounded{\includegraphics[keepaspectratio]{images/Alfi Nur Azumah-23030630002-MatB23-EMT Plot 2D-091.png}}
\caption{images/Alfi\%20Nur\%20Azumah-23030630002-MatB23-EMT\%20Plot\%202D-091.png}
\end{figure}

Jika x adalah vektor yang diurutkan, dan y adalah vektor interval, maka plot2d akan memplot rentang interval yang terisi pada bidang. Gaya isiannya sama dengan gaya poligon.

\textgreater t=-1:0.01:1; x=\textsubscript{t-0.01,t+0.01}; y=x\^{}3-x;

\textgreater plot2d(t,y):

\begin{figure}
\centering
\pandocbounded{\includegraphics[keepaspectratio]{images/Alfi Nur Azumah-23030630002-MatB23-EMT Plot 2D-092.png}}
\caption{images/Alfi\%20Nur\%20Azumah-23030630002-MatB23-EMT\%20Plot\%202D-092.png}
\end{figure}

Dimungkinkan untuk mengisi wilayah nilai untuk fungsi tertentu. Untuk ini, level harus berupa matriks 2xn. Baris pertama adalah batas bawah dan baris kedua berisi batas atas.

\textgreater expr := ``2*x\textsuperscript{2+x*y+3*y}4+y''; // define an expression f(x,y)

\textgreater plot2d(expr,level={[}0;1{]},style=``-'',color=blue): // 0 \textless= f(x,y) \textless= 1

\begin{figure}
\centering
\pandocbounded{\includegraphics[keepaspectratio]{images/Alfi Nur Azumah-23030630002-MatB23-EMT Plot 2D-093.png}}
\caption{images/Alfi\%20Nur\%20Azumah-23030630002-MatB23-EMT\%20Plot\%202D-093.png}
\end{figure}

Kita juga dapat mengisi rentang nilai seperti

\textgreater plot2d(``(x\textsuperscript{2+y}2)\textsuperscript{2-x}2+y\^{}2'',r=1.2,level={[}-1;0{]},style=``/''):

\begin{figure}
\centering
\pandocbounded{\includegraphics[keepaspectratio]{images/Alfi Nur Azumah-23030630002-MatB23-EMT Plot 2D-094.png}}
\caption{images/Alfi\%20Nur\%20Azumah-23030630002-MatB23-EMT\%20Plot\%202D-094.png}
\end{figure}

\textgreater plot2d(``cos(x)'',``sin(x)\^{}3'',xmin=0,xmax=2pi,\textgreater filled,style=``/''):

\begin{figure}
\centering
\pandocbounded{\includegraphics[keepaspectratio]{images/Alfi Nur Azumah-23030630002-MatB23-EMT Plot 2D-095.png}}
\caption{images/Alfi\%20Nur\%20Azumah-23030630002-MatB23-EMT\%20Plot\%202D-095.png}
\end{figure}

\subsection{3.13 Grafik Fungsi Parametrik}\label{grafik-fungsi-parametrik}

Nilai-nilai x tidak perlu diurutkan. (x,y) hanya menggambarkan sebuah kurva. Jika x diurutkan, kurva tersebut adalah grafik dari sebuah fungsi.

Dalam contoh berikut, kita memplot spiral

Kita perlu menggunakan banyak titik agar terlihat halus atau fungsi adaptive() untuk mengevaluasi ekspresi (lihat fungsi adaptive() untuk detail lebih lanjut).

\textgreater t=linspace(0,1,1000); \ldots{}\\
\textgreater{} plot2d(t*cos(2*pi*t),t*sin(2*pi*t),r=1):

\begin{figure}
\centering
\pandocbounded{\includegraphics[keepaspectratio]{images/Alfi Nur Azumah-23030630002-MatB23-EMT Plot 2D-096.png}}
\caption{images/Alfi\%20Nur\%20Azumah-23030630002-MatB23-EMT\%20Plot\%202D-096.png}
\end{figure}

Atau, ada kemungkinan untuk menggunakan dua ekspresi untuk kurva. Berikut ini memplot kurva yang sama seperti di atas.

\textgreater plot2d(``x*cos(2*pi*x)'',``x*sin(2*pi*x)'',xmin=0,xmax=1,r=1):

\begin{figure}
\centering
\pandocbounded{\includegraphics[keepaspectratio]{images/Alfi Nur Azumah-23030630002-MatB23-EMT Plot 2D-097.png}}
\caption{images/Alfi\%20Nur\%20Azumah-23030630002-MatB23-EMT\%20Plot\%202D-097.png}
\end{figure}

\textgreater t=linspace(0,1,1000); r=exp(-t); x=r*cos(2pi*t); y=r*sin(2pi*t);

\textgreater plot2d(x,y,r=1):

\begin{figure}
\centering
\pandocbounded{\includegraphics[keepaspectratio]{images/Alfi Nur Azumah-23030630002-MatB23-EMT Plot 2D-098.png}}
\caption{images/Alfi\%20Nur\%20Azumah-23030630002-MatB23-EMT\%20Plot\%202D-098.png}
\end{figure}

Pada contoh berikutnya, kita memplot kurva dengan

\textgreater t=linspace(0,2pi,1000); r=1+sin(3*t)/2; x=r*cos(t); y=r*sin(t); \ldots{}\\
\textgreater{} plot2d(x,y,\textgreater filled,fillcolor=red,style=``/'',r=1.5):

\begin{figure}
\centering
\pandocbounded{\includegraphics[keepaspectratio]{images/Alfi Nur Azumah-23030630002-MatB23-EMT Plot 2D-099.png}}
\caption{images/Alfi\%20Nur\%20Azumah-23030630002-MatB23-EMT\%20Plot\%202D-099.png}
\end{figure}

\subsection{3.14 Menggambar Grafik Bilangan Kompleks}\label{menggambar-grafik-bilangan-kompleks}

Rangkaian bilangan kompleks juga dapat diplot. Kemudian titik-titik grid akan dihubungkan. Jika sejumlah garis grid ditentukan (atau vektor garis grid 1x2) dalam argumen cgrid, hanya garis-garis grid tersebut yang terlihat.

Matriks bilangan kompleks akan secara otomatis diplot sebagai grid dalam bidang kompleks.

Dalam contoh berikut, kami memplot gambar lingkaran satuan di bawah fungsi eksponensial. Parameter cgrid menyembunyikan beberapa kurva grid.

\textgreater aspect(); r=linspace(0,1,50); a=linspace(0,2pi,80)'; z=r*exp(I*a);\ldots{}\\
\textgreater{} plot2d(z,a=-1.25,b=1.25,c=-1.25,d=1.25,cgrid=10):

\begin{figure}
\centering
\pandocbounded{\includegraphics[keepaspectratio]{images/Alfi Nur Azumah-23030630002-MatB23-EMT Plot 2D-100.png}}
\caption{images/Alfi\%20Nur\%20Azumah-23030630002-MatB23-EMT\%20Plot\%202D-100.png}
\end{figure}

\textgreater aspect(1.25); r=linspace(0,1,50); a=linspace(0,2pi,200)'; z=r*exp(I*a);

\textgreater plot2d(exp(z),cgrid={[}40,10{]}):

\begin{figure}
\centering
\pandocbounded{\includegraphics[keepaspectratio]{images/Alfi Nur Azumah-23030630002-MatB23-EMT Plot 2D-101.png}}
\caption{images/Alfi\%20Nur\%20Azumah-23030630002-MatB23-EMT\%20Plot\%202D-101.png}
\end{figure}

\textgreater r=linspace(0,1,10); a=linspace(0,2pi,40)'; z=r*exp(I*a);

\textgreater plot2d(exp(z),\textgreater points,\textgreater add):

\begin{figure}
\centering
\pandocbounded{\includegraphics[keepaspectratio]{images/Alfi Nur Azumah-23030630002-MatB23-EMT Plot 2D-102.png}}
\caption{images/Alfi\%20Nur\%20Azumah-23030630002-MatB23-EMT\%20Plot\%202D-102.png}
\end{figure}

Vektor bilangan kompleks secara otomatis diplot sebagai kurva pada bidang kompleks dengan bagian riil dan bagian imajiner.

Dalam contoh, kita memplot lingkaran satuan dengan

\textgreater t=linspace(0,2pi,1000); \ldots{}\\
\textgreater{} plot2d(exp(I*t)+exp(4*I*t),r=2):

\begin{figure}
\centering
\pandocbounded{\includegraphics[keepaspectratio]{images/Alfi Nur Azumah-23030630002-MatB23-EMT Plot 2D-103.png}}
\caption{images/Alfi\%20Nur\%20Azumah-23030630002-MatB23-EMT\%20Plot\%202D-103.png}
\end{figure}

\subsection{3.15 Plot Statistik}\label{plot-statistik}

Ada banyak fungsi yang dikhususkan pada plot statistik. Salah satu plot yang sering digunakan adalah plot kolom.

Penjumlahan kumulatif dari nilai-nilai berdistribusi normal 0-1 menghasilkan pergerakan acak.

\textgreater plot2d(cumsum(randnormal(1,1000))):

\begin{figure}
\centering
\pandocbounded{\includegraphics[keepaspectratio]{images/Alfi Nur Azumah-23030630002-MatB23-EMT Plot 2D-104.png}}
\caption{images/Alfi\%20Nur\%20Azumah-23030630002-MatB23-EMT\%20Plot\%202D-104.png}
\end{figure}

Menggunakan dua baris menunjukkan jalan dalam dua dimensi.

\textgreater X=cumsum(randnormal(2,1000)); plot2d(X{[}1{]},X{[}2{]}):

\begin{figure}
\centering
\pandocbounded{\includegraphics[keepaspectratio]{images/Alfi Nur Azumah-23030630002-MatB23-EMT Plot 2D-105.png}}
\caption{images/Alfi\%20Nur\%20Azumah-23030630002-MatB23-EMT\%20Plot\%202D-105.png}
\end{figure}

\textgreater columnsplot(cumsum(random(10)),style=``/'',color=blue):

\begin{figure}
\centering
\pandocbounded{\includegraphics[keepaspectratio]{images/Alfi Nur Azumah-23030630002-MatB23-EMT Plot 2D-106.png}}
\caption{images/Alfi\%20Nur\%20Azumah-23030630002-MatB23-EMT\%20Plot\%202D-106.png}
\end{figure}

Ia juga dapat menampilkan string sebagai label.

\textgreater months={[}``Jan'',``Feb'',``Mar'',``Apr'',``May'',``Jun'', \ldots{}\\
\textgreater{} ``Jul'',``Aug'',``Sep'',``Oct'',``Nov'',``Dec''{]};

\textgreater values={[}10,12,12,18,22,28,30,26,22,18,12,8{]};

\textgreater columnsplot(values,lab=months,color=red,style=``-'');

\textgreater title(``Temperature''):

\begin{figure}
\centering
\pandocbounded{\includegraphics[keepaspectratio]{images/Alfi Nur Azumah-23030630002-MatB23-EMT Plot 2D-107.png}}
\caption{images/Alfi\%20Nur\%20Azumah-23030630002-MatB23-EMT\%20Plot\%202D-107.png}
\end{figure}

\textgreater k=0:10;

\textgreater plot2d(k,bin(10,k),\textgreater bar):

\begin{figure}
\centering
\pandocbounded{\includegraphics[keepaspectratio]{images/Alfi Nur Azumah-23030630002-MatB23-EMT Plot 2D-108.png}}
\caption{images/Alfi\%20Nur\%20Azumah-23030630002-MatB23-EMT\%20Plot\%202D-108.png}
\end{figure}

\textgreater plot2d(k,bin(10,k)); plot2d(k,bin(10,k),\textgreater points,\textgreater add):

\begin{figure}
\centering
\pandocbounded{\includegraphics[keepaspectratio]{images/Alfi Nur Azumah-23030630002-MatB23-EMT Plot 2D-109.png}}
\caption{images/Alfi\%20Nur\%20Azumah-23030630002-MatB23-EMT\%20Plot\%202D-109.png}
\end{figure}

\textgreater plot2d(normal(1000),normal(1000),\textgreater points,grid=6,style=``..''):

\begin{figure}
\centering
\pandocbounded{\includegraphics[keepaspectratio]{images/Alfi Nur Azumah-23030630002-MatB23-EMT Plot 2D-110.png}}
\caption{images/Alfi\%20Nur\%20Azumah-23030630002-MatB23-EMT\%20Plot\%202D-110.png}
\end{figure}

\textgreater plot2d(normal(1,1000),\textgreater distribution,style=``O''):

\begin{figure}
\centering
\pandocbounded{\includegraphics[keepaspectratio]{images/Alfi Nur Azumah-23030630002-MatB23-EMT Plot 2D-111.png}}
\caption{images/Alfi\%20Nur\%20Azumah-23030630002-MatB23-EMT\%20Plot\%202D-111.png}
\end{figure}

\textgreater plot2d(``qnormal'',0,5;2.5,0.5,\textgreater filled):

\begin{figure}
\centering
\pandocbounded{\includegraphics[keepaspectratio]{images/Alfi Nur Azumah-23030630002-MatB23-EMT Plot 2D-112.png}}
\caption{images/Alfi\%20Nur\%20Azumah-23030630002-MatB23-EMT\%20Plot\%202D-112.png}
\end{figure}

Untuk memplot distribusi statistik eksperimental, Anda dapat menggunakan distribution=n dengan plot2d.

\textgreater w=randexponential(1,1000); // exponential distribution

\textgreater plot2d(w,\textgreater distribution): // or distribution=n with n intervals

\begin{figure}
\centering
\pandocbounded{\includegraphics[keepaspectratio]{images/Alfi Nur Azumah-23030630002-MatB23-EMT Plot 2D-113.png}}
\caption{images/Alfi\%20Nur\%20Azumah-23030630002-MatB23-EMT\%20Plot\%202D-113.png}
\end{figure}

Atau Anda dapat menghitung distribusi dari data dan memplot hasilnya dengan \textgreater bar di plot3d, atau dengan plot kolom.

\textgreater w=normal(1000); // 0-1-normal distribution

\textgreater\{x,y\}=histo(w,10,v={[}-6,-4,-2,-1,0,1,2,4,6{]}); // interval bounds v

\textgreater plot2d(x,y,\textgreater bar):

\begin{figure}
\centering
\pandocbounded{\includegraphics[keepaspectratio]{images/Alfi Nur Azumah-23030630002-MatB23-EMT Plot 2D-114.png}}
\caption{images/Alfi\%20Nur\%20Azumah-23030630002-MatB23-EMT\%20Plot\%202D-114.png}
\end{figure}

Fungsi statplot() mengatur gaya dengan string sederhana.

\textgreater statplot(1:10,cumsum(random(10)),``b''):

\begin{figure}
\centering
\pandocbounded{\includegraphics[keepaspectratio]{images/Alfi Nur Azumah-23030630002-MatB23-EMT Plot 2D-115.png}}
\caption{images/Alfi\%20Nur\%20Azumah-23030630002-MatB23-EMT\%20Plot\%202D-115.png}
\end{figure}

\textgreater n=10; i=0:n; \ldots{}\\
\textgreater{} plot2d(i,bin(n,i)/2\^{}n,a=0,b=10,c=0,d=0.3); \ldots{}\\
\textgreater{} plot2d(i,bin(n,i)/2\^{}n,points=true,style=``ow'',add=true,color=blue):

\begin{figure}
\centering
\pandocbounded{\includegraphics[keepaspectratio]{images/Alfi Nur Azumah-23030630002-MatB23-EMT Plot 2D-116.png}}
\caption{images/Alfi\%20Nur\%20Azumah-23030630002-MatB23-EMT\%20Plot\%202D-116.png}
\end{figure}

Selain itu, data dapat diplot sebagai batang. Dalam kasus ini, x harus diurutkan dan satu elemen lebih panjang dari y. Batang akan memanjang dari x{[}i{]} ke x{[}i+1{]} dengan nilai y{[}i{]}. Jika x memiliki ukuran yang sama dengan y, maka akan memanjang satu elemen dengan spasi terakhir.

Gaya isian dapat digunakan seperti di atas.

\textgreater n=10; k=bin(n,0:n); \ldots{}\\
\textgreater{} plot2d(-0.5:n+0.5,k,bar=true,fillcolor=lightgray):

\begin{figure}
\centering
\pandocbounded{\includegraphics[keepaspectratio]{images/Alfi Nur Azumah-23030630002-MatB23-EMT Plot 2D-117.png}}
\caption{images/Alfi\%20Nur\%20Azumah-23030630002-MatB23-EMT\%20Plot\%202D-117.png}
\end{figure}

Data untuk diagram batang (batang=1) dan histogram (histogram=1) dapat diberikan secara eksplisit dalam xv dan yv, atau dapat dihitung dari distribusi empiris dalam xv dengan \textgreater distribusi (atau distribusi=n). Histogram nilai xv akan dihitung secara otomatis dengan \textgreater histogram. Jika \textgreater even ditentukan, nilai xv akan dihitung dalam interval integer.

\textgreater plot2d(normal(10000),distribution=50):

\begin{figure}
\centering
\pandocbounded{\includegraphics[keepaspectratio]{images/Alfi Nur Azumah-23030630002-MatB23-EMT Plot 2D-118.png}}
\caption{images/Alfi\%20Nur\%20Azumah-23030630002-MatB23-EMT\%20Plot\%202D-118.png}
\end{figure}

\textgreater k=0:10; m=bin(10,k); x=(0:11)-0.5; plot2d(x,m,\textgreater bar):

\begin{figure}
\centering
\pandocbounded{\includegraphics[keepaspectratio]{images/Alfi Nur Azumah-23030630002-MatB23-EMT Plot 2D-119.png}}
\caption{images/Alfi\%20Nur\%20Azumah-23030630002-MatB23-EMT\%20Plot\%202D-119.png}
\end{figure}

\textgreater columnsplot(m,k):

\begin{figure}
\centering
\pandocbounded{\includegraphics[keepaspectratio]{images/Alfi Nur Azumah-23030630002-MatB23-EMT Plot 2D-120.png}}
\caption{images/Alfi\%20Nur\%20Azumah-23030630002-MatB23-EMT\%20Plot\%202D-120.png}
\end{figure}

\textgreater plot2d(random(600)*6,histogram=6):

\begin{figure}
\centering
\pandocbounded{\includegraphics[keepaspectratio]{images/Alfi Nur Azumah-23030630002-MatB23-EMT Plot 2D-121.png}}
\caption{images/Alfi\%20Nur\%20Azumah-23030630002-MatB23-EMT\%20Plot\%202D-121.png}
\end{figure}

Untuk distribusi, ada parameter distribution=n, yang menghitung nilai secara otomatis dan mencetak distribusi relatif dengan n sub-interval.

\textgreater plot2d(normal(1,1000),distribution=10,style=``\textbackslash/''):

\begin{figure}
\centering
\pandocbounded{\includegraphics[keepaspectratio]{images/Alfi Nur Azumah-23030630002-MatB23-EMT Plot 2D-122.png}}
\caption{images/Alfi\%20Nur\%20Azumah-23030630002-MatB23-EMT\%20Plot\%202D-122.png}
\end{figure}

Dengan parameter even=true, ini akan menggunakan interval integer.

\textgreater plot2d(intrandom(1,1000,10),distribution=10,even=true):

\begin{figure}
\centering
\pandocbounded{\includegraphics[keepaspectratio]{images/Alfi Nur Azumah-23030630002-MatB23-EMT Plot 2D-123.png}}
\caption{images/Alfi\%20Nur\%20Azumah-23030630002-MatB23-EMT\%20Plot\%202D-123.png}
\end{figure}

Perhatikan bahwa ada banyak plot statistik yang mungkin berguna. Lihatlah tutorial tentang statistik.

\textgreater columnsplot(getmultiplicities(1:6,intrandom(1,6000,6))):

\begin{figure}
\centering
\pandocbounded{\includegraphics[keepaspectratio]{images/Alfi Nur Azumah-23030630002-MatB23-EMT Plot 2D-124.png}}
\caption{images/Alfi\%20Nur\%20Azumah-23030630002-MatB23-EMT\%20Plot\%202D-124.png}
\end{figure}

\textgreater plot2d(normal(1,1000),\textgreater distribution); \ldots{}\\
\textgreater{} plot2d(``qnormal(x)'',color=red,thickness=2,\textgreater add):

\begin{figure}
\centering
\pandocbounded{\includegraphics[keepaspectratio]{images/Alfi Nur Azumah-23030630002-MatB23-EMT Plot 2D-125.png}}
\caption{images/Alfi\%20Nur\%20Azumah-23030630002-MatB23-EMT\%20Plot\%202D-125.png}
\end{figure}

Ada juga banyak plot khusus untuk statistik. Boxplot menunjukkan kuartil dari distribusi ini dan banyak outlier. Menurut definisi, outlier dalam boxplot adalah data yang melebihi 1,5 kali rentang tengah 50\% dari plot.

\textgreater M=normal(5,1000); boxplot(quartiles(M)):

\begin{figure}
\centering
\pandocbounded{\includegraphics[keepaspectratio]{images/Alfi Nur Azumah-23030630002-MatB23-EMT Plot 2D-126.png}}
\caption{images/Alfi\%20Nur\%20Azumah-23030630002-MatB23-EMT\%20Plot\%202D-126.png}
\end{figure}

\subsection{3.16 Fungsi Implisit}\label{fungsi-implisit}

Plot implisit menunjukkan garis level yang menyelesaikan f(x,y)=level, di mana ``level'' dapat berupa nilai tunggal atau vektor nilai. Jika level=``auto'', akan ada garis level nc, yang akan menyebar antara minimum dan maksimum fungsi secara merata. Warna yang lebih gelap atau lebih terang dapat ditambahkan dengan \textgreater hue untuk menunjukkan nilai fungsi. Untuk fungsi implisit, xv harus berupa fungsi atau ekspresi parameter x dan y, atau, sebagai alternatif, xv dapat berupa matriks nilai.

Euler dapat menandai garis level dari fungsi apa pun.

Untuk menggambar himpunan f(x,y)=c untuk satu atau lebih konstanta c, Anda dapat menggunakan plot2d() dengan plot implisitnya di bidang. Parameter untuk c adalah level=c, di mana c dapat berupa vektor garis level. Selain itu, skema warna dapat digambar di latar belakang untuk menunjukkan nilai fungsi untuk setiap titik dalam plot. Parameter ``n'' menentukan kehalusan plot.

\textgreater aspect(1.5);

\textgreater plot2d(``x\textsuperscript{2+y}2-x*y-x'',r=1.5,level=0,contourcolor=red):

\begin{figure}
\centering
\pandocbounded{\includegraphics[keepaspectratio]{images/Alfi Nur Azumah-23030630002-MatB23-EMT Plot 2D-127.png}}
\caption{images/Alfi\%20Nur\%20Azumah-23030630002-MatB23-EMT\%20Plot\%202D-127.png}
\end{figure}

\textgreater expr := ``2*x\textsuperscript{2+x*y+3*y}4+y''; // define an expression f(x,y)

\textgreater plot2d(expr,level=0): // Solutions of f(x,y)=0

\begin{figure}
\centering
\pandocbounded{\includegraphics[keepaspectratio]{images/Alfi Nur Azumah-23030630002-MatB23-EMT Plot 2D-128.png}}
\caption{images/Alfi\%20Nur\%20Azumah-23030630002-MatB23-EMT\%20Plot\%202D-128.png}
\end{figure}

\textgreater plot2d(expr,level=0:0.5:20,\textgreater hue,contourcolor=white,n=200): // nice

\begin{figure}
\centering
\pandocbounded{\includegraphics[keepaspectratio]{images/Alfi Nur Azumah-23030630002-MatB23-EMT Plot 2D-129.png}}
\caption{images/Alfi\%20Nur\%20Azumah-23030630002-MatB23-EMT\%20Plot\%202D-129.png}
\end{figure}

\textgreater plot2d(expr,level=0:0.5:20,\textgreater hue,\textgreater spectral,n=200,grid=4): // nicer

\begin{figure}
\centering
\pandocbounded{\includegraphics[keepaspectratio]{images/Alfi Nur Azumah-23030630002-MatB23-EMT Plot 2D-130.png}}
\caption{images/Alfi\%20Nur\%20Azumah-23030630002-MatB23-EMT\%20Plot\%202D-130.png}
\end{figure}

Ini juga berlaku untuk plot data. Namun, Anda harus menentukan rentang untuk label sumbu.

\textgreater x=-2:0.05:1; y=x'; z=expr(x,y);

\textgreater plot2d(z,level=0,a=-1,b=2,c=-2,d=1,\textgreater hue):

\begin{figure}
\centering
\pandocbounded{\includegraphics[keepaspectratio]{images/Alfi Nur Azumah-23030630002-MatB23-EMT Plot 2D-131.png}}
\caption{images/Alfi\%20Nur\%20Azumah-23030630002-MatB23-EMT\%20Plot\%202D-131.png}
\end{figure}

\textgreater plot2d(``x\textsuperscript{3-y}2'',\textgreater contour,\textgreater hue,\textgreater spectral):

\begin{figure}
\centering
\pandocbounded{\includegraphics[keepaspectratio]{images/Alfi Nur Azumah-23030630002-MatB23-EMT Plot 2D-132.png}}
\caption{images/Alfi\%20Nur\%20Azumah-23030630002-MatB23-EMT\%20Plot\%202D-132.png}
\end{figure}

\textgreater plot2d(``x\textsuperscript{3-y}2'',level=0,contourwidth=3,\textgreater add,contourcolor=red):

\begin{figure}
\centering
\pandocbounded{\includegraphics[keepaspectratio]{images/Alfi Nur Azumah-23030630002-MatB23-EMT Plot 2D-133.png}}
\caption{images/Alfi\%20Nur\%20Azumah-23030630002-MatB23-EMT\%20Plot\%202D-133.png}
\end{figure}

\textgreater z=z+normal(size(z))*0.2;

\textgreater plot2d(z,level=0.5,a=-1,b=2,c=-2,d=1):

\begin{figure}
\centering
\pandocbounded{\includegraphics[keepaspectratio]{images/Alfi Nur Azumah-23030630002-MatB23-EMT Plot 2D-134.png}}
\caption{images/Alfi\%20Nur\%20Azumah-23030630002-MatB23-EMT\%20Plot\%202D-134.png}
\end{figure}

\textgreater plot2d(expr,level={[}0:0.2:5;0.05:0.2:5.05{]},color=lightgray):

\begin{figure}
\centering
\pandocbounded{\includegraphics[keepaspectratio]{images/Alfi Nur Azumah-23030630002-MatB23-EMT Plot 2D-135.png}}
\caption{images/Alfi\%20Nur\%20Azumah-23030630002-MatB23-EMT\%20Plot\%202D-135.png}
\end{figure}

\textgreater plot2d(``x\textsuperscript{2+y}3+x*y'',level=1,r=4,n=100):

\begin{figure}
\centering
\pandocbounded{\includegraphics[keepaspectratio]{images/Alfi Nur Azumah-23030630002-MatB23-EMT Plot 2D-136.png}}
\caption{images/Alfi\%20Nur\%20Azumah-23030630002-MatB23-EMT\%20Plot\%202D-136.png}
\end{figure}

\textgreater plot2d(``x\textsuperscript{2+2*y}2-x*y'',level=0:0.1:10,n=10,contourcolor=white,\textgreater hue):

\begin{figure}
\centering
\pandocbounded{\includegraphics[keepaspectratio]{images/Alfi Nur Azumah-23030630002-MatB23-EMT Plot 2D-137.png}}
\caption{images/Alfi\%20Nur\%20Azumah-23030630002-MatB23-EMT\%20Plot\%202D-137.png}
\end{figure}

Dimungkinkan juga untuk mengisi himpunan

dengan rentang level.

Dimungkinkan untuk mengisi wilayah nilai untuk fungsi tertentu. Untuk ini, level harus berupa matriks 2xn. Baris pertama adalah batas bawah dan baris kedua berisi batas atas.

\textgreater plot2d(expr,level={[}0;1{]},style=``-'',color=blue): // 0 \textless= f(x,y) \textless= 1

\begin{figure}
\centering
\pandocbounded{\includegraphics[keepaspectratio]{images/Alfi Nur Azumah-23030630002-MatB23-EMT Plot 2D-138.png}}
\caption{images/Alfi\%20Nur\%20Azumah-23030630002-MatB23-EMT\%20Plot\%202D-138.png}
\end{figure}

Plot implisit juga dapat menunjukkan rentang level. Level harus berupa matriks 2xn interval level, di mana baris pertama berisi awal dan baris kedua berisi akhir setiap interval. Atau, vektor baris sederhana dapat digunakan untuk level, dan parameter dl memperluas nilai level ke interval.

\textgreater plot2d(``x\textsuperscript{4+y}4'',r=1.5,level={[}0;1{]},color=blue,style=``/''):

\begin{figure}
\centering
\pandocbounded{\includegraphics[keepaspectratio]{images/Alfi Nur Azumah-23030630002-MatB23-EMT Plot 2D-139.png}}
\caption{images/Alfi\%20Nur\%20Azumah-23030630002-MatB23-EMT\%20Plot\%202D-139.png}
\end{figure}

\textgreater plot2d(``x\textsuperscript{2+y}3+x*y'',level={[}0,2,4;1,3,5{]},style=``/'',r=2,n=100):

\begin{figure}
\centering
\pandocbounded{\includegraphics[keepaspectratio]{images/Alfi Nur Azumah-23030630002-MatB23-EMT Plot 2D-140.png}}
\caption{images/Alfi\%20Nur\%20Azumah-23030630002-MatB23-EMT\%20Plot\%202D-140.png}
\end{figure}

\textgreater plot2d(``x\textsuperscript{2+y}3+x*y'',level=-10:20,r=2,style=``-'',dl=0.1,n=100):

\begin{figure}
\centering
\pandocbounded{\includegraphics[keepaspectratio]{images/Alfi Nur Azumah-23030630002-MatB23-EMT Plot 2D-141.png}}
\caption{images/Alfi\%20Nur\%20Azumah-23030630002-MatB23-EMT\%20Plot\%202D-141.png}
\end{figure}

\textgreater plot2d(``sin(x)*cos(y)'',r=pi,\textgreater hue,\textgreater levels,n=100):

\begin{figure}
\centering
\pandocbounded{\includegraphics[keepaspectratio]{images/Alfi Nur Azumah-23030630002-MatB23-EMT Plot 2D-142.png}}
\caption{images/Alfi\%20Nur\%20Azumah-23030630002-MatB23-EMT\%20Plot\%202D-142.png}
\end{figure}

Dimungkinkan juga untuk menandai suatu wilayah

Hal ini dilakukan dengan menambahkan level dengan dua baris.

\textgreater plot2d(``(x\textsuperscript{2+y}2-1)\textsuperscript{3-x}2*y\^{}3'',r=1.9, \ldots{}\\
\textgreater{} style=``\#'',color=red,\textless outline, \ldots{}\\
\textgreater{} level={[}-2;0{]},n=100):

\begin{figure}
\centering
\pandocbounded{\includegraphics[keepaspectratio]{images/Alfi Nur Azumah-23030630002-MatB23-EMT Plot 2D-143.png}}
\caption{images/Alfi\%20Nur\%20Azumah-23030630002-MatB23-EMT\%20Plot\%202D-143.png}
\end{figure}

Dimungkinkan untuk menentukan level tertentu. Misalnya, kita dapat memplot solusi persamaan seperti

\textgreater plot2d(``x\textsuperscript{3-x*y+x}2*y\^{}2'',r=6,level=1,n=100):

\begin{figure}
\centering
\pandocbounded{\includegraphics[keepaspectratio]{images/Alfi Nur Azumah-23030630002-MatB23-EMT Plot 2D-144.png}}
\caption{images/Alfi\%20Nur\%20Azumah-23030630002-MatB23-EMT\%20Plot\%202D-144.png}
\end{figure}

\textgreater function starplot1 (v, style=``/'', color=green, lab=none) \ldots{}

\begin{verbatim}
  if !holding() then clg; endif;
  w=window(); window(0,0,1024,1024);
  h=holding(1);
  r=max(abs(v))*1.2;
  setplot(-r,r,-r,r);
  n=cols(v); t=linspace(0,2pi,n);
  v=v|v[1]; c=v*cos(t); s=v*sin(t);
  cl=barcolor(color); st=barstyle(style);
  loop 1 to n
    polygon([0,c[#],c[#+1]],[0,s[#],s[#+1]],1);
    if lab!=none then
      rlab=v[#]+r*0.1;
      {col,row}=toscreen(cos(t[#])*rlab,sin(t[#])*rlab);
      ctext(""+lab[#],col,row-textheight()/2);
    endif;
  end;
  barcolor(cl); barstyle(st);
  holding(h);
  window(w);
endfunction
\end{verbatim}

Tidak ada tanda centang pada grid atau sumbu di sini. Selain itu, kami menggunakan jendela penuh untuk plot.

Kami memanggil reset sebelum menguji plot ini untuk mengembalikan grafik ke default. Ini tidak perlu, jika Anda yakin bahwa plot Anda berfungsi.

\textgreater reset; starplot1(normal(1,10)+5,color=red,lab=1:10):

\begin{figure}
\centering
\pandocbounded{\includegraphics[keepaspectratio]{images/Alfi Nur Azumah-23030630002-MatB23-EMT Plot 2D-145.png}}
\caption{images/Alfi\%20Nur\%20Azumah-23030630002-MatB23-EMT\%20Plot\%202D-145.png}
\end{figure}

Terkadang, Anda mungkin ingin memplot sesuatu yang tidak dapat dilakukan oleh plot2d, tetapi hampir.

Dalam fungsi berikut, kita melakukan plot impuls logaritmik. plot2d dapat melakukan plot logaritmik, tetapi tidak untuk batang impuls.

\textgreater function logimpulseplot1 (x,y) \ldots{}

\begin{verbatim}
  {x0,y0}=makeimpulse(x,log(y)/log(10));
  plot2d(x0,y0,>bar,grid=0);
  h=holding(1);
  frame();
  xgrid(ticks(x));
  p=plot();
  for i=-10 to 10;
    if i<=p[4] and i>=p[3] then
       ygrid(i,yt="10^"+i);
    endif;
  end;
  holding(h);
endfunction
\end{verbatim}

Mari kita mengujinya dengan nilai-nilai yang terdistribusi secara eksponensial.

\textgreater aspect(1.5); x=1:10; y=-log(random(size(x)))*200; \ldots{}\\
\textgreater{} logimpulseplot1(x,y):

\begin{figure}
\centering
\pandocbounded{\includegraphics[keepaspectratio]{images/Alfi Nur Azumah-23030630002-MatB23-EMT Plot 2D-146.png}}
\caption{images/Alfi\%20Nur\%20Azumah-23030630002-MatB23-EMT\%20Plot\%202D-146.png}
\end{figure}

Mari kita animasikan kurva 2D menggunakan plot langsung. Perintah plot(x,y) cukup memplot kurva ke dalam jendela plot. setplot(a,b,c,d) mengatur jendela ini.

Fungsi wait(0) memaksa plot untuk muncul di jendela grafik. Jika tidak, penggambaran ulang akan dilakukan dalam interval waktu yang jarang.

\textgreater function animliss (n,m) \ldots{}

\begin{verbatim}
t=linspace(0,2pi,500);
f=0;
c=framecolor(0);
l=linewidth(2);
setplot(-1,1,-1,1);
repeat
  clg;
  plot(sin(n*t),cos(m*t+f));
  wait(0);
  if testkey() then break; endif;
  f=f+0.02;
end;
framecolor(c);
linewidth(l);
endfunction
\end{verbatim}

Tekan tombol apa saja untuk menghentikan animasi ini.

\textgreater animliss(2,3); // lihat hasilnya, jika sudah puas, tekan ENTER

\subsection{3.17 Plot Logaritmik}\label{plot-logaritmik}

EMT menggunakan parameter ``logplot'' untuk skala logaritmik.

Plot logaritmik dapat diplot menggunakan skala logaritmik dalam y dengan logplot=1, atau menggunakan skala logaritmik dalam x dan y dengan logplot=2, atau dalam x dengan logplot=3.

\begin{itemize}
\tightlist
\item
  logplot=1: y-logaritmik
\item
  logplot=2: x-y-logaritmik
\item
  logplot=3: x-logaritmik
\end{itemize}

\textgreater plot2d(``exp(x\textsuperscript{3-x)*x}2'',1,5,logplot=1):

\begin{figure}
\centering
\pandocbounded{\includegraphics[keepaspectratio]{images/Alfi Nur Azumah-23030630002-MatB23-EMT Plot 2D-147.png}}
\caption{images/Alfi\%20Nur\%20Azumah-23030630002-MatB23-EMT\%20Plot\%202D-147.png}
\end{figure}

\textgreater plot2d(``exp(x+sin(x))'',0,100,logplot=1):

\begin{figure}
\centering
\pandocbounded{\includegraphics[keepaspectratio]{images/Alfi Nur Azumah-23030630002-MatB23-EMT Plot 2D-148.png}}
\caption{images/Alfi\%20Nur\%20Azumah-23030630002-MatB23-EMT\%20Plot\%202D-148.png}
\end{figure}

\textgreater plot2d(``exp(x+sin(x))'',10,100,logplot=2):

\begin{figure}
\centering
\pandocbounded{\includegraphics[keepaspectratio]{images/Alfi Nur Azumah-23030630002-MatB23-EMT Plot 2D-149.png}}
\caption{images/Alfi\%20Nur\%20Azumah-23030630002-MatB23-EMT\%20Plot\%202D-149.png}
\end{figure}

\textgreater plot2d(``gamma(x)'',1,10,logplot=1):

\begin{figure}
\centering
\pandocbounded{\includegraphics[keepaspectratio]{images/Alfi Nur Azumah-23030630002-MatB23-EMT Plot 2D-150.png}}
\caption{images/Alfi\%20Nur\%20Azumah-23030630002-MatB23-EMT\%20Plot\%202D-150.png}
\end{figure}

\textgreater plot2d(``log(x*(2+sin(x/100)))'',10,1000,logplot=3):

\begin{figure}
\centering
\pandocbounded{\includegraphics[keepaspectratio]{images/Alfi Nur Azumah-23030630002-MatB23-EMT Plot 2D-151.png}}
\caption{images/Alfi\%20Nur\%20Azumah-23030630002-MatB23-EMT\%20Plot\%202D-151.png}
\end{figure}

Hal ini juga berlaku untuk plot data.

\textgreater x=10\^{}(1:20); y=x\^{}2-x;

\textgreater plot2d(x,y,logplot=2):

\begin{figure}
\centering
\pandocbounded{\includegraphics[keepaspectratio]{images/Alfi Nur Azumah-23030630002-MatB23-EMT Plot 2D-152.png}}
\caption{images/Alfi\%20Nur\%20Azumah-23030630002-MatB23-EMT\%20Plot\%202D-152.png}
\end{figure}

\textbf{Contoh soal}

\begin{enumerate}
\def\labelenumi{\arabic{enumi}.}
\tightlist
\item
  Buatlah plot 2D dari
\end{enumerate}

\(y=8x^3e^{-x^3}\)

dan turunannya terhadap x, dengan grafik turunannya berwarna merah putus putus. Plot dari x=-2 hingga x=3 dan y=-6 hingga y=6 dengan grid=2. Berikan keterangan pada masing masing grafik

\textgreater function f(x) \&= 8*x\textsuperscript{3*exp(-x}3); \ldots{}\\
\textgreater{} plot2d(\&f(x),a=-2,b=3,c=-6,d=6, grid=2); \ldots{}\\
\textgreater{} plot2d(\&diff(f(x),x),\textgreater add,color=blue,style=``--''); \ldots{}\\
\textgreater{} labelbox({[}``fungsi'',``turunan''{]},styles={[}``-'',``--''{]}, \ldots{}\\
\textgreater{} colors={[}black,blue{]},w=0.4):

\begin{figure}
\centering
\pandocbounded{\includegraphics[keepaspectratio]{images/Alfi Nur Azumah-23030630002-MatB23-EMT Plot 2D-153.png}}
\caption{images/Alfi\%20Nur\%20Azumah-23030630002-MatB23-EMT\%20Plot\%202D-153.png}
\end{figure}

\begin{enumerate}
\def\labelenumi{\arabic{enumi}.}
\setcounter{enumi}{1}
\tightlist
\item
  Buatlah titik titik (1,4.5),(2.5,3),(3.5,6),(4,4) dan hubungkan keempat titik tersebut dengan garis putus putus
\end{enumerate}

\textgreater xdata={[}1,2.5,3.5,4{]}; ydata={[}4.5,3.5,6,4{]}; // data

\textgreater plot2d(xdata,ydata,a=0.5,b=4.5,c=3.0,d=6.5, style=``.''); // lines

\textgreater plot2d(xdata,ydata,\textgreater points,\textgreater add,style=``o''): // add points

\begin{figure}
\centering
\pandocbounded{\includegraphics[keepaspectratio]{images/Alfi Nur Azumah-23030630002-MatB23-EMT Plot 2D-154.png}}
\caption{images/Alfi\%20Nur\%20Azumah-23030630002-MatB23-EMT\%20Plot\%202D-154.png}
\end{figure}

\begin{enumerate}
\def\labelenumi{\arabic{enumi}.}
\setcounter{enumi}{2}
\tightlist
\item
  Buatlah plot 2D dari fungsi
\end{enumerate}

\(y= \frac{x^3}{a}\)

dengan a=3 dan interval dari x=-2 sampai x=1. Buatlah plot tersebut dalam grid 6.

\textgreater a:=3; plot2d(``x\^{}3/a'',-2,1,grid=6):

\begin{figure}
\centering
\pandocbounded{\includegraphics[keepaspectratio]{images/Alfi Nur Azumah-23030630002-MatB23-EMT Plot 2D-155.png}}
\caption{images/Alfi\%20Nur\%20Azumah-23030630002-MatB23-EMT\%20Plot\%202D-155.png}
\end{figure}

\begin{enumerate}
\def\labelenumi{\arabic{enumi}.}
\setcounter{enumi}{3}
\tightlist
\item
  Buatlah peta contour dari
\end{enumerate}

\(x^3+y^2+xy\)

dengan gradasi gelap terang warna spectral dan n=200

\textgreater plot2d(``x\textsuperscript{3+y}2+x*y'',\textgreater contour,\textgreater hue,\textgreater spectral,n=200):

\begin{figure}
\centering
\pandocbounded{\includegraphics[keepaspectratio]{images/Alfi Nur Azumah-23030630002-MatB23-EMT Plot 2D-156.png}}
\caption{images/Alfi\%20Nur\%20Azumah-23030630002-MatB23-EMT\%20Plot\%202D-156.png}
\end{figure}

\begin{enumerate}
\def\labelenumi{\arabic{enumi}.}
\setcounter{enumi}{4}
\tightlist
\item
  Buatlah plot 2D dari fungsi
\end{enumerate}

\(y(x,a)=x+ax^3\)

dengan nilai parameter a=4 dan interval x dari -2 sampai 2

\textgreater function f(x,a):=x+a*x\^{}3;

\textgreater a=4; plot2d(``f'',-2,2;a):

\begin{figure}
\centering
\pandocbounded{\includegraphics[keepaspectratio]{images/Alfi Nur Azumah-23030630002-MatB23-EMT Plot 2D-157.png}}
\caption{images/Alfi\%20Nur\%20Azumah-23030630002-MatB23-EMT\%20Plot\%202D-157.png}
\end{figure}

\begin{enumerate}
\def\labelenumi{\arabic{enumi}.}
\setcounter{enumi}{5}
\tightlist
\item
  Buatlah plot 2D dari fungsi
\end{enumerate}

\(y=sin(x)+cos(x)\)

dengan x dari -pi sampai 2pi dengan grid 10, ketebalan 2 dan beri nama plot

\textgreater plot2d(``sin(x)+cos(x)'',-pi,2*pi,grid=10,thickness=2, title=``y=sin(x)+cos(x)''):

\begin{figure}
\centering
\pandocbounded{\includegraphics[keepaspectratio]{images/Alfi Nur Azumah-23030630002-MatB23-EMT Plot 2D-158.png}}
\caption{images/Alfi\%20Nur\%20Azumah-23030630002-MatB23-EMT\%20Plot\%202D-158.png}
\end{figure}

\begin{enumerate}
\def\labelenumi{\arabic{enumi}.}
\setcounter{enumi}{6}
\tightlist
\item
  Diketahui rata-rata waktu istiharat pekerja di suatu perusahaan yaitu hari Senin 5 jam, Selasa 7 jam, Rabu 5 jam, Kamis 4 jam, Jumat 3 jam, Sabtu 13 jam, dan Minggu 15 jam. Buatlah data tersebut dalam diagram batang
\end{enumerate}

\textgreater hari={[}``Senin'',``Selasa'',``Rabu'',``Kamis'',``Jumat'',``Sabtu'',``Minggu''{]};

\textgreater values={[}5,7,5,4,3,13,15{]};

\textgreater columnsplot(values,lab=hari,color=blue,style=``-'');

\textgreater title(``Waktu istirahat''):

\begin{figure}
\centering
\pandocbounded{\includegraphics[keepaspectratio]{images/Alfi Nur Azumah-23030630002-MatB23-EMT Plot 2D-159.png}}
\caption{images/Alfi\%20Nur\%20Azumah-23030630002-MatB23-EMT\%20Plot\%202D-159.png}
\end{figure}

\begin{enumerate}
\def\labelenumi{\arabic{enumi}.}
\setcounter{enumi}{7}
\tightlist
\item
  Buatlah plot 2D dari turunan fungsi
\end{enumerate}

\(y=2x^3+x^2-4x\)

terhadap x dengan grid 2, ketebalan 3 dan warna hijau.

\textgreater function f(x) \&= 2*x\textsuperscript{3+x}2-4*x;

\textgreater plot2d(\&diff(f(x),x), grid=2, color=green, thickness=3):

\begin{figure}
\centering
\pandocbounded{\includegraphics[keepaspectratio]{images/Alfi Nur Azumah-23030630002-MatB23-EMT Plot 2D-160.png}}
\caption{images/Alfi\%20Nur\%20Azumah-23030630002-MatB23-EMT\%20Plot\%202D-160.png}
\end{figure}

\begin{enumerate}
\def\labelenumi{\arabic{enumi}.}
\setcounter{enumi}{8}
\tightlist
\item
  Buatlah plot 2D dari fungsi
\end{enumerate}

\(\frac{a^4-x^2}{a+x}\)

dengan a=2, dari x=-1 sampai x=3

\textgreater a=2; plot2d(``(a\textsuperscript{4-x}2)/a+x'',-1,3):

\begin{figure}
\centering
\pandocbounded{\includegraphics[keepaspectratio]{images/Alfi Nur Azumah-23030630002-MatB23-EMT Plot 2D-161.png}}
\caption{images/Alfi\%20Nur\%20Azumah-23030630002-MatB23-EMT\%20Plot\%202D-161.png}
\end{figure}

\begin{enumerate}
\def\labelenumi{\arabic{enumi}.}
\setcounter{enumi}{9}
\tightlist
\item
  Buatlah plot 2D dari sin(x) dan cos(x) dari -4pi sampai 0
\end{enumerate}

\textgreater plot2d(``sin(x)'',-4*pi,0, grid=2); \ldots{}\\
\textgreater{} plot2d(``cos(x)'',\textgreater add,color=green,style=``\,``); \ldots{}\\
\textgreater{} labelbox({[}``sin(x)'',``cos(x)''{]},styles={[}``-'',``-''{]}, \ldots{}\\
\textgreater{} colors={[}black,green{]},w=0.4):

\begin{figure}
\centering
\pandocbounded{\includegraphics[keepaspectratio]{images/Alfi Nur Azumah-23030630002-MatB23-EMT Plot 2D-162.png}}
\caption{images/Alfi\%20Nur\%20Azumah-23030630002-MatB23-EMT\%20Plot\%202D-162.png}
\end{figure}

\subsection{3.18 Rujukan Lengkap Fungsi plot2d()}\label{rujukan-lengkap-fungsi-plot2d}

function plot2d (xv, yv, btest, a, b, c, d, xmin, xmax, r, n, ..\\
logplot, grid, frame, framecolor, square, color, thickness, style, ..\\
auto, add, user, delta, points, addpoints, pointstyle, bar, histogram, ..\\
distribution, even, steps, own, adaptive, hue, level, contour, ..\\
nc, filled, fillcolor, outline, title, xl, yl, maps, contourcolor, ..\\
contourwidth, ticks, margin, clipping, cx, cy, insimg, spectral, ..\\
cgrid, vertical, smaller, dl, niveau, levels)

Multipurpose plot function for plots in the plane (2D plots). This function can do plots of functions of one variables, data plots, curves in the plane, bar plots, grids of complex numbers, and implicit plots of functions of two variables.

Parameters

x,y : equations, functions or data vectors

a,b,c,d : Plot area (default a=-2,b=2)

r : if r is set, then a=cx-r, b=cx+r, c=cy-r, d=cy+r

\begin{verbatim}
        r can be a vector [rx,ry] or a vector [rx1,rx2,ry1,ry2].
\end{verbatim}

xmin,xmax : range of the parameter for curves

auto : Determine y-range automatically (default)

square : if true, try to keep square x-y-ranges

n : number of intervals (default is adaptive)

grid : 0 = no grid and labels,

\begin{verbatim}
        1 = axis only,


        2 = normal grid (see below for the number of grid lines)


        3 = inside axis


        4 = no grid


        5 = full grid including margin


        6 = ticks at the frame


        7 = axis only


        8 = axis only, sub-ticks
\end{verbatim}

frame : 0 = no frame

framecolor: color of the frame and the grid

margin : number between 0 and 0.4 for the margin around the plot

color : Color of curves. If this is a vector of colors,

\begin{verbatim}
        it will be used for each row of a matrix of plots. In the case of


        point plots, it should be a column vector. If a row vector or a


        full matrix of colors is used for point plots, it will be used for


        each data point.
\end{verbatim}

thickness : line thickness for curves

\begin{verbatim}
        This value can be smaller than 1 for very thin lines.
\end{verbatim}

style : Plot style for lines, markers, and fills.

\begin{verbatim}
        For points use


        "[]", "&lt;&gt;", ".", "..", "...",


        "*", "+", "|", "-", "o"


        "[]#", "&lt;&gt;#", "o#" (filled shapes)


        "[]w", "&lt;&gt;w", "ow" (non-transparent)


        For lines use


        "-", "--", "-.", ".", ".-.", "-.-", "-&gt;"


        For filled polygons or bar plots use


        "#", "#O", "O", "/", "\", "\/",


        "+", "|", "-", "t"
\end{verbatim}

points : plot single points instead of line segments

addpoints : if true, plots line segments and points

add : add the plot to the existing plot

user : enable user interaction for functions

delta : step size for user interaction

bar : bar plot (x are the interval bounds, y the interval values)

histogram : plots the frequencies of x in n subintervals

distribution=n : plots the distribution of x with n subintervals

even : use inter values for automatic histograms.

steps : plots the function as a step function (steps=1,2)

adaptive : use adaptive plots (n is the minimal number of steps)

level : plot level lines of an implicit function of two variables

outline : draws boundary of level ranges.

If the level value is a 2xn matrix, ranges of levels will be drawn

in the color using the given fill style. If outline is true, it

will be drawn in the contour color. Using this feature, regions of

f(x,y) between limits can be marked.

hue : add hue color to the level plot to indicate the function

\begin{verbatim}
        value
\end{verbatim}

contour : Use level plot with automatic levels

nc : number of automatic level lines

title : plot title (default ``\,``)

xl, yl : labels for the x- and y-axis

smaller : if \textgreater0, there will be more space to the left for labels.

vertical :

Turns vertical labels on or off. This changes the global variable

verticallabels locally for one plot. The value 1 sets only vertical

text, the value 2 uses vertical numerical labels on the y axis.

filled : fill the plot of a curve

fillcolor : fill color for bar and filled curves

outline : boundary for filled polygons

logplot : set logarithmic plots

\begin{verbatim}
        1 = logplot in y,


        2 = logplot in xy,


        3 = logplot in x
\end{verbatim}

own :

A string, which points to an own plot routine. With \textgreater user, you get

the same user interaction as in plot2d. The range will be set

before each call to your function.

maps : map expressions (0 is faster), functions are always mapped.

contourcolor : color of contour lines

contourwidth : width of contour lines

clipping : toggles the clipping (default is true)

title :

This can be used to describe the plot. The title will appear above

the plot. Moreover, a label for the x and y axis can be added with

xl=``string'' or yl=``string''. Other labels can be added with the

functions label() or labelbox(). The title can be a unicode

string or an image of a Latex formula.

cgrid :

Determines the number of grid lines for plots of complex grids.

Should be a divisor of the the matrix size minus 1 (number of

subintervals). cgrid can be a vector {[}cx,cy{]}.

Overview

The function can plot

\begin{itemize}
\tightlist
\item
  expressions, call collections or functions of one variable,
\item
  parametric curves,
\item
  x data against y data,
\item
  implicit functions,
\item
  bar plots,
\item
  complex grids,
\item
  polygons.
\end{itemize}

If a function or expression for xv is given, plot2d() will compute values in the given range using the function or expression. The expression must be an expression in the variable x. The range must be defined in the parameters a and b unless the default range {[}-2,2{]} should be used. The y-range will be computed automatically, unless c and d are specified, or a radius r, which yields the range {[}-r,r{]} for x and y. For plots of functions, plot2d will use an adaptive evaluation of the function by default. To speed up the plot for complicated functions, switch this off with \textless adaptive, and optionally decrease the number of intervals n.~Moreover, plot2d() will by default use mapping. I.e., it will compute the plot element for element. If your expression or your functions can handle a vector x, you can switch that off with \textless maps for faster evaluation.

Note that adaptive plots are always computed element for element. If functions or expressions for both xv and for yv are specified, plot2d() will compute a curve with the xv values as x-coordinates and the yv values as y-coordinates. In this case, a range should be defined for the parameter using xmin, xmax. Expressions contained

\backmatter
\end{document}
