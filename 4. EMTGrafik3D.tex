% Options for packages loaded elsewhere
\PassOptionsToPackage{unicode}{hyperref}
\PassOptionsToPackage{hyphens}{url}
\documentclass[
]{book}
\usepackage{xcolor}
\usepackage{amsmath,amssymb}
\setcounter{secnumdepth}{-\maxdimen} % remove section numbering
\usepackage{iftex}
\ifPDFTeX
  \usepackage[T1]{fontenc}
  \usepackage[utf8]{inputenc}
  \usepackage{textcomp} % provide euro and other symbols
\else % if luatex or xetex
  \usepackage{unicode-math} % this also loads fontspec
  \defaultfontfeatures{Scale=MatchLowercase}
  \defaultfontfeatures[\rmfamily]{Ligatures=TeX,Scale=1}
\fi
\usepackage{lmodern}
\ifPDFTeX\else
  % xetex/luatex font selection
\fi
% Use upquote if available, for straight quotes in verbatim environments
\IfFileExists{upquote.sty}{\usepackage{upquote}}{}
\IfFileExists{microtype.sty}{% use microtype if available
  \usepackage[]{microtype}
  \UseMicrotypeSet[protrusion]{basicmath} % disable protrusion for tt fonts
}{}
\makeatletter
\@ifundefined{KOMAClassName}{% if non-KOMA class
  \IfFileExists{parskip.sty}{%
    \usepackage{parskip}
  }{% else
    \setlength{\parindent}{0pt}
    \setlength{\parskip}{6pt plus 2pt minus 1pt}}
}{% if KOMA class
  \KOMAoptions{parskip=half}}
\makeatother
\usepackage{graphicx}
\makeatletter
\newsavebox\pandoc@box
\newcommand*\pandocbounded[1]{% scales image to fit in text height/width
  \sbox\pandoc@box{#1}%
  \Gscale@div\@tempa{\textheight}{\dimexpr\ht\pandoc@box+\dp\pandoc@box\relax}%
  \Gscale@div\@tempb{\linewidth}{\wd\pandoc@box}%
  \ifdim\@tempb\p@<\@tempa\p@\let\@tempa\@tempb\fi% select the smaller of both
  \ifdim\@tempa\p@<\p@\scalebox{\@tempa}{\usebox\pandoc@box}%
  \else\usebox{\pandoc@box}%
  \fi%
}
% Set default figure placement to htbp
\def\fps@figure{htbp}
\makeatother
\setlength{\emergencystretch}{3em} % prevent overfull lines
\providecommand{\tightlist}{%
  \setlength{\itemsep}{0pt}\setlength{\parskip}{0pt}}
\usepackage{bookmark}
\IfFileExists{xurl.sty}{\usepackage{xurl}}{} % add URL line breaks if available
\urlstyle{same}
\hypersetup{
  hidelinks,
  pdfcreator={LaTeX via pandoc}}

\author{}
\date{}

\begin{document}
\frontmatter

\mainmatter
\chapter{Menggambar Plot 3D dengan EMT}\label{menggambar-plot-3d-dengan-emt}

Ini adalah pengenalan plot 3D di Euler. Kita membutuhkan plot 3D untuk memvisualisasikan fungsi dari dua variabel.

Euler menggambar fungsi tersebut menggunakan algoritma pengurutan untuk menyembunyikan bagian di latar belakang. Secara umum, Euler menggunakan proyeksi pusat. Standarnya adalah dari kuadran x-y positif menuju titik asal x=y=z=0, tetapi sudut=0° terlihat dari arah sumbu y. Sudut pandang dan tinggi dapat diubah.

Euler dapat merencanakan

\begin{itemize}
\tightlist
\item
  permukaan dengan bayangan dan garis level atau rentang level,
\item
  awan poin,
\item
  kurva parametrik,
\item
  permukaan implisit.
\end{itemize}

Plot 3D dari suatu fungsi menggunakan plot3d. Cara termudah adalah dengan memplot ekspresi dalam x dan y. Parameter r mengatur kisaran plot di sekitar (0,0).

\textgreater aspect(1.5); plot3d(``x\^{}2+sin(y)'',r=pi):

\begin{figure}
\centering
\pandocbounded{\includegraphics[keepaspectratio]{images/EMT_Grafik3D_Alfi Nur Azumah-001.png}}
\caption{images/EMT\_Grafik3D\_Alfi\%20Nur\%20Azumah-001.png}
\end{figure}

\subsection{4.1 Menggambar Grafik Fungsi Dua Variabel dalam Bentuk}\label{menggambar-grafik-fungsi-dua-variabel-dalam-bentuk}

\textbf{Ekspresi Langsung}

Grafik fungsi dua variabel dalam bentuk ekspresi langsung adalah representasi visual dari hubungan matematis antara dua variabel independen yang dinyatakan dalam bentuk persamaan atau ekspresi matematis. Biasanya, grafik ini digunakan untuk menggambarkan hubungan antara dua variabel dalam bidang dua dimensi dan tiga dimensi. Dalam konteks ini, variabel independen (x dan y) adalah variabel input, sedangkan variabel dependen (z) adalah variabel output yang dihasilkan oleh ekspresi matematis.

Rumus umum untuk menggambar grafik fungsi dua variabel dalam bentuk ekspresi langsung adalah:

\[z = f(x, y)\]Dalam rumus ini:

\begin{itemize}
\item
  z adalah variabel dependen yang ingin kita gambar dalam grafik.
\item
  f(x, y) adalah ekspresi matematis yang menghubungkan variabel z dengan variabel independen x dan y. Ekspresi ini dapat berupa fungsi linear, fungsi kuadrat, fungsi akar kuadrat, eksponensial, logaritma, trigonometri, nilai mutlak, atau jenis fungsi matematis lainnya, tergantung pada hubungan yang ingin diilustrasikan.
\end{itemize}

\textbf{1. Grafik Fungsi Linear}

Fungsi linear dua variabel biasanya dinyatakan dalam bentuk

\[f(x,y)=ax+by+c\]

dimana a,b, dan c adalah konstanta. Grafik fungsi linear ini adalah sebuah bidang datar, dan bentuknya akan bervariasi tergantung pada nilai a dan b.

Contoh:

\[f(x,y)=2x+3y+5\]\textgreater plot3d(``2*x+3*y+5''):

\begin{figure}
\centering
\pandocbounded{\includegraphics[keepaspectratio]{images/EMT_Grafik3D_Alfi Nur Azumah-005.png}}
\caption{images/EMT\_Grafik3D\_Alfi\%20Nur\%20Azumah-005.png}
\end{figure}

\[f(x,y)=-2x-3y-5\]\textgreater plot3d(``-2*x-3*y-5''):

\begin{figure}
\centering
\pandocbounded{\includegraphics[keepaspectratio]{images/EMT_Grafik3D_Alfi Nur Azumah-007.png}}
\caption{images/EMT\_Grafik3D\_Alfi\%20Nur\%20Azumah-007.png}
\end{figure}

\[f(x,y)=5x-7y+9\]\textgreater plot3d(``5*x-7*y+9''):

\begin{figure}
\centering
\pandocbounded{\includegraphics[keepaspectratio]{images/EMT_Grafik3D_Alfi Nur Azumah-009.png}}
\caption{images/EMT\_Grafik3D\_Alfi\%20Nur\%20Azumah-009.png}
\end{figure}

\textbf{2. Grafik Fungsi Kuadrat}

Fungsi kuadrat dua variabel biasanya dinyatakan dalam bentuk

\[f(x,y)=ax^2+by^2+cxy+dx+ey+f\]

dimana a, b, c, d, e, dan f adalah konstanta. Grafik fungsi kuadrat ini adalah sebuah permukaan yang dapat memiliki berbagai bentuk tergantung pada nilai-nilai konstantanya.

Contoh:

\[f(x,y)= x^2+y^2+4xy+8x+3y+1\]\textgreater plot3d(``x\textsuperscript{2+y}2+4*x*y+8*x+3*y+1''):

\begin{figure}
\centering
\pandocbounded{\includegraphics[keepaspectratio]{images/EMT_Grafik3D_Alfi Nur Azumah-012.png}}
\caption{images/EMT\_Grafik3D\_Alfi\%20Nur\%20Azumah-012.png}
\end{figure}

\[f(x,y)= 3x^2-y^2+7xy-5\]\textgreater plot3d(``3*x\textsuperscript{2-y}2+7*x*y-5''):

\begin{figure}
\centering
\pandocbounded{\includegraphics[keepaspectratio]{images/EMT_Grafik3D_Alfi Nur Azumah-014.png}}
\caption{images/EMT\_Grafik3D\_Alfi\%20Nur\%20Azumah-014.png}
\end{figure}

\textbf{3. Grafik Fungsi Akar Kuadrat}

Grafik fungsi akar kuadrat dua variabel

\[f(x,y)=\sqrt{x^2+y^2}\]

adalah grafik permukaan yang menggambarkan jarak titik (x, y) dari titik asal (0, 0) dalam ruang tiga dimensi.

Contoh:

\[f(x,y)=\sqrt{x^2+y^2}\]\textgreater plot3d(``sqrt(x\textsuperscript{2+y}2)''):

\begin{figure}
\centering
\pandocbounded{\includegraphics[keepaspectratio]{images/EMT_Grafik3D_Alfi Nur Azumah-017.png}}
\caption{images/EMT\_Grafik3D\_Alfi\%20Nur\%20Azumah-017.png}
\end{figure}

\[f(x,y)=\sqrt{2x^2+7y^2}\]\textgreater plot3d(``sqrt(2*x\textsuperscript{2+7*y}2)''):

\begin{figure}
\centering
\pandocbounded{\includegraphics[keepaspectratio]{images/EMT_Grafik3D_Alfi Nur Azumah-019.png}}
\caption{images/EMT\_Grafik3D\_Alfi\%20Nur\%20Azumah-019.png}
\end{figure}

\[f(x,y)=\sqrt{10x^2+y^2}\]\textgreater plot3d(``sqrt(10*x\textsuperscript{2+y}2)''):

\begin{figure}
\centering
\pandocbounded{\includegraphics[keepaspectratio]{images/EMT_Grafik3D_Alfi Nur Azumah-021.png}}
\caption{images/EMT\_Grafik3D\_Alfi\%20Nur\%20Azumah-021.png}
\end{figure}

\textbf{4. Grafik Fungsi Eksponensial}

Fungsi eksponensial dua variabel bisa dinyatakan

\[f(x,y)=a.b^{xy}\]

dimana a dan b adalah konstanta, x dan y adalah variabel. Fungsi ini menggambarkan pertumbuhan eksponensial yang bergantung pada nilai x dan y.

Contoh:

\[f(x,y)= 2.3^{xy}\]\textgreater plot3d(``2*3\^{}(x*y)''):

\begin{figure}
\centering
\pandocbounded{\includegraphics[keepaspectratio]{images/EMT_Grafik3D_Alfi Nur Azumah-024.png}}
\caption{images/EMT\_Grafik3D\_Alfi\%20Nur\%20Azumah-024.png}
\end{figure}

\[f(x,y)= -3.8^{xy}\]\textgreater plot3d(``-3*8\^{}(x*y)''):

\begin{figure}
\centering
\pandocbounded{\includegraphics[keepaspectratio]{images/EMT_Grafik3D_Alfi Nur Azumah-026.png}}
\caption{images/EMT\_Grafik3D\_Alfi\%20Nur\%20Azumah-026.png}
\end{figure}

\textbf{5. Grafik Fungsi Logaritma}

Grafik fungsi logaritma dua variabel adalah grafik yang menggambarkan nilai logaritma dari suatu ekspresi yang melibatkan dua variabel (biasanya x dan y). Fungsi logaritma dua variabel ini dinyatakan sebagai

\[f(x,y)=log_b(xy)\]

dimana b adalah basis logaritma. Basis logaritma ini dapat berbeda-beda.

Contoh:

\[f(x,y)=log(xy), basis 10\]\textgreater plot3d(``log(x*y)''):

\begin{figure}
\centering
\pandocbounded{\includegraphics[keepaspectratio]{images/EMT_Grafik3D_Alfi Nur Azumah-029.png}}
\caption{images/EMT\_Grafik3D\_Alfi\%20Nur\%20Azumah-029.png}
\end{figure}

\[f(x,y)=log(2x.9y), basis 10\]\textgreater plot3d(``log(2x*9y)''):

\begin{figure}
\centering
\pandocbounded{\includegraphics[keepaspectratio]{images/EMT_Grafik3D_Alfi Nur Azumah-031.png}}
\caption{images/EMT\_Grafik3D\_Alfi\%20Nur\%20Azumah-031.png}
\end{figure}

\textbf{6. Grafik Fungsi Trigonometri}

Fungsi trigonometri dua variabel adalah fungsi matematika yang melibatkan operasi trigonometri (seperti sin, cos, tan) pada kedua variabel x dan y.

Contoh:

\[f(x,y)=sin(x).cos(y)\]\textgreater plot3d(``sin(x)*cos(y)''):

\begin{figure}
\centering
\pandocbounded{\includegraphics[keepaspectratio]{images/EMT_Grafik3D_Alfi Nur Azumah-033.png}}
\caption{images/EMT\_Grafik3D\_Alfi\%20Nur\%20Azumah-033.png}
\end{figure}

\[f(x,y)=sin(2x).cos(y)\]\textgreater plot3d(``sin(2x)*cos(y)''):

\begin{figure}
\centering
\pandocbounded{\includegraphics[keepaspectratio]{images/EMT_Grafik3D_Alfi Nur Azumah-035.png}}
\caption{images/EMT\_Grafik3D\_Alfi\%20Nur\%20Azumah-035.png}
\end{figure}

\[f(x,y)=sin(x).tan(y)\]\textgreater plot3d(``sin(x)*tan(y)''):

\begin{figure}
\centering
\pandocbounded{\includegraphics[keepaspectratio]{images/EMT_Grafik3D_Alfi Nur Azumah-037.png}}
\caption{images/EMT\_Grafik3D\_Alfi\%20Nur\%20Azumah-037.png}
\end{figure}

\[f(x,y)=cos(x).tan(y)\]\textgreater plot3d(``cos(x)*tan(y)''):

\begin{figure}
\centering
\pandocbounded{\includegraphics[keepaspectratio]{images/EMT_Grafik3D_Alfi Nur Azumah-039.png}}
\caption{images/EMT\_Grafik3D\_Alfi\%20Nur\%20Azumah-039.png}
\end{figure}

\[f(x,y)=cosec(x).sec(y)\]\textgreater plot3d(``cosec(x)*sec(y)''):

\begin{figure}
\centering
\pandocbounded{\includegraphics[keepaspectratio]{images/EMT_Grafik3D_Alfi Nur Azumah-041.png}}
\caption{images/EMT\_Grafik3D\_Alfi\%20Nur\%20Azumah-041.png}
\end{figure}

\[f(x,y)=cot(x).cosec(y)\]\textgreater plot3d(``cot(x)*cosec(y)''):

\begin{figure}
\centering
\pandocbounded{\includegraphics[keepaspectratio]{images/EMT_Grafik3D_Alfi Nur Azumah-043.png}}
\caption{images/EMT\_Grafik3D\_Alfi\%20Nur\%20Azumah-043.png}
\end{figure}

\textbf{7. Grafik Fungsi Nilai Mutlak}

Fungsi nilai mutlak dua variabel, juga dikenal sebagai fungsi modul dua variabel, dinyatakan sebagai

\[f(x,y)=|g(x,y)|\]

dimana g(x,y) adalah fungsi dua variabel.

Contoh:

\[f(x,y)=|x^2 - y^2|\]\textgreater plot3d(``abs(x\^{}2 - y\^{}2)''):

\begin{figure}
\centering
\pandocbounded{\includegraphics[keepaspectratio]{images/EMT_Grafik3D_Alfi Nur Azumah-046.png}}
\caption{images/EMT\_Grafik3D\_Alfi\%20Nur\%20Azumah-046.png}
\end{figure}

\[f(x,y)=|x^2 + y^2|\]\textgreater plot3d(``abs(x\^{}2 + y\^{}2)''):

\begin{figure}
\centering
\pandocbounded{\includegraphics[keepaspectratio]{images/EMT_Grafik3D_Alfi Nur Azumah-048.png}}
\caption{images/EMT\_Grafik3D\_Alfi\%20Nur\%20Azumah-048.png}
\end{figure}

\[f(x,y)=|-2x^2 - 5y^2|\]\textgreater plot3d(``abs(-2x\^{}2 - 5y\^{}2)''):

\begin{figure}
\centering
\pandocbounded{\includegraphics[keepaspectratio]{images/EMT_Grafik3D_Alfi Nur Azumah-050.png}}
\caption{images/EMT\_Grafik3D\_Alfi\%20Nur\%20Azumah-050.png}
\end{figure}

\[f(x,y)=|x^2 - 8y^2|\]\textgreater plot3d(``abs(x\^{}2 - 8y\^{}2)''):

\begin{figure}
\centering
\pandocbounded{\includegraphics[keepaspectratio]{images/EMT_Grafik3D_Alfi Nur Azumah-052.png}}
\caption{images/EMT\_Grafik3D\_Alfi\%20Nur\%20Azumah-052.png}
\end{figure}

\textbf{Latihan soal}

Buatkan grafik dari fungsi berikut:

\begin{enumerate}
\def\labelenumi{\arabic{enumi}.}
\tightlist
\item
  \[f(x,y)=8x-3y+7\]\textgreater plot3d(``8*x - 3*y +7''):
\end{enumerate}

\begin{figure}
\centering
\pandocbounded{\includegraphics[keepaspectratio]{images/EMT_Grafik3D_Alfi Nur Azumah-054.png}}
\caption{images/EMT\_Grafik3D\_Alfi\%20Nur\%20Azumah-054.png}
\end{figure}

\begin{enumerate}
\def\labelenumi{\arabic{enumi}.}
\setcounter{enumi}{1}
\tightlist
\item
  \[f(x,y)=cos(5x).sin(9y)\]\textgreater plot3d(``cos(5*x)*sin(9*y)''):
\end{enumerate}

\begin{figure}
\centering
\pandocbounded{\includegraphics[keepaspectratio]{images/EMT_Grafik3D_Alfi Nur Azumah-056.png}}
\caption{images/EMT\_Grafik3D\_Alfi\%20Nur\%20Azumah-056.png}
\end{figure}

\begin{enumerate}
\def\labelenumi{\arabic{enumi}.}
\setcounter{enumi}{2}
\tightlist
\item
  \[f(x,y)=\sqrt{x^2+20y^2}\]\textgreater plot3d(``sqrt(x\textsuperscript{2+20*y}2)''):
\end{enumerate}

\begin{figure}
\centering
\pandocbounded{\includegraphics[keepaspectratio]{images/EMT_Grafik3D_Alfi Nur Azumah-058.png}}
\caption{images/EMT\_Grafik3D\_Alfi\%20Nur\%20Azumah-058.png}
\end{figure}

\subsection{4.2 Menggambar Grafik Fungsi Dua Variabel yang Rumusnya Disimpan dalam Variabel Ekspresi}\label{menggambar-grafik-fungsi-dua-variabel-yang-rumusnya-disimpan-dalam-variabel-ekspresi}

\textbf{Apa yang dimaksud dengan Grafik Fungsi Dua Variabel?}

Grafik fungsi dua variabel adalah representasi visual dari hubungan antara sebuah fungsi matematika dengan dua variabel independen (biasanya disebut sebagai ``x'' dan ``y'') dan variabel dependen (biasanya disebut sebagai ``z'' atau ``f(x, y)'').

Dalam grafik ini, sumbu x dan sumbu y digunakan untuk menggambarkan nilai-nilai dua variabel independen, sementara permukaan atau grafik 3D digunakan untuk menggambarkan nilai-nilai variabel dependen yang dihasilkan oleh fungsi tersebut.

Grafik fungsi dua variabel membantu memvisualisasikan bagaimana nilai variabel dependen (z) berubah seiring perubahan kedua variabel independen (x dan y) sesuai dengan aturan fungsi tersebut.

\textbf{Fungsi matematika yg terlibat dalam Menggambar Grafik}

\textbf{A. Fungsi Dua Variabel}

\begin{enumerate}
\def\labelenumi{\arabic{enumi}.}
\tightlist
\item
  Fungsi Linear
\end{enumerate}

Bentuk umum

f(x, y) = ax + by + c, di mana a, b, dan c adalah konstanta. Grafiknya adalah bidang datar.

\begin{enumerate}
\def\labelenumi{\arabic{enumi}.}
\setcounter{enumi}{1}
\tightlist
\item
  Fungsi Kuadratik
\end{enumerate}

Bentuk umum f(x, y) = ax\^{}2 + by\^{}2 + cxy + dx + ey + f.

Grafiknya dapat berupa permukaan yang berbentuk paraboloid, baik terbuka ke atas atau ke bawah.

\begin{enumerate}
\def\labelenumi{\arabic{enumi}.}
\setcounter{enumi}{2}
\tightlist
\item
  Fungsi Trigonometri
\end{enumerate}

Bentuk umum sinus dan cosinus

(x, y) = sin(x) + cos(y)

akan menghasilkan permukaan yang berulang-ulang naik dan turun.

\begin{enumerate}
\def\labelenumi{\arabic{enumi}.}
\setcounter{enumi}{3}
\tightlist
\item
  Fungsi Pecahan
\end{enumerate}

Bentuk umum f(x, y) = g(x, y) / h(x, y), di mana g(x, y) dan h(x, y) adalah fungsi-fungsi lain.

grafiknya dapat menghasilkan berbagai pola yang tergantung pada sifat fungsi g dan h.

\textbf{Menggambar Grafik Fungsi}

Perintah yang digunakan untuk menggambar grafik fungsi dalam EMT yaitu dengan menggunakan plot3d.

Untuk menampilkan grafik, akhiri perintah plot3d dengan tanda (:).

Tanda (:) akan menampilkan grafik di layar yang berbeda.

\textgreater plot3d(``x\textsuperscript{2+y}2''):

\begin{figure}
\centering
\pandocbounded{\includegraphics[keepaspectratio]{images/EMT_Grafik3D_Alfi Nur Azumah-059.png}}
\caption{images/EMT\_Grafik3D\_Alfi\%20Nur\%20Azumah-059.png}
\end{figure}

\textgreater plot3d(``x\^{}3+x*sin(y)'',-5,5,0,6*pi):

\begin{figure}
\centering
\pandocbounded{\includegraphics[keepaspectratio]{images/EMT_Grafik3D_Alfi Nur Azumah-060.png}}
\caption{images/EMT\_Grafik3D\_Alfi\%20Nur\%20Azumah-060.png}
\end{figure}

\textbf{Menyimpan Variabel Ekspresi}

Untuk menyimpan sebuah fungsi, dapat dilakukan dengan menggunakan perintah function. Lalu, ketika ingin memanggil atau membuat grafik dari fungsi tersebut, cukup dengan memanggil nama fungsi tersebut.

\textgreater function a(x,y):= x\textsuperscript{2+y}3

\textgreater plot3d(``a''):

\begin{figure}
\centering
\pandocbounded{\includegraphics[keepaspectratio]{images/EMT_Grafik3D_Alfi Nur Azumah-061.png}}
\caption{images/EMT\_Grafik3D\_Alfi\%20Nur\%20Azumah-061.png}
\end{figure}

\textgreater function f(x,y):= x\textsuperscript{3-y}2

\textgreater plot3d(``f''):

\begin{figure}
\centering
\pandocbounded{\includegraphics[keepaspectratio]{images/EMT_Grafik3D_Alfi Nur Azumah-062.png}}
\caption{images/EMT\_Grafik3D\_Alfi\%20Nur\%20Azumah-062.png}
\end{figure}

\textbf{Contoh Latihan Soal}

\[f(x,y)= x^3+y^4\]\textgreater function f(x,y):= x\textsuperscript{3+y}4

\textgreater plot3d(``f''):

\begin{figure}
\centering
\pandocbounded{\includegraphics[keepaspectratio]{images/EMT_Grafik3D_Alfi Nur Azumah-064.png}}
\caption{images/EMT\_Grafik3D\_Alfi\%20Nur\%20Azumah-064.png}
\end{figure}

\textgreater plot3d(``a'',\textgreater user, \ldots{}\\
\textgreater{} title=``Turn with the vector keys (press return to finish)''):

\begin{figure}
\centering
\pandocbounded{\includegraphics[keepaspectratio]{images/EMT_Grafik3D_Alfi Nur Azumah-065.png}}
\caption{images/EMT\_Grafik3D\_Alfi\%20Nur\%20Azumah-065.png}
\end{figure}

Perintah ini mengizinkan pengguna untuk menggambar grafik 3D dari fungsi yang mereka masukkan sendiri, serta memberikan petunjuk interaktif tentang cara berinteraksi dengan grafik. Pengguna dapat memutar tampilan grafik menggunakan tombol arah pada keyboard, dan menekan ``return'' untuk menyelesaikan interaksi.

\textgreater plot3d(``a'',r=5,n=80,fscale=4,scale=1.2,frame=3):

\begin{figure}
\centering
\pandocbounded{\includegraphics[keepaspectratio]{images/EMT_Grafik3D_Alfi Nur Azumah-066.png}}
\caption{images/EMT\_Grafik3D\_Alfi\%20Nur\%20Azumah-066.png}
\end{figure}

perintah ini akan menggambar grafik tiga dimensi dari fungsi ``a'' dalam rentang x dan y dari -5 hingga 5, dengan 80 titik untuk detail yang lebih halus. Nilai fungsi akan diperbesar 4 kali, dan grafik akan ditampilkan dengan skala 1.2 untuk tampilan yang lebih jelas. Bingkai grafis akan ditentukan oleh parameter frame=3.

\textgreater view

\begin{verbatim}
[5,  2.6,  2,  0.4]
\end{verbatim}

\textgreater plot3d(``a'',distance=3,zoom=2,angle=0,height=0):

\begin{figure}
\centering
\pandocbounded{\includegraphics[keepaspectratio]{images/EMT_Grafik3D_Alfi Nur Azumah-067.png}}
\caption{images/EMT\_Grafik3D\_Alfi\%20Nur\%20Azumah-067.png}
\end{figure}

\textgreater plot3d(``x\textsuperscript{4+y}2'',a=0,b=1,c=-1,d=1,angle=-20°,height=20°, \ldots{}\\
\textgreater{} center={[}0.4,0,0{]},zoom=5):

\begin{figure}
\centering
\pandocbounded{\includegraphics[keepaspectratio]{images/EMT_Grafik3D_Alfi Nur Azumah-068.png}}
\caption{images/EMT\_Grafik3D\_Alfi\%20Nur\%20Azumah-068.png}
\end{figure}

\textgreater plot3d(``a'',r=2,\textless fscale,\textless scale,distance=13,height=50°, \ldots{}\\
\textgreater{} center={[}0,0,-2{]},frame=3):

\begin{figure}
\centering
\pandocbounded{\includegraphics[keepaspectratio]{images/EMT_Grafik3D_Alfi Nur Azumah-069.png}}
\caption{images/EMT\_Grafik3D\_Alfi\%20Nur\%20Azumah-069.png}
\end{figure}

\textgreater plot3d(``a'',r=5,\textgreater polar, \ldots{}\\
\textgreater{} fscale=2,\textgreater hue,n=100,zoom=4,\textgreater contour,color=blue):

\begin{figure}
\centering
\pandocbounded{\includegraphics[keepaspectratio]{images/EMT_Grafik3D_Alfi Nur Azumah-070.png}}
\caption{images/EMT\_Grafik3D\_Alfi\%20Nur\%20Azumah-070.png}
\end{figure}

\textgreater plot3d(``x'',``a'',``y'',r=2,zoom=3.5,frame=3):

\begin{figure}
\centering
\pandocbounded{\includegraphics[keepaspectratio]{images/EMT_Grafik3D_Alfi Nur Azumah-071.png}}
\caption{images/EMT\_Grafik3D\_Alfi\%20Nur\%20Azumah-071.png}
\end{figure}

\textbf{B. Fungsi Linear}

\textgreater function e(x,y):= 20x+10y-5

\textgreater plot3d(``e''):

\begin{figure}
\centering
\pandocbounded{\includegraphics[keepaspectratio]{images/EMT_Grafik3D_Alfi Nur Azumah-072.png}}
\caption{images/EMT\_Grafik3D\_Alfi\%20Nur\%20Azumah-072.png}
\end{figure}

\textgreater plot3d(``e'',\textgreater user, \ldots{}\\
\textgreater{} title=``Turn with the vector keys (press return to finish)''):

\begin{figure}
\centering
\pandocbounded{\includegraphics[keepaspectratio]{images/EMT_Grafik3D_Alfi Nur Azumah-073.png}}
\caption{images/EMT\_Grafik3D\_Alfi\%20Nur\%20Azumah-073.png}
\end{figure}

\textgreater plot3d(``e'',r=10,n=80,fscale=4,scale=1.2,frame=3):

\begin{figure}
\centering
\pandocbounded{\includegraphics[keepaspectratio]{images/EMT_Grafik3D_Alfi Nur Azumah-074.png}}
\caption{images/EMT\_Grafik3D\_Alfi\%20Nur\%20Azumah-074.png}
\end{figure}

\textgreater view

\begin{verbatim}
[5,  2.6,  2,  0.4]
\end{verbatim}

\textgreater plot3d(``e'',distance=3,zoom=2,angle=0,height=0):

\begin{figure}
\centering
\pandocbounded{\includegraphics[keepaspectratio]{images/EMT_Grafik3D_Alfi Nur Azumah-075.png}}
\caption{images/EMT\_Grafik3D\_Alfi\%20Nur\%20Azumah-075.png}
\end{figure}

\textgreater plot3d(``e'',a=0,b=1,c=-1,d=1,angle=-20°,height=20°, \ldots{}\\
\textgreater{} center={[}0.4,0,0{]},zoom=5):

\begin{figure}
\centering
\pandocbounded{\includegraphics[keepaspectratio]{images/EMT_Grafik3D_Alfi Nur Azumah-076.png}}
\caption{images/EMT\_Grafik3D\_Alfi\%20Nur\%20Azumah-076.png}
\end{figure}

\textgreater plot3d(``e'',r=2,\textless fscale,\textless scale,distance=13,height=50°, \ldots{}\\
\textgreater{} center={[}0,0,-2{]},frame=3):

\begin{figure}
\centering
\pandocbounded{\includegraphics[keepaspectratio]{images/EMT_Grafik3D_Alfi Nur Azumah-077.png}}
\caption{images/EMT\_Grafik3D\_Alfi\%20Nur\%20Azumah-077.png}
\end{figure}

\textgreater plot3d(``e'',r=5,\textgreater polar, \ldots{}\\
\textgreater{} fscale=2,\textgreater hue,n=100,zoom=4,\textgreater contour,color=gray):

\begin{figure}
\centering
\pandocbounded{\includegraphics[keepaspectratio]{images/EMT_Grafik3D_Alfi Nur Azumah-078.png}}
\caption{images/EMT\_Grafik3D\_Alfi\%20Nur\%20Azumah-078.png}
\end{figure}

\textgreater plot3d(``x'',``e'',``y'',r=2,zoom=3.5,frame=3):

\begin{figure}
\centering
\pandocbounded{\includegraphics[keepaspectratio]{images/EMT_Grafik3D_Alfi Nur Azumah-079.png}}
\caption{images/EMT\_Grafik3D\_Alfi\%20Nur\%20Azumah-079.png}
\end{figure}

\textbf{C. Fungsi Trigonometri}

\textgreater function f(x,y):= sin(x+y)

\textgreater plot3d(``f'')

\textgreater plot3d(``f'',\textgreater user, \ldots{}\\
\textgreater{} title=``Turn with the vector keys (press return to finish)''):

\begin{figure}
\centering
\pandocbounded{\includegraphics[keepaspectratio]{images/EMT_Grafik3D_Alfi Nur Azumah-080.png}}
\caption{images/EMT\_Grafik3D\_Alfi\%20Nur\%20Azumah-080.png}
\end{figure}

\textgreater plot3d(``f'',r=10,n=80,fscale=4,scale=1.2,frame=3):

\begin{figure}
\centering
\pandocbounded{\includegraphics[keepaspectratio]{images/EMT_Grafik3D_Alfi Nur Azumah-081.png}}
\caption{images/EMT\_Grafik3D\_Alfi\%20Nur\%20Azumah-081.png}
\end{figure}

\textgreater view

\begin{verbatim}
[5,  2.6,  2,  0.4]
\end{verbatim}

\textgreater plot3d(``f'',distance=3,zoom=2,angle=0,height=0):

\begin{figure}
\centering
\pandocbounded{\includegraphics[keepaspectratio]{images/EMT_Grafik3D_Alfi Nur Azumah-082.png}}
\caption{images/EMT\_Grafik3D\_Alfi\%20Nur\%20Azumah-082.png}
\end{figure}

\textgreater plot3d(``f'',a=0,b=1,c=-1,d=1,angle=-20°,height=20°, \ldots{}\\
\textgreater{} center={[}0.4,0,0{]},zoom=5):

\begin{figure}
\centering
\pandocbounded{\includegraphics[keepaspectratio]{images/EMT_Grafik3D_Alfi Nur Azumah-083.png}}
\caption{images/EMT\_Grafik3D\_Alfi\%20Nur\%20Azumah-083.png}
\end{figure}

\textgreater plot3d(``f'',r=5,\textgreater polar, \ldots{}\\
\textgreater{} fscale=2,\textgreater hue,n=100,zoom=4,\textgreater contour,color=gray):

\begin{figure}
\centering
\pandocbounded{\includegraphics[keepaspectratio]{images/EMT_Grafik3D_Alfi Nur Azumah-084.png}}
\caption{images/EMT\_Grafik3D\_Alfi\%20Nur\%20Azumah-084.png}
\end{figure}

\subsection{4.3 Menggambar Grafik Fungsi Dua Variabel yang Fungsinya}\label{menggambar-grafik-fungsi-dua-variabel-yang-fungsinya}

\textbf{Didefinisikan sebagai Fungsi Numerik}

Sebelum masuk ke cara memvisualisasikan grafik, perlu diketahui apa itu fungsi dua variabel dan apa itu fungsi numerik.

\begin{enumerate}
\def\labelenumi{\arabic{enumi}.}
\tightlist
\item
  Fungsi Dua Variabel
\end{enumerate}

Fungsi dua variabel adalah jenis fungsi di mana ada dua variabel bebas (biasanya dinotasikan sebagai x dan y) yang menentukan nilai dari fungsi tersebut. Dengan kata lain, untuk setiap kombinasi nilai dari x dan y, fungsi ini akan menghasilkan satu nilai output tertentu. Fungsi dari dua variabel yang mana setiap kombinasi nilai dari kedua variabel tersebut menghasilkan sebuah nilai tunggal.

\begin{enumerate}
\def\labelenumi{\arabic{enumi}.}
\setcounter{enumi}{1}
\tightlist
\item
  Fungsi Numerik
\end{enumerate}

Fungsi dimana setiap pasangan variabel independen adalah angka atau bilangan nyata. Secara sederhana, ketika memasukkan angka atau bilangan nyata ke variabel-variabel dalam fungsi maka hasil akhir yang dihasilkan juga angka atau bilangan nyata, bukan simbol atau ekspresi yang belum dihitung.

Contoh:

ada fungsi

\[f(x,y)=2x+y\]Ketika kita memasukkan bilangan nyata ke x dan y maka akan dihasilkan suatu bilangan nyata juga. Misal masukkan x=1 dan y=1. Akan diperoleh

\[2(1)+1=3\] \textbf{Visualisasi Grafik}

Untuk memvisualisaikan fungsi dua variabel dengan fungsinya didefinisikan sebagai fungsi numerik, akan dibuat grafik 3D dengan sintaks plot3d. Untuk membedakan fungsi numerik dengan simbolik, pada kali ini untuk setiap fungsi numerik dua variabel hanya akan memuat variabel x dan y. Namun, dalam pemakaian secara umum, bisa digunakan variabel apapun.

Penulisan Sintaks:

\begin{enumerate}
\def\labelenumi{\arabic{enumi})}
\tightlist
\item
  definisikan fungsi numerik
\end{enumerate}

function f(x,y):= ax+by dengan a dan b adalah suatu konstanta dan fungsi tidak selalu direpresentastikan dengan f tetapi bisa dengan huruf apapun. Contoh : g(x,y)

\begin{enumerate}
\def\labelenumi{\arabic{enumi})}
\setcounter{enumi}{1}
\tightlist
\item
  sintaks plot3d
\end{enumerate}

plot3d(``f''):

Contoh Visualisasi Grafik:

\begin{enumerate}
\def\labelenumi{\arabic{enumi}.}
\tightlist
\item
  Visualisasi grafik fungsi linear dua variabel
\end{enumerate}

\[f(x,y) = 3x+7y\]\textgreater function f(x,y):= 2*x+5*y

\textgreater plot3d(``f''):

\begin{figure}
\centering
\pandocbounded{\includegraphics[keepaspectratio]{images/EMT_Grafik3D_Alfi Nur Azumah-088.png}}
\caption{images/EMT\_Grafik3D\_Alfi\%20Nur\%20Azumah-088.png}
\end{figure}

\begin{enumerate}
\def\labelenumi{\arabic{enumi}.}
\setcounter{enumi}{1}
\tightlist
\item
  Visualisasi grafik fungsi kuadrat dua variabel
\end{enumerate}

\[f(x,y)=x^2+2xy+y^2\]\textgreater function f(x,y):= x\textsuperscript{2+2*x*y+y}2

\textgreater plot3d(``f''):

\begin{figure}
\centering
\pandocbounded{\includegraphics[keepaspectratio]{images/EMT_Grafik3D_Alfi Nur Azumah-090.png}}
\caption{images/EMT\_Grafik3D\_Alfi\%20Nur\%20Azumah-090.png}
\end{figure}

\begin{enumerate}
\def\labelenumi{\arabic{enumi}.}
\setcounter{enumi}{2}
\tightlist
\item
  Visualisasi Grafik Fungsi Eksponen Dua Variabel
\end{enumerate}

\[g(x,y) = x^{2y+8}\]\textgreater function g(x,y):= x\^{}(2y+8)

\textgreater plot3d(``g''):

\begin{figure}
\centering
\pandocbounded{\includegraphics[keepaspectratio]{images/EMT_Grafik3D_Alfi Nur Azumah-092.png}}
\caption{images/EMT\_Grafik3D\_Alfi\%20Nur\%20Azumah-092.png}
\end{figure}

\begin{enumerate}
\def\labelenumi{\arabic{enumi}.}
\setcounter{enumi}{3}
\tightlist
\item
  Grafik Fungsi Logaritma Dua Variabel
\end{enumerate}

\[f(x,y)=\log(xy)\]\textgreater function f(x,y):= log(x*y)

\textgreater plot3d(``f''):

\begin{figure}
\centering
\pandocbounded{\includegraphics[keepaspectratio]{images/EMT_Grafik3D_Alfi Nur Azumah-094.png}}
\caption{images/EMT\_Grafik3D\_Alfi\%20Nur\%20Azumah-094.png}
\end{figure}

\begin{enumerate}
\def\labelenumi{\arabic{enumi}.}
\setcounter{enumi}{4}
\tightlist
\item
  Grafik Fungsi Trigonometri Dua Variabel
\end{enumerate}

\[h(x,y)=sin(xy)cos(y)\]\textgreater function h(x,y):= sin(x*y)*cos(y)

\textgreater plot3d(``h''):

\begin{figure}
\centering
\pandocbounded{\includegraphics[keepaspectratio]{images/EMT_Grafik3D_Alfi Nur Azumah-096.png}}
\caption{images/EMT\_Grafik3D\_Alfi\%20Nur\%20Azumah-096.png}
\end{figure}

\begin{enumerate}
\def\labelenumi{\arabic{enumi}.}
\setcounter{enumi}{5}
\tightlist
\item
  Grafik Fungsi Nilai Mutlak Dua Variabel
\end{enumerate}

\[i(x,y)=|(2x+y)|\]\textgreater function i(x,y):= abs(2x+y)

\textgreater plot3d(``i''):

\begin{figure}
\centering
\pandocbounded{\includegraphics[keepaspectratio]{images/EMT_Grafik3D_Alfi Nur Azumah-098.png}}
\caption{images/EMT\_Grafik3D\_Alfi\%20Nur\%20Azumah-098.png}
\end{figure}

\textbf{Latihan Soal}

Buatlah visualisasi grafik dari fungsi berikut ini!

\begin{enumerate}
\def\labelenumi{\arabic{enumi}.}
\tightlist
\item
\end{enumerate}

\[f(x,y)=2^x+3y\]\textgreater function f(x,y):=2\^{}x+3*y

\textgreater plot3d (``f''):

\begin{figure}
\centering
\pandocbounded{\includegraphics[keepaspectratio]{images/EMT_Grafik3D_Alfi Nur Azumah-100.png}}
\caption{images/EMT\_Grafik3D\_Alfi\%20Nur\%20Azumah-100.png}
\end{figure}

\begin{enumerate}
\def\labelenumi{\arabic{enumi}.}
\setcounter{enumi}{1}
\tightlist
\item
\end{enumerate}

\[g(x,y)=sin(x^2y+1)\]\textgreater function g(x,y):= sin(x\^{}2*y+1)

\textgreater plot3d(``g''):

\begin{figure}
\centering
\pandocbounded{\includegraphics[keepaspectratio]{images/EMT_Grafik3D_Alfi Nur Azumah-102.png}}
\caption{images/EMT\_Grafik3D\_Alfi\%20Nur\%20Azumah-102.png}
\end{figure}

\begin{enumerate}
\def\labelenumi{\arabic{enumi}.}
\setcounter{enumi}{2}
\tightlist
\item
\end{enumerate}

\[h(x,y)=|4x+sin(y^3+1)|\]\textgreater function h(x,y):=abs(4*x + sin(y\^{}3+1))

\textgreater plot3d(``h''):

\begin{figure}
\centering
\pandocbounded{\includegraphics[keepaspectratio]{images/EMT_Grafik3D_Alfi Nur Azumah-104.png}}
\caption{images/EMT\_Grafik3D\_Alfi\%20Nur\%20Azumah-104.png}
\end{figure}

\subsection{4.4 Menggambar Grafik Fungsi Dua Variabel yang Fungsinya}\label{menggambar-grafik-fungsi-dua-variabel-yang-fungsinya-1}

\textbf{Didefinisikan sebagai Fungsi Simbolik}

Grafik Fungsi dua variabel yang fungsinya didefinisikan sebagai fungsi simbolik adalah suatu grafik yang memvisualisasikan Persamaan Linear Dua Variabel (PLDV) dalam koordinat kartesius dengan fungsinya merupakan fungsi simbolik.

Proses visualisasi ini memungkinkan Kita untuk melihat dan memahami bagaimana fungsi tersebut berperilaku dalam tiga dimensi.

Sedangkan yang dimaksud dengan fungsi simbolik yaitu fungsi yang didefinisikan dalam bentuk matematika simbolik, artinya kita memiliki ekspresi matematika yang menggambarkan hubungan antara dua variabel. misalnya, jika kita memiliki fungsi

\[f(x, y) = x^2 + y^2\]

kita dapat menggambar grafiknya untuk melihat bentuk permukaan yang dihasilkan oleh fungsi ini dalam tiga dimensi. Grafik ini mungkin akan berupa tumpukan parabola yang membentuk kerucut.

Karakteristik fungsi simbolik adalah dengan mengganti tanda := menjadi \&=

Perbedaan utama antara fungsi numerik dan fungsi simbolik adalah bahwa fungsi numerik memberikan hasil numerik secara langsung (menghasilkan angka), sementara fungsi simbolik memungkinkan kita untuk bekerja dengan simbol matematika sebelum menghitung nilai numeriknya. Pilihan antara keduanya tergantung pada kebutuhan analisis matematika yang kita lakukan.

Tujuan menggambar grafik fungsi dua variabel adalah untuk memahami pola, sifat, dan hubungan antara dua variabel tersebut secara visual, yang dapat membantu dalam analisis dan pemahaman masalah matematika atau ilmu pengetahuan yang melibatkan fungsi ini.

\textbf{Langkah-langkah Membuat Grafik}

\begin{enumerate}
\def\labelenumi{\arabic{enumi})}
\tightlist
\item
  Definisikan fungsinya terlebih dahulu. Tentukan fungsi dua variabel yang akan divisualisasikan dalam bentuk simbolik.
\end{enumerate}

Misal diambil fungsi

\[f(x,y)=x^2+y^2\]

Maka perintah yang dapat dituliskan yaitu:

\textgreater function f(x,y)\&= x\textsuperscript{2+y}2;

\begin{enumerate}
\def\labelenumi{\arabic{enumi})}
\setcounter{enumi}{1}
\tightlist
\item
  Panggil fungsinya
\end{enumerate}

\textgreater plot3d(``f(x,y)''):

\begin{figure}
\centering
\pandocbounded{\includegraphics[keepaspectratio]{images/EMT_Grafik3D_Alfi Nur Azumah-107.png}}
\caption{images/EMT\_Grafik3D\_Alfi\%20Nur\%20Azumah-107.png}
\end{figure}

\begin{enumerate}
\def\labelenumi{\arabic{enumi})}
\setcounter{enumi}{2}
\tightlist
\item
  Tentukan rentang variabelnya
\end{enumerate}

\textgreater plot3d(``f(x,y)'',-5,5,0,6*pi):

\begin{figure}
\centering
\pandocbounded{\includegraphics[keepaspectratio]{images/EMT_Grafik3D_Alfi Nur Azumah-108.png}}
\caption{images/EMT\_Grafik3D\_Alfi\%20Nur\%20Azumah-108.png}
\end{figure}

\textbf{Membuat Grafik Fungsi Linear}

\[g(x,y)=x+3y\]\textgreater function g(x,y) \&= x+3*y;

\textgreater plot3d(``g(x,y)''):

\begin{figure}
\centering
\pandocbounded{\includegraphics[keepaspectratio]{images/EMT_Grafik3D_Alfi Nur Azumah-110.png}}
\caption{images/EMT\_Grafik3D\_Alfi\%20Nur\%20Azumah-110.png}
\end{figure}

\[M(x,y)=-2x-4y\]\textgreater function M(x,y) \&= -2*x-4*y;

\textgreater plot3d(``M(x,y)''):

\begin{figure}
\centering
\pandocbounded{\includegraphics[keepaspectratio]{images/EMT_Grafik3D_Alfi Nur Azumah-112.png}}
\caption{images/EMT\_Grafik3D\_Alfi\%20Nur\%20Azumah-112.png}
\end{figure}

Rentang variabel:

\textgreater plot3d(``M(x,y)'',-2,2,0,6*pi):

\begin{figure}
\centering
\pandocbounded{\includegraphics[keepaspectratio]{images/EMT_Grafik3D_Alfi Nur Azumah-113.png}}
\caption{images/EMT\_Grafik3D\_Alfi\%20Nur\%20Azumah-113.png}
\end{figure}

\textbf{Membuat Grafik Fungsi Kuadrat}

\[Q(x,y)=x^2-y^2\]\textgreater function Q(x,y) \&= x\^{}2 - y\^{}2;

\textgreater plot3d(``Q(x,y)''):

\begin{figure}
\centering
\pandocbounded{\includegraphics[keepaspectratio]{images/EMT_Grafik3D_Alfi Nur Azumah-115.png}}
\caption{images/EMT\_Grafik3D\_Alfi\%20Nur\%20Azumah-115.png}
\end{figure}

\[P(x,y)=3x^2-4xy+2y^2\]\textgreater function P(x,y) \&= 3*x\textsuperscript{2-4*x*y+2*y}2;

\textgreater plot3d(``P(x,y)''):

\begin{figure}
\centering
\pandocbounded{\includegraphics[keepaspectratio]{images/EMT_Grafik3D_Alfi Nur Azumah-117.png}}
\caption{images/EMT\_Grafik3D\_Alfi\%20Nur\%20Azumah-117.png}
\end{figure}

Rentang variabel:

\textgreater plot3d(``P(x,y)'',-10,10,0,9*pi):

\begin{figure}
\centering
\pandocbounded{\includegraphics[keepaspectratio]{images/EMT_Grafik3D_Alfi Nur Azumah-118.png}}
\caption{images/EMT\_Grafik3D\_Alfi\%20Nur\%20Azumah-118.png}
\end{figure}

\textbf{Membuat Grafik Fungsi Akar Kuadrat}

\[A(x,y)= \sqrt{10x^2+2y^2}\]\textgreater function A(x,y) \&= sqrt(10*x\textsuperscript{2+2*y}2);

\textgreater plot3d(``A''):

\begin{figure}
\centering
\pandocbounded{\includegraphics[keepaspectratio]{images/EMT_Grafik3D_Alfi Nur Azumah-120.png}}
\caption{images/EMT\_Grafik3D\_Alfi\%20Nur\%20Azumah-120.png}
\end{figure}

\[W(x,y)= \sqrt{x^2+2y^2}\]\textgreater function W(x,y) \&= sqrt(x\textsuperscript{2+2*y}2);

\textgreater plot3d(``W''):

\begin{figure}
\centering
\pandocbounded{\includegraphics[keepaspectratio]{images/EMT_Grafik3D_Alfi Nur Azumah-122.png}}
\caption{images/EMT\_Grafik3D\_Alfi\%20Nur\%20Azumah-122.png}
\end{figure}

Rentang Variabel:

\textgreater plot3d(``W(x,y)'',-20,100,0,5*pi):

\begin{figure}
\centering
\pandocbounded{\includegraphics[keepaspectratio]{images/EMT_Grafik3D_Alfi Nur Azumah-123.png}}
\caption{images/EMT\_Grafik3D\_Alfi\%20Nur\%20Azumah-123.png}
\end{figure}

\textbf{Membuat Grafik Fungsi Eksponensial}

\[F(x,y)= 9.12^{xy}\]\textgreater function F(x,y) \&= 9*12\^{}(x*y);

\textgreater plot3d(``F''):

\begin{figure}
\centering
\pandocbounded{\includegraphics[keepaspectratio]{images/EMT_Grafik3D_Alfi Nur Azumah-125.png}}
\caption{images/EMT\_Grafik3D\_Alfi\%20Nur\%20Azumah-125.png}
\end{figure}

\[H(x,y)=5.(-20)^{xy}\]\textgreater function H(x,y) \&= 5*-12\^{}(x*y);

\textgreater plot3d(``H''):

\begin{figure}
\centering
\pandocbounded{\includegraphics[keepaspectratio]{images/EMT_Grafik3D_Alfi Nur Azumah-127.png}}
\caption{images/EMT\_Grafik3D\_Alfi\%20Nur\%20Azumah-127.png}
\end{figure}

\[T(x,y)=(-21).5^{xy}\]\textgreater function T(x,y) \&= -21*-5\^{}(x*y);

\textgreater plot3d(``T''):

\begin{figure}
\centering
\pandocbounded{\includegraphics[keepaspectratio]{images/EMT_Grafik3D_Alfi Nur Azumah-129.png}}
\caption{images/EMT\_Grafik3D\_Alfi\%20Nur\%20Azumah-129.png}
\end{figure}

\textbf{Membuat Grafik Fungsi Logaritma}

\[B(x,y)=log(x.2y), Basis 10\]\textgreater function B(x,y) \&= log(x*2*y);

\textgreater plot3d(``B''):

\begin{figure}
\centering
\pandocbounded{\includegraphics[keepaspectratio]{images/EMT_Grafik3D_Alfi Nur Azumah-131.png}}
\caption{images/EMT\_Grafik3D\_Alfi\%20Nur\%20Azumah-131.png}
\end{figure}

\[C(x,y)=log(30x.5y), basis 10\]\textgreater function C(x,y) \&= log(30*x*5*y);

\textgreater plot3d(``C''):

\begin{figure}
\centering
\pandocbounded{\includegraphics[keepaspectratio]{images/EMT_Grafik3D_Alfi Nur Azumah-133.png}}
\caption{images/EMT\_Grafik3D\_Alfi\%20Nur\%20Azumah-133.png}
\end{figure}

\textbf{Membuat Grafik Fungsi Trigonometri}

\[D(x,y)=sin(2x).cos(3y)\]\textgreater function D(x,y) \&= sin(2*x)*cos(3*y);

\textgreater plot3d(``D''):

\begin{figure}
\centering
\pandocbounded{\includegraphics[keepaspectratio]{images/EMT_Grafik3D_Alfi Nur Azumah-135.png}}
\caption{images/EMT\_Grafik3D\_Alfi\%20Nur\%20Azumah-135.png}
\end{figure}

\[J(x,y)=sin(2x).tan(y)\]\textgreater function J(x,y) \&= sin(2*x)*tan(y);

\textgreater plot3d(``J''):

\begin{figure}
\centering
\pandocbounded{\includegraphics[keepaspectratio]{images/EMT_Grafik3D_Alfi Nur Azumah-137.png}}
\caption{images/EMT\_Grafik3D\_Alfi\%20Nur\%20Azumah-137.png}
\end{figure}

\[G(x,y)=sec(2x).cot(5y)\]\textgreater function G(x,y) \&= sec(2*x)*cot(y);

\textgreater plot3d(``G''):

\begin{figure}
\centering
\pandocbounded{\includegraphics[keepaspectratio]{images/EMT_Grafik3D_Alfi Nur Azumah-139.png}}
\caption{images/EMT\_Grafik3D\_Alfi\%20Nur\%20Azumah-139.png}
\end{figure}

\textbf{Membuat Grafik Fungsi Nilai Mutlak}

\[T(x,y)=|2x^2-y^2|\]\textgreater function T(x,y) \&= abs(2*x\^{}2 - y\^{}2);

\textgreater plot3d(``T''):

\begin{figure}
\centering
\pandocbounded{\includegraphics[keepaspectratio]{images/EMT_Grafik3D_Alfi Nur Azumah-141.png}}
\caption{images/EMT\_Grafik3D\_Alfi\%20Nur\%20Azumah-141.png}
\end{figure}

\[M(x,y)=|-10x^2-3y^2|\]\textgreater function M(x,y) \&= abs(-10*x\^{}2 - 3*y\^{}2);

\textgreater plot3d(``M''):

\begin{figure}
\centering
\pandocbounded{\includegraphics[keepaspectratio]{images/EMT_Grafik3D_Alfi Nur Azumah-143.png}}
\caption{images/EMT\_Grafik3D\_Alfi\%20Nur\%20Azumah-143.png}
\end{figure}

Rentang Variabel :

\textgreater plot3d(``M(x,y)'',-15,2,0,8*pi):

\begin{figure}
\centering
\pandocbounded{\includegraphics[keepaspectratio]{images/EMT_Grafik3D_Alfi Nur Azumah-144.png}}
\caption{images/EMT\_Grafik3D\_Alfi\%20Nur\%20Azumah-144.png}
\end{figure}

\textbf{Latihan}

Buatlah grafik dari fungsi berikut:

\[A(x,y)=x^2y+3y^2\]\textgreater function A(x,y) \&= x\textsuperscript{2*y+3*y}2;

\textgreater plot3d (``A''):

\begin{figure}
\centering
\pandocbounded{\includegraphics[keepaspectratio]{images/EMT_Grafik3D_Alfi Nur Azumah-146.png}}
\caption{images/EMT\_Grafik3D\_Alfi\%20Nur\%20Azumah-146.png}
\end{figure}

\[B(x,y)=y^2-2x^2y+4x^3+20x^2\]\textgreater function B(x,y)\&= y\textsuperscript{2-2*x}2*y+4*x\textsuperscript{3+20*x}2;

\textgreater plot3d(``B''):

\begin{figure}
\centering
\pandocbounded{\includegraphics[keepaspectratio]{images/EMT_Grafik3D_Alfi Nur Azumah-148.png}}
\caption{images/EMT\_Grafik3D\_Alfi\%20Nur\%20Azumah-148.png}
\end{figure}

\[C(x,y)=cosec(9x)-tan(2y)\]\textgreater function C(x,y)\&= cosec(9*x)-tan(2*y);

\textgreater plot3d (``C''):

\begin{figure}
\centering
\pandocbounded{\includegraphics[keepaspectratio]{images/EMT_Grafik3D_Alfi Nur Azumah-150.png}}
\caption{images/EMT\_Grafik3D\_Alfi\%20Nur\%20Azumah-150.png}
\end{figure}

Beri rentang variabel untuk fungsi C(x,y):

-100,50,0,2*pi

\textgreater plot3d(``C(x,y)'',-100,50,0,2*pi):

\begin{figure}
\centering
\pandocbounded{\includegraphics[keepaspectratio]{images/EMT_Grafik3D_Alfi Nur Azumah-151.png}}
\caption{images/EMT\_Grafik3D\_Alfi\%20Nur\%20Azumah-151.png}
\end{figure}

\subsection{\texorpdfstring{4.5 Menggambar Data \(x\), \(y\), \(z\) pada ruang Tiga Dimensi (3D)}{4.5 Menggambar Data x, y, z pada ruang Tiga Dimensi (3D)}}\label{menggambar-data-x-y-z-pada-ruang-tiga-dimensi-3d}

Menggambar data pada ruang tiga dimensi (3D) adalah proses

visualisasi data yang mengubah informasi dalam tiga dimensi, yaitu panjang, lebar, dan tinggi, menjadi representasi visual yang dapat dipahami dan dianalisis.

Tujuan dari menggambar data 3D adalah untuk membantu pemahaman dan

interpretasi data yang lebih baik, terutama ketika data tersebut memiliki komponen yang tidak dapat direpresentasikan dengan baik dalam dua dimensi.

Sama seperti plot2d, plot3d menerima data. Untuk objek 3D, kita perlu menyediakan matriks nilai x-, y- dan z, atau tiga fungsi atau ekspresi fx(x,y), fy(x,y), fz(x,y).

\[\gamma(t,s) = (x(t,s),y(t,s),z(t,s))\]Karena x,y,z adalah matriks, kita asumsikan bahwa (t,s) melalui sebuah kotak persegi. Hasilnya, kita dapat memplot gambar persegi panjang di ruang angkasa.

Kita dapat menggunakan bahasa matriks Euler untuk menghasilkan koordinat secara efektif.

Dalam contoh berikut, kami menggunakan vektor nilai t dan vektor kolom nilai s untuk membuat parameter permukaan bola. Dalam gambar kita dapat menandai daerah, dalam kasus kita daerah kutub.

\textbf{Contoh 1 grafik fungsi}

\textgreater t=-1:0.1:1; s=(-1:0.1:1)'; plot3d(t,s,t*s):

\begin{figure}
\centering
\pandocbounded{\includegraphics[keepaspectratio]{images/EMT_Grafik3D_Alfi Nur Azumah-153.png}}
\caption{images/EMT\_Grafik3D\_Alfi\%20Nur\%20Azumah-153.png}
\end{figure}

Penjelasan sintaks dari plot

\begin{itemize}
\item
  plot3d = membawa euler untuk mengetahui perintah apa yang harus dilakukan
\item
  ('' \ldots'') = tempat kita untuk memasukkan perintah yang kita inginkan
\end{itemize}

\textbf{Contoh 2}

kita akan memebentuk plot dengan fungsi dibawah ini

\[x^2 + y^2\]\textgreater plot3d(``x\textsuperscript{2+y}2''):

\begin{figure}
\centering
\pandocbounded{\includegraphics[keepaspectratio]{images/EMT_Grafik3D_Alfi Nur Azumah-155.png}}
\caption{images/EMT\_Grafik3D\_Alfi\%20Nur\%20Azumah-155.png}
\end{figure}

Selanjutnya kita akan menggambar garis pada plot dengan menggunakan grid

\textgreater plot3d(``x\textsuperscript{2+y}2'',grid=2):

\begin{figure}
\centering
\pandocbounded{\includegraphics[keepaspectratio]{images/EMT_Grafik3D_Alfi Nur Azumah-156.png}}
\caption{images/EMT\_Grafik3D\_Alfi\%20Nur\%20Azumah-156.png}
\end{figure}

Jika kita ingin memodifikasi plot dengan menambahkan warna pada plot, bisa menggunakan fillcolor.

Fillcolor dapat diisi dengan 1 warna yang sama atau 2 warna yang berbeda

\textgreater plot3d(``x\textsuperscript{2+y}2'',grid= 2,fillcolor={[}blue,blue{]}):

\begin{figure}
\centering
\pandocbounded{\includegraphics[keepaspectratio]{images/EMT_Grafik3D_Alfi Nur Azumah-157.png}}
\caption{images/EMT\_Grafik3D\_Alfi\%20Nur\%20Azumah-157.png}
\end{figure}

\textbf{Contoh 3}

Jika kita ingin membuat plot 3d pada fungsi

\[2x^2+y^3\]kita bisa menggunakan perintah seperti dibawah ini

\textgreater plot3d(``2x\textsuperscript{2+y}3'',grid=10,\textgreater hue, color=red);

\textgreater insimg()

\begin{figure}
\centering
\pandocbounded{\includegraphics[keepaspectratio]{images/EMT_Grafik3D_Alfi Nur Azumah-159.png}}
\caption{images/EMT\_Grafik3D\_Alfi\%20Nur\%20Azumah-159.png}
\end{figure}

Jika kita mau menebalkan warna pada gambar diatas makam bisa menggunakan perintah

\textgreater plot3d(``2x\textsuperscript{2+y}3'',grid=10,fillcolor={[}red,red{]});

\textgreater insimg()

\begin{figure}
\centering
\pandocbounded{\includegraphics[keepaspectratio]{images/EMT_Grafik3D_Alfi Nur Azumah-160.png}}
\caption{images/EMT\_Grafik3D\_Alfi\%20Nur\%20Azumah-160.png}
\end{figure}

\textbf{Contoh 4}

Tentu saja, titik cloud juga dimungkinkan. Untuk memplot data titik dalam ruang, kita membutuhkan tiga vektor untuk koordinat titik-titik tersebut.

Gayanya sama seperti di plot2d dengan points=true;

\textgreater n=500;\ldots{}\\
\textgreater{} plot3d(normal(1,n),normal(1,n),normal(1,n),points=true,style=``.''):

\begin{figure}
\centering
\pandocbounded{\includegraphics[keepaspectratio]{images/EMT_Grafik3D_Alfi Nur Azumah-161.png}}
\caption{images/EMT\_Grafik3D\_Alfi\%20Nur\%20Azumah-161.png}
\end{figure}

\textbf{Contoh Soal 5}

Dalam contoh berikut, kita membuat tampilan bayangan dari bola yang

terdistorsi. Koordinat biasa untuk bola adalah

\[\gamma(t,s) = (\cos(t)\cos(s),\sin(t)\sin(s),\cos(s))\]dengan

\[0 \le t \le 2\pi, \quad \frac{-\pi}{2} \le s \le \frac{\pi}{2}.\]Kami mendistorsi ini dengan sebuah faktor

\[d(t,s) = \frac{\cos(4t)+\cos(8s)}{4}\]\textgreater t=linspace(0,2pi,320); s=linspace(-pi/2,pi/2,160)';\ldots{}\\
\textgreater{} d=1+0.2*(cos(4*t)+cos(8*s));\ldots{}\\
\textgreater{} plot3d(cos(t)*cos(s)*d,sin(t)*cos(s)*d,sin(s)*d,hue=1,\ldots{}\\
\textgreater{} light={[}1,0,1{]},frame=0,zoom=5):

\begin{figure}
\centering
\pandocbounded{\includegraphics[keepaspectratio]{images/EMT_Grafik3D_Alfi Nur Azumah-165.png}}
\caption{images/EMT\_Grafik3D\_Alfi\%20Nur\%20Azumah-165.png}
\end{figure}

\subsection{4.6 Membuat Gambar Grafik Tiga Dimensi (3D) yang Bersifat Interaktif dan animasi grafik 3D}\label{membuat-gambar-grafik-tiga-dimensi-3d-yang-bersifat-interaktif-dan-animasi-grafik-3d}

Membuat gambar grafik tiga dimensi (3D) yang bersifat interaktif adalah proses menciptakan visualisasi tiga dimensi yang memungkinkan pengguna berinteraksi dengan objek-objek 3D. Interaktivitas dalam gambar 3D memungkinkan pengguna untuk melakukan tindakan seperti mengubah sudut pandang, memindahkan objek, atau berinteraksi dengan elemen-elemen dalam adegan 3D. Sedangkan animasi grafik 3D dapat mencakup pergerakan, tetapi juga dapat berarti perubahan dalam tampilan atau atribut objek tanpa pergerakan fisik yang mencolok.

Interaksi user dimungkinkan dengan parameter \textgreater user. dengan perintah \textgreater user kita dapat menekan tombol berikut.

\begin{itemize}
\item
  kiri, kanan, atas, bawah: memutar sudut pandang
\item
  +,-: memperbesar atau memperkecil
\item
  a: menghasilkan anaglyph (lihat di bawah)
\item
  l : tombol nyalakan sumber cahaya (lihat dibawah)
\item
  spasi: reset ke default
\item
  kembali: akhiri interaksi
\end{itemize}

\textgreater plot3d(``exp(-x\textsuperscript{2+y}2)'',\textgreater user,\ldots{}\\
\textgreater{} title=``Coba gerakkan''):

\begin{figure}
\centering
\pandocbounded{\includegraphics[keepaspectratio]{images/EMT_Grafik3D_Alfi Nur Azumah-166.png}}
\caption{images/EMT\_Grafik3D\_Alfi\%20Nur\%20Azumah-166.png}
\end{figure}

\textgreater plot3d(``exp(x\textsuperscript{2+y}2)'',\textgreater user,\ldots{}\\
\textgreater{} title=``Coba gerakkan)''):

\begin{figure}
\centering
\pandocbounded{\includegraphics[keepaspectratio]{images/EMT_Grafik3D_Alfi Nur Azumah-167.png}}
\caption{images/EMT\_Grafik3D\_Alfi\%20Nur\%20Azumah-167.png}
\end{figure}

\textbf{Animasi 3D}

\textgreater function testplot () := plot3d(``x\textsuperscript{2+y}3'');\ldots{}\\
\textgreater{} rotate(``testplot''); testplot():

\begin{figure}
\centering
\pandocbounded{\includegraphics[keepaspectratio]{images/EMT_Grafik3D_Alfi Nur Azumah-168.png}}
\caption{images/EMT\_Grafik3D\_Alfi\%20Nur\%20Azumah-168.png}
\end{figure}

Fungsi rotate yaitu untuk memutar plot.

Fungsi ini akan membuat animasi plot 3D dari fungsi

\[x^2 + y^3\]

yang berputar di sekitar sumbu z dari sudut 0 hingga 360 derajat

\textgreater plot3d(``exp(-(x\textsuperscript{2+y}2)/5)'',r=10,n=80,fscale=8,scale=1.2,frame=3,\textgreater user):

\begin{figure}
\centering
\pandocbounded{\includegraphics[keepaspectratio]{images/EMT_Grafik3D_Alfi Nur Azumah-170.png}}
\caption{images/EMT\_Grafik3D\_Alfi\%20Nur\%20Azumah-170.png}
\end{figure}

Ada beberapa parameter untuk menskalakan fungsi atau mengubah tampilan grafik.

fscale: menskalakan ke nilai fungsi (defaultnya adalah \textless fscale).

scale: angka atau vektor 1x2 untuk diskalakan ke arah x dan y.

frame: jenis bingkai (default 1).

\textgreater plot3d(``x\^{}2+y'',distance=10,zoom=5,angle=0,height=5):

\begin{figure}
\centering
\pandocbounded{\includegraphics[keepaspectratio]{images/EMT_Grafik3D_Alfi Nur Azumah-171.png}}
\caption{images/EMT\_Grafik3D\_Alfi\%20Nur\%20Azumah-171.png}
\end{figure}

Tampilan dapat diubah dengan berbagai cara.

\begin{itemize}
\tightlist
\item
  distance: jarak pandang ke plot.
\item
  zoom: nilai zoom.
\item
  angle: sudut terhadap sumbu y negatif dalam radian.
\item
  height: ketinggian tampilan dalam radian.
\end{itemize}

\textgreater plot3d(``x\textsuperscript{4+y}2'',a=0,b=1,c=-1,d=1,angle=-20°,height=20°,\ldots{}\\
\textgreater{} center={[}0,0,1{]},zoom=3):

\begin{figure}
\centering
\pandocbounded{\includegraphics[keepaspectratio]{images/EMT_Grafik3D_Alfi Nur Azumah-172.png}}
\caption{images/EMT\_Grafik3D\_Alfi\%20Nur\%20Azumah-172.png}
\end{figure}

Plot selalu terlihat berada di tengah kubus plot. Kita dapat memindahkan bagian tengah dengan center parameter.

\textgreater plot3d(``x\^{}2+1'',a=-1,b=1,rotate=true,grid=5):

\begin{figure}
\centering
\pandocbounded{\includegraphics[keepaspectratio]{images/EMT_Grafik3D_Alfi Nur Azumah-173.png}}
\caption{images/EMT\_Grafik3D\_Alfi\%20Nur\%20Azumah-173.png}
\end{figure}

Parameter memutar memutar fungsi dalam x di sekitar sumbu x.

\begin{itemize}
\tightlist
\item
  rotate=1: Menggunakan sumbu x
\item
  rotate=2: Menggunakan sumbu z
\end{itemize}

\subsection{4.7 Menggambar Fungsi Parametrik Tiga Dimensi (3D)}\label{menggambar-fungsi-parametrik-tiga-dimensi-3d}

\textbf{Pengertian}

Fungsi parametrik adalah jenis fungsi matematika yang menggambarkan hubungan antara dua atau lebih variabel, di mana setiap variabel dinyatakan sebagai fungsi dari satu atau lebih parameter. Fungsi parametrik digunakan untuk menggambarkan kurva, lintasan, atau hubungan antara berbagai variabel yang bergantung pada parameter-parameter tertentu.

Fungsi parametrik merupakan salah satu cara mendefinisikan kurva atau permukaan dalam ruang 2D atau 3D menggunakan satu atau lebih parameter independen.

\begin{itemize}
\tightlist
\item
  Dalam 2D, kurva dinyatakan sebagai x(t) dan y(t), di mana adalah t adalah parameter yang mengontrol posisi sepanjang kurva.
\item
  Dalam 3D, kita menggunakan tiga persamaan parametrik untuk mendeskripsikan posisi x,y,z sebagai fungsi dari parameter t.Fungsi ini ditulis sebagai: x=f(t), y=g(t), z=h(t),
\end{itemize}

\textbf{Contoh Soal}

\textgreater plot3d(``cos(x)*cos(y)'',``sin(x)*cos(y)'',``sin(y)'', a=0,b=2*pi,c=pi/2,d=-pi/2,\ldots{}\\
\textgreater{} \textgreater hue,color=red,light={[}0,1,0{]},\textless frame,\ldots{}\\
\textgreater{} n=90,grid={[}25,50{]},wirecolor=black,zoom=4):

\begin{figure}
\centering
\pandocbounded{\includegraphics[keepaspectratio]{images/EMT_Grafik3D_Alfi Nur Azumah-174.png}}
\caption{images/EMT\_Grafik3D\_Alfi\%20Nur\%20Azumah-174.png}
\end{figure}

\textgreater aspect(16/9); allwindow; \ldots{}\\
\textgreater{} x:=linspace(0,2*pi,100); y:=(-1:0.1:1)'; \ldots{}\\
\textgreater{} plot3d(sin(x)*(y/2*sin(x/2)),cos(x)*(y/2*sin(x/2)),y/2*cos(x/2), \ldots{}\\
\textgreater{} \textless frame,hue=2,max=0.5,scale=1.5):

\begin{figure}
\centering
\pandocbounded{\includegraphics[keepaspectratio]{images/EMT_Grafik3D_Alfi Nur Azumah-175.png}}
\caption{images/EMT\_Grafik3D\_Alfi\%20Nur\%20Azumah-175.png}
\end{figure}

\textgreater aspect(16/9); allwindow;\ldots{}\\
\textgreater{} x:=linspace(0,2*pi,100); y:=(-1:0.1:1)';\ldots{}\\
\textgreater{} plot3d(sin(x)*(y/2*sin(x/2)),cos(x)*(y/2*sin(x/2)),y/2*cos(x/2),\ldots{}\\
\textgreater{} \textgreater lines,\textless frame,xmin=0,xmax=10,n=10,\textgreater user):

\begin{figure}
\centering
\pandocbounded{\includegraphics[keepaspectratio]{images/EMT_Grafik3D_Alfi Nur Azumah-176.png}}
\caption{images/EMT\_Grafik3D\_Alfi\%20Nur\%20Azumah-176.png}
\end{figure}

\textgreater reset;\ldots{}\\
\textgreater{} S:=normal(10,1250); plot3d(S{[}3{]},S{[}6{]},S{[}9{]},\textgreater points,style=``.''):

\begin{figure}
\centering
\pandocbounded{\includegraphics[keepaspectratio]{images/EMT_Grafik3D_Alfi Nur Azumah-177.png}}
\caption{images/EMT\_Grafik3D\_Alfi\%20Nur\%20Azumah-177.png}
\end{figure}

\textgreater S:=normal(10,1250); T:=cumsum(normal(10,1250));\ldots{}\\
\textgreater{} plot3d(T{[}2{]},T{[}5{]},T{[}8{]},\textgreater wire,\ldots{}\\
\textgreater{} linewidth=2,\textgreater anaglyph,zoom=3):

\begin{figure}
\centering
\pandocbounded{\includegraphics[keepaspectratio]{images/EMT_Grafik3D_Alfi Nur Azumah-178.png}}
\caption{images/EMT\_Grafik3D\_Alfi\%20Nur\%20Azumah-178.png}
\end{figure}

\textgreater P=cumsum(normal(5,75));\ldots{}\\
\textgreater{} plot3d(P{[}3{]},P{[}4{]},P{[}5{]},\textgreater anaglyph,\textgreater wire):

\begin{figure}
\centering
\pandocbounded{\includegraphics[keepaspectratio]{images/EMT_Grafik3D_Alfi Nur Azumah-179.png}}
\caption{images/EMT\_Grafik3D\_Alfi\%20Nur\%20Azumah-179.png}
\end{figure}

\subsection{4.8 Menggambar Fungsi Implisit Tiga Dimensi (3D)}\label{menggambar-fungsi-implisit-tiga-dimensi-3d}

Fungsi implisit (implicit function) adalah fungsi yang memuat lebih dari satu variabel, berjenis variabel bebas dan variabel terikat yang berada dalam satu ruas sehingga tidak bisa dipisahkan pada ruas yang berbeda.

\[F(x,y,z)=0\]

(1 persamaan dan 3 variabel), terdiri dari 2 variabel bebas dan 1 terikat

Misalnya, F(x, y, z) = x\^{}2 + y\^{}2 + z\^{}2 - 1 = 0 adalah persamaan implisit yang menggambarkan bola dengan jari-jari 1 dan pusat di (0,0,0).

\textbf{Plot Implisit}

Ada juga plot implisit dalam tiga dimensi. Euler menghasilkan pemotongan melalui objek. Fitur plot3d mencakup plot implisit. Plot ini menunjukkan himpunan nol suatu fungsi dalam tiga variabel. Solusi dari

\[f(x,y,z) = 0\]dapat divisualisasikan dalam potongan yang sejajar dengan bidang x-y, x-z, z-x dan y-z.

\begin{itemize}
\tightlist
\item
  implisit=1: dipotong sejajar bidang y-z
\item
  implisit=2: dipotong sejajar dengan bidang x-z
\item
  implisit=3: dipotong sejajar dengan bidang z-x (yang berarti pemotongan dilakukan dengan mempertahankan nilai y konstan)
\item
  implisit=4: dipotong sejajar~bidang~x-y
\end{itemize}

Tambahkan nilai-nilai ini, jika Anda mau. Dalam contoh kita memplot

\[M = \{ (x,y,z) : x^2+y^3+zy=1 \}\] \textbf{Contoh Fungsi Implisit}

\textgreater plot3d(``x\textsuperscript{2+y}3+z*y-1'',r=7,implicit=4):

\begin{figure}
\centering
\pandocbounded{\includegraphics[keepaspectratio]{images/EMT_Grafik3D_Alfi Nur Azumah-183.png}}
\caption{images/EMT\_Grafik3D\_Alfi\%20Nur\%20Azumah-183.png}
\end{figure}

\textgreater plot3d(``2*x\^{}2 + 3*y\^{}2 + z\^{}2 - 25'',r=8,implicit=2):

\begin{figure}
\centering
\pandocbounded{\includegraphics[keepaspectratio]{images/EMT_Grafik3D_Alfi Nur Azumah-184.png}}
\caption{images/EMT\_Grafik3D\_Alfi\%20Nur\%20Azumah-184.png}
\end{figure}

\textgreater plot3d(``4*x\^{}3 + 3*y\^{}4 + 6*z\^{}2 - 10'',r=3,implicit=3):

\begin{figure}
\centering
\pandocbounded{\includegraphics[keepaspectratio]{images/EMT_Grafik3D_Alfi Nur Azumah-185.png}}
\caption{images/EMT\_Grafik3D\_Alfi\%20Nur\%20Azumah-185.png}
\end{figure}

\textgreater plot3d(``x\^{}5 + 5*y\^{}3 + 3*z\^{}2 - 5*x - 7*y - 5*z + 10'',r=5,implicit=2):

\begin{figure}
\centering
\pandocbounded{\includegraphics[keepaspectratio]{images/EMT_Grafik3D_Alfi Nur Azumah-186.png}}
\caption{images/EMT\_Grafik3D\_Alfi\%20Nur\%20Azumah-186.png}
\end{figure}

\textgreater plot3d(``x\^{}2 + y\^{}2 - z\^{}2'',r=5,implicit=3):

\begin{figure}
\centering
\pandocbounded{\includegraphics[keepaspectratio]{images/EMT_Grafik3D_Alfi Nur Azumah-187.png}}
\caption{images/EMT\_Grafik3D\_Alfi\%20Nur\%20Azumah-187.png}
\end{figure}

\textgreater plot3d(``x\^{}3 + 2*y\^{}2 +3*z\^{}3-4'',r=5,implicit=3):

\begin{figure}
\centering
\pandocbounded{\includegraphics[keepaspectratio]{images/EMT_Grafik3D_Alfi Nur Azumah-188.png}}
\caption{images/EMT\_Grafik3D\_Alfi\%20Nur\%20Azumah-188.png}
\end{figure}

\textgreater plot3d(``x\textsuperscript{2+y}2+z\^{}2+2*x*y+4*y*z+8*z*x-20'',r=5,implicit=3):

\begin{figure}
\centering
\pandocbounded{\includegraphics[keepaspectratio]{images/EMT_Grafik3D_Alfi Nur Azumah-189.png}}
\caption{images/EMT\_Grafik3D\_Alfi\%20Nur\%20Azumah-189.png}
\end{figure}

\textgreater c=1; d=1;

\textgreater plot3d(``((x\textsuperscript{2+y}2-c\textsuperscript{2)}2+(z\textsuperscript{2-1)}2)*((y\textsuperscript{2+z}2-c\textsuperscript{2)}2+(x\textsuperscript{2-1)}2)*((z\textsuperscript{2+x}2-c\textsuperscript{2)}2+(y\textsuperscript{2-1)}2)-d'',r=2,\textless frame,\textgreater implicit,\textgreater user):

\begin{figure}
\centering
\pandocbounded{\includegraphics[keepaspectratio]{images/EMT_Grafik3D_Alfi Nur Azumah-190.png}}
\caption{images/EMT\_Grafik3D\_Alfi\%20Nur\%20Azumah-190.png}
\end{figure}

\textgreater c=1; d=1;

\textgreater plot3d(``((x\textsuperscript{2+y}2+c\textsuperscript{2)}2+(z\textsuperscript{2-1)}2)*((y\textsuperscript{2+z}2-c\textsuperscript{2)}2+(x\textsuperscript{2-1)}2)*((z\textsuperscript{2+x}2-c\textsuperscript{2)}2+(y\textsuperscript{2-1)}2)-d'',r=2,\textless frame,\textgreater implicit,\textgreater user):

\begin{figure}
\centering
\pandocbounded{\includegraphics[keepaspectratio]{images/EMT_Grafik3D_Alfi Nur Azumah-191.png}}
\caption{images/EMT\_Grafik3D\_Alfi\%20Nur\%20Azumah-191.png}
\end{figure}

\textgreater plot3d(``x\textsuperscript{3+y}5+5*x*z+z\^{}3'',\textgreater implicit,r=3,zoom=2):

\begin{figure}
\centering
\pandocbounded{\includegraphics[keepaspectratio]{images/EMT_Grafik3D_Alfi Nur Azumah-192.png}}
\caption{images/EMT\_Grafik3D\_Alfi\%20Nur\%20Azumah-192.png}
\end{figure}

\textgreater plot3d(``x\textsuperscript{2+y}2+4*x*z+z\^{}3-2'',\textgreater implicit,r=2,zoom=2.5):

\begin{figure}
\centering
\pandocbounded{\includegraphics[keepaspectratio]{images/EMT_Grafik3D_Alfi Nur Azumah-193.png}}
\caption{images/EMT\_Grafik3D\_Alfi\%20Nur\%20Azumah-193.png}
\end{figure}

\textgreater plot3d(``x\textsuperscript{2*y}2+x\textsuperscript{3+y}3*x'',\textgreater implicit,r=5,zoom=2.5):

\begin{figure}
\centering
\pandocbounded{\includegraphics[keepaspectratio]{images/EMT_Grafik3D_Alfi Nur Azumah-194.png}}
\caption{images/EMT\_Grafik3D\_Alfi\%20Nur\%20Azumah-194.png}
\end{figure}

\textgreater plot3d(``x*y-z\^{}2+2*x*y*z-0'',\textgreater implicit,r=5,zoom=2.5):

\begin{figure}
\centering
\pandocbounded{\includegraphics[keepaspectratio]{images/EMT_Grafik3D_Alfi Nur Azumah-195.png}}
\caption{images/EMT\_Grafik3D\_Alfi\%20Nur\%20Azumah-195.png}
\end{figure}

\textbf{Latihan soal}

\begin{enumerate}
\def\labelenumi{\arabic{enumi}.}
\tightlist
\item
  Gambarlah Fungsi implisit berikut dalam 3D
\end{enumerate}

\[f(x,y,z)=x^2+y^2-z^2-1\]\textgreater plot3d(``x\textsuperscript{2+y}2-z\^{}2-1'',r=8,implicit=3):

\begin{figure}
\centering
\pandocbounded{\includegraphics[keepaspectratio]{images/EMT_Grafik3D_Alfi Nur Azumah-197.png}}
\caption{images/EMT\_Grafik3D\_Alfi\%20Nur\%20Azumah-197.png}
\end{figure}

\begin{enumerate}
\def\labelenumi{\arabic{enumi}.}
\setcounter{enumi}{1}
\tightlist
\item
  Gambarlah fungsi 3D dari fungsi implisit berikut ini
\end{enumerate}

\[f(x,y,z)=xy+x^3y^2+xz^3-9=0\]

dengan r=4

\textgreater plot3d(``x*y+x\textsuperscript{3*y}2+x*z\^{}3-9'', r=4, implicit=3):

\begin{figure}
\centering
\pandocbounded{\includegraphics[keepaspectratio]{images/EMT_Grafik3D_Alfi Nur Azumah-199.png}}
\caption{images/EMT\_Grafik3D\_Alfi\%20Nur\%20Azumah-199.png}
\end{figure}

\subsection{4.9 Mengatur tampilan, warna dan sudut pandang gambar permukaan}\label{mengatur-tampilan-warna-dan-sudut-pandang-gambar-permukaan}

\textbf{Tiga Dimensi (3D) Dan Menampilkan kontur dan bidang kontur permukaan Tiga Dimensi(3D)}

Untuk plot, Euler menambahkan garis grid. Sebagai gantinya dimungkinkan untuk menggunakan garis level dan rona satu warna atau rona berwarna spektral. Euler dapat menggambar tinggi fungsi pada plot dengan bayangan. Di semua plot 3D, Euler dapat menghasilkan anaglyph merah/sian.

\begin{itemize}
\item
  hue: Menyalakan bayangan cahaya alih-alih kabel.
\item
  kontur: Memplot garis kontur otomatis pada plot.
\item
  level=\ldots{} (atau level): Sebuah vektor nilai untuk garis kontur.
\end{itemize}

Standarnya adalah level=``auto'', yang menghitung beberapa garis level secara otomatis. Seperti yang Anda lihat di plot, level sebenarnya adalah rentang level.

Gaya default dapat diubah. Untuk plot kontur berikut, kami menggunakan grid yang lebih halus untuk 100x100 poin, skala fungsi dan plot, dan menggunakan sudut pandang yang berbeda.

\textgreater plot3d(``exp(-x\textsuperscript{2-y}2)'',r=2,n=100,level=``thin'', \ldots{}\\
\textgreater{} \textgreater contour,\textgreater spectral,fscale=1,scale=1.1,angle=45°,height=20°):

\begin{figure}
\centering
\pandocbounded{\includegraphics[keepaspectratio]{images/EMT_Grafik3D_Alfi Nur Azumah-200.png}}
\caption{images/EMT\_Grafik3D\_Alfi\%20Nur\%20Azumah-200.png}
\end{figure}

\textgreater plot3d(``exp(x*y)'',angle=100°,\textgreater contour,color=green):

\begin{figure}
\centering
\pandocbounded{\includegraphics[keepaspectratio]{images/EMT_Grafik3D_Alfi Nur Azumah-201.png}}
\caption{images/EMT\_Grafik3D\_Alfi\%20Nur\%20Azumah-201.png}
\end{figure}

Bayangan default menggunakan warna abu-abu. Tetapi rentang warna spektral juga tersedia.

-\textgreater{} spektral: Menggunakan skema spektral default

\begin{itemize}
\tightlist
\item
  color=\ldots: Menggunakan warna khusus atau skema spektral
\end{itemize}

Untuk plot berikut, kami menggunakan skema spektral default dan menambah jumlah titik untuk mendapatkan tampilan yang sangat halus.

\textgreater plot3d(``x\textsuperscript{2+y}2'',\textgreater spectral,\textgreater contour,n=100):

\begin{figure}
\centering
\pandocbounded{\includegraphics[keepaspectratio]{images/EMT_Grafik3D_Alfi Nur Azumah-202.png}}
\caption{images/EMT\_Grafik3D\_Alfi\%20Nur\%20Azumah-202.png}
\end{figure}

Alih-alih garis level otomatis, kita juga dapat mengatur nilai garis level. Ini akan menghasilkan garis level tipis alih-alih rentang level.

\textgreater plot3d(``x\textsuperscript{2-y}2'',0,1,0,1,angle=220°,level=-1:0.2:1,color=redgreen):

\begin{figure}
\centering
\pandocbounded{\includegraphics[keepaspectratio]{images/EMT_Grafik3D_Alfi Nur Azumah-203.png}}
\caption{images/EMT\_Grafik3D\_Alfi\%20Nur\%20Azumah-203.png}
\end{figure}

Dalam plot berikut, kami menggunakan dua pita level yang sangat luas dari -0,1 hingga 1, dan dari 0,9 hingga 1. Ini dimasukkan sebagai matriks dengan batas level sebagai kolom.

Selain itu, kami melapisi kisi dengan 10 interval di setiap arah.

\textgreater plot3d(``x\textsuperscript{2+y}3'',level={[}-0.1,0.9;0,1{]}, \ldots{}\\
\textgreater{} \textgreater spectral,angle=30°,grid=10,contourcolor=gray):

\begin{figure}
\centering
\pandocbounded{\includegraphics[keepaspectratio]{images/EMT_Grafik3D_Alfi Nur Azumah-204.png}}
\caption{images/EMT\_Grafik3D\_Alfi\%20Nur\%20Azumah-204.png}
\end{figure}

Dalam contoh berikut, kami memplot himpunan, di mana

\[f(x,y) = x^y-y^x = 0\]Kami menggunakan satu garis tipis untuk garis level.

\textgreater plot3d(``x\textsuperscript{y-y}x'',level=0,a=0,b=6,c=0,d=6,contourcolor=red,n=100):

\begin{figure}
\centering
\pandocbounded{\includegraphics[keepaspectratio]{images/EMT_Grafik3D_Alfi Nur Azumah-206.png}}
\caption{images/EMT\_Grafik3D\_Alfi\%20Nur\%20Azumah-206.png}
\end{figure}

Dimungkinkan untuk menunjukkan bidang kontur di bawah plot. Warna dan jarak ke plot dapat ditentukan.

\textgreater plot3d(``x\textsuperscript{2+y}4'',\textgreater cp,cpcolor=green,cpdelta=0.2):

\begin{figure}
\centering
\pandocbounded{\includegraphics[keepaspectratio]{images/EMT_Grafik3D_Alfi Nur Azumah-207.png}}
\caption{images/EMT\_Grafik3D\_Alfi\%20Nur\%20Azumah-207.png}
\end{figure}

Berikut adalah beberapa gaya lagi. Kami selalu mematikan frame, dan menggunakan berbagai skema warna untuk plot dan grid.

\textgreater figure(2,2); \ldots{}\\
\textgreater{} expr=``y\textsuperscript{3-x}2''; \ldots{}\\
\textgreater{} figure(1); \ldots{}\\
\textgreater{} plot3d(expr,\textless frame,\textgreater cp,cpcolor=spectral); \ldots{}\\
\textgreater{} figure(2); \ldots{}\\
\textgreater{} plot3d(expr,\textless frame,\textgreater spectral,grid=10,cp=2); \ldots{}\\
\textgreater{} figure(3); \ldots{}\\
\textgreater{} plot3d(expr,\textless frame,\textgreater contour,color=gray,nc=5,cp=3,cpcolor=greenred); \ldots{}\\
\textgreater{} figure(4); \ldots{}\\
\textgreater{} plot3d(expr,\textless frame,\textgreater hue,grid=10,\textgreater transparent,\textgreater cp,cpcolor=gray); \ldots{}\\
\textgreater{} figure(0):

\begin{figure}
\centering
\pandocbounded{\includegraphics[keepaspectratio]{images/EMT_Grafik3D_Alfi Nur Azumah-208.png}}
\caption{images/EMT\_Grafik3D\_Alfi\%20Nur\%20Azumah-208.png}
\end{figure}

Ada beberapa skema spektral lainnya, bernomor dari 1 hingga 9. Tetapi Anda juga dapat menggunakan warna=nilai, di mana nilai

\begin{itemize}
\tightlist
\item
  spektral: untuk rentang dari biru ke merah
\item
  putih: untuk rentang yang lebih redup
\item
  kuningbiru,ungu hijau,birukuning,hijaumerah
\item
  birukuning, hijau ungu, kuning biru, merah hijau
\end{itemize}

\textgreater figure(3,3); \ldots{}\\
\textgreater{} for i=1:9; \ldots{}\\
\textgreater{} figure(i); plot3d(``x\textsuperscript{2+y}2'',spectral=i,\textgreater contour,\textgreater cp,\textless frame,zoom=4); \ldots{}\\
\textgreater{} end; \ldots{}\\
\textgreater{} figure(0):

\begin{figure}
\centering
\pandocbounded{\includegraphics[keepaspectratio]{images/EMT_Grafik3D_Alfi Nur Azumah-209.png}}
\caption{images/EMT\_Grafik3D\_Alfi\%20Nur\%20Azumah-209.png}
\end{figure}

Sumber cahaya dapat diubah dengan l dan tombol kursor selama interaksi pengguna. Itu juga dapat diatur dengan parameter.

\begin{itemize}
\tightlist
\item
  cahaya: arah untuk cahaya
\item
  amb: cahaya sekitar antara 0 dan 1
\end{itemize}

Perhatikan bahwa program tidak membuat perbedaan antara sisi plot. Tidak ada bayangan. Untuk ini, Anda perlu Povray.

\textgreater plot3d(``-x\textsuperscript{2-y}2'', \ldots{}\\
\textgreater{} hue=true,light={[}0,1,1{]},amb=0,user=true, \ldots{}\\
\textgreater{} title=``Press l and cursor keys (return to exit)''):

\begin{figure}
\centering
\pandocbounded{\includegraphics[keepaspectratio]{images/EMT_Grafik3D_Alfi Nur Azumah-210.png}}
\caption{images/EMT\_Grafik3D\_Alfi\%20Nur\%20Azumah-210.png}
\end{figure}

Parameter warna mengubah warna permukaan. Warna garis level juga dapat diubah.

\textgreater plot3d(``-x\textsuperscript{2-y}2'',color=rgb(0.2,0.2,0),hue=true,frame=false, \ldots{}\\
\textgreater{} zoom=3,contourcolor=red,level=-2:0.1:1,dl=0.01):

\begin{figure}
\centering
\pandocbounded{\includegraphics[keepaspectratio]{images/EMT_Grafik3D_Alfi Nur Azumah-211.png}}
\caption{images/EMT\_Grafik3D\_Alfi\%20Nur\%20Azumah-211.png}
\end{figure}

Warna 0 memberikan efek pelangi khusus.

\textgreater plot3d(``x\textsuperscript{2/(x}2+y\^{}2+1)'',color=0,hue=true,grid=10):

\begin{figure}
\centering
\pandocbounded{\includegraphics[keepaspectratio]{images/EMT_Grafik3D_Alfi Nur Azumah-212.png}}
\caption{images/EMT\_Grafik3D\_Alfi\%20Nur\%20Azumah-212.png}
\end{figure}

Permukaannya juga bisa transparan.

\textgreater plot3d(``x\textsuperscript{2+y}2'',\textgreater transparent,grid=10,wirecolor=red):

\begin{figure}
\centering
\pandocbounded{\includegraphics[keepaspectratio]{images/EMT_Grafik3D_Alfi Nur Azumah-213.png}}
\caption{images/EMT\_Grafik3D\_Alfi\%20Nur\%20Azumah-213.png}
\end{figure}

\subsection{4.10 Menggambar Diagram Batang Tiga Dimensi}\label{menggambar-diagram-batang-tiga-dimensi}

Bar plots/plot batang juga dimungkinkan. Untuk itu, kita harus menyediakannya

\begin{itemize}
\tightlist
\item
  x: vektor baris dengan n+1 elemen
\item
  y: vektor kolom dengan n+1 elemen
\item
  z: matriks nilai nxn.
\end{itemize}

z bisa lebih besar, tetapi hanya nilai nxn yang akan digunakan.

Dalam contoh ini, pertama-tama kita menghitung nilainya. Kemudian kita sesuaikan x dan y, sehingga vektor-vektornya berpusat pada nilai yang digunakan.

\textgreater x=-1:0.1:1; y=x'; z=x\textsuperscript{2+y}2; \ldots{}\\
\textgreater{} xa=(x\textbar1.1)-0.05; ya=(y\_1.1)-0.05; \ldots{}\\
\textgreater{} plot3d(xa,ya,z,bar=true):

\begin{figure}
\centering
\pandocbounded{\includegraphics[keepaspectratio]{images/EMT_Grafik3D_Alfi Nur Azumah-214.png}}
\caption{images/EMT\_Grafik3D\_Alfi\%20Nur\%20Azumah-214.png}
\end{figure}

\textgreater x=-0.01:0.1:1; y=x'; z=x+2/3*y; \ldots{}\\
\textgreater{} xa=(x\textbar1.1)-0.05; ya=(y\_1.1)-0.05; \ldots{}\\
\textgreater{} plot3d(xa,ya,z,bar=true):

\begin{figure}
\centering
\pandocbounded{\includegraphics[keepaspectratio]{images/EMT_Grafik3D_Alfi Nur Azumah-215.png}}
\caption{images/EMT\_Grafik3D\_Alfi\%20Nur\%20Azumah-215.png}
\end{figure}

\textgreater x=-0.01:0.1:1; y=x'; z=1/2*x+1/2*y; \ldots{}\\
\textgreater{} xa=(x\textbar1.1); ya=(y\_1.1); \ldots{}\\
\textgreater{} plot3d(xa,ya,z,bar=true):

\begin{figure}
\centering
\pandocbounded{\includegraphics[keepaspectratio]{images/EMT_Grafik3D_Alfi Nur Azumah-216.png}}
\caption{images/EMT\_Grafik3D\_Alfi\%20Nur\%20Azumah-216.png}
\end{figure}

Dimungkinkan untuk membagi plot suatu permukaan menjadi dua bagian atau lebih.

\textgreater x=-1:0.1:1; y=x'; z=1/10*x+1/10*y; d=zeros(size(x)); \ldots{}\\
\textgreater{} plot3d(x,y,z,disconnect=2:2:5):

\begin{figure}
\centering
\pandocbounded{\includegraphics[keepaspectratio]{images/EMT_Grafik3D_Alfi Nur Azumah-217.png}}
\caption{images/EMT\_Grafik3D\_Alfi\%20Nur\%20Azumah-217.png}
\end{figure}

\textgreater x=-1:0.1:1; y=x'; z=x+y; d=zeros(size(x)); \ldots{}\\
\textgreater{} plot3d(x,y,z,disconnect=2:2:20):

\begin{figure}
\centering
\pandocbounded{\includegraphics[keepaspectratio]{images/EMT_Grafik3D_Alfi Nur Azumah-218.png}}
\caption{images/EMT\_Grafik3D\_Alfi\%20Nur\%20Azumah-218.png}
\end{figure}

Jika memuat atau menghasilkan matriks data M dari file dan perlu memplotnya dalam 3D, Anda dapat menskalakan matriks ke {[}-1,1{]} dengan skala(M), atau menskalakan matriks dengan \textgreater zscale. Hal ini dapat dikombinasikan dengan faktor penskalaan individual yang diterapkan sebagai tambahan.

\textgreater i=1:20; j=i'; \ldots{}\\
\textgreater{} plot3d(i*j\^{}2+100*normal(20,20),\textgreater zscale,scale={[}1,1,1.5{]},angle=-40°,zoom=1.8):

\begin{figure}
\centering
\pandocbounded{\includegraphics[keepaspectratio]{images/EMT_Grafik3D_Alfi Nur Azumah-219.png}}
\caption{images/EMT\_Grafik3D\_Alfi\%20Nur\%20Azumah-219.png}
\end{figure}

\textgreater Z=intrandom(5,100,6); v=zeros(5,6); \ldots{}\\
\textgreater{} loop 1 to 5; v{[}\#{]}=getmultiplicities(1:6,Z{[}\#{]}); end; \ldots{}\\
\textgreater{} columnsplot3d(v',scols=1:5,ccols={[}1:5{]}):

\begin{figure}
\centering
\pandocbounded{\includegraphics[keepaspectratio]{images/EMT_Grafik3D_Alfi Nur Azumah-220.png}}
\caption{images/EMT\_Grafik3D\_Alfi\%20Nur\%20Azumah-220.png}
\end{figure}

\textgreater Z=intrandom(6,100,6); v=zeros(6,2); \ldots{}\\
\textgreater{} loop 1 to 6; v{[}\#{]}=getmultiplicities(1:2,Z{[}\#{]}); end; \ldots{}\\
\textgreater{} columnsplot3d(v',scols=1:6,ccols={[}1:6{]}):

\begin{figure}
\centering
\pandocbounded{\includegraphics[keepaspectratio]{images/EMT_Grafik3D_Alfi Nur Azumah-221.png}}
\caption{images/EMT\_Grafik3D\_Alfi\%20Nur\%20Azumah-221.png}
\end{figure}

\textgreater Z=intrandom(7,1000,6); v=zeros(7,1); \ldots{}\\
\textgreater{} loop 1 to 7; v{[}\#{]}=getmultiplicities(1:1,Z{[}\#{]}); end; \ldots{}\\
\textgreater{} columnsplot3d(v',scols=1:7,ccols={[}1:7{]}):

\begin{figure}
\centering
\pandocbounded{\includegraphics[keepaspectratio]{images/EMT_Grafik3D_Alfi Nur Azumah-222.png}}
\caption{images/EMT\_Grafik3D\_Alfi\%20Nur\%20Azumah-222.png}
\end{figure}

\subsection{4.11 Menggambar Permukaan Benda Putar}\label{menggambar-permukaan-benda-putar}

\textgreater plot2d(``(x\textsuperscript{2+y}2-1)\textsuperscript{3-x}2*y\^{}3'',r=1.3, \ldots{}\\
\textgreater{} style=``\#'',color=red,\textless outline, \ldots{}\\
\textgreater{} level={[}-2;0{]},n=100):

\begin{figure}
\centering
\pandocbounded{\includegraphics[keepaspectratio]{images/EMT_Grafik3D_Alfi Nur Azumah-223.png}}
\caption{images/EMT\_Grafik3D\_Alfi\%20Nur\%20Azumah-223.png}
\end{figure}

\textgreater ekspresi \&= (x\textsuperscript{2+y}2-1)\textsuperscript{3-x}2*y\^{}3; \$ekspresi

\[\left(y^2+x^2-1\right)^3-x^2\,y^3\]Kami ingin memutar kurva jantung di sekitar sumbu y. Berikut adalah ungkapan, yang mendefinisikan hati:

\[f(x,y)=(x^2+y^2-1)^3-x^2.y^3.\]Selanjutnya kita atur

\[x=r.cos(a),\quad y=r.sin(a).\]\textgreater function fr(r,a) \&= ekspresi with {[}x=r*cos(a),y=r*sin(a){]} \textbar{} trigreduce; \$fr(r,a)

\[\left(r^2-1\right)^3+\frac{\left(\sin \left(5\,a\right)-\sin \left(3\,a\right)-2\,\sin a\right)\,r^5}{16}\]Hal ini memungkinkan untuk mendefinisikan fungsi numerik, yang memecahkan r, jika a diberikan. Dengan fungsi itu kita dapat memplot jantung yang diputar sebagai permukaan parametrik.

\textgreater function map f(a) := bisect(``fr'',0,2;a); \ldots{}\\
\textgreater{} t=linspace(-pi/2,pi/2,100); r=f(t); \ldots{}\\
\textgreater{} s=linspace(pi,2pi,100)'; \ldots{}\\
\textgreater{} plot3d(r*cos(t)*sin(s),r*cos(t)*cos(s),r*sin(t), \ldots{}\\
\textgreater{} \textgreater hue,\textless frame,color=red,zoom=4,amb=0,max=0.7,grid=12,height=50°):

\begin{figure}
\centering
\pandocbounded{\includegraphics[keepaspectratio]{images/EMT_Grafik3D_Alfi Nur Azumah-228.png}}
\caption{images/EMT\_Grafik3D\_Alfi\%20Nur\%20Azumah-228.png}
\end{figure}

Berikut ini adalah plot 3D dari gambar di atas yang diputar di sekitar sumbu z. Kami mendefinisikan fungsi, yang menggambarkan objek.

\textgreater function f(x,y,z) \ldots{}

\begin{verbatim}
r=x^2+y^2;
return (r+z^2-1)^3-r*z^3;
 endfunction
\end{verbatim}

\textgreater plot3d(``f(x,y,z)'', \ldots{}\\
\textgreater{} xmin=0,xmax=1.2,ymin=-1.2,ymax=1.2,zmin=-1.2,zmax=1.4, \ldots{}\\
\textgreater{} implicit=1,angle=-30°,zoom=2.5,n={[}10,60,60{]},\textgreater anaglyph):

\begin{figure}
\centering
\pandocbounded{\includegraphics[keepaspectratio]{images/EMT_Grafik3D_Alfi Nur Azumah-229.png}}
\caption{images/EMT\_Grafik3D\_Alfi\%20Nur\%20Azumah-229.png}
\end{figure}

\subsection{4.12 Menggambar Grafik 3D dengan Povray di EMT}\label{menggambar-grafik-3d-dengan-povray-di-emt}

Menggambar Povray Dengan bantuan file Euler povray.e, Euler dapat menghasilkan file Povray. Hasilnya sangat bagus untuk dilihat.

Untuk dapat menjalankan sintaks dalam povray perlu menginstal Povray (32bit atau 64bit) dari http://www.povray.org/, dan meletakkan sub-direktori ``bin'' dari Povray ke pathway, atau mengatur variabel ``defaultpovray'' dengan path lengkap yang menunjuk ke ``pvengine.exe''.

Interface Povray dari Euler menghasilkan file Povray di direktori home pengguna, dan memanggil Povray untuk mengurai file-file ini. Nama file default adalah current.pov, dan direktori default adalah eulerhome(), biasanya c:\Users\Username\Euler. Povray menghasilkan file PNG, yang dapat dimuat oleh Euler ke dalam buku catatan. Untuk membersihkan file-file ini, gunakan povclear().

Sintaks yang digunakan untuk menjalankan povray adalah pov3d. Fungsi pov3d memiliki komponen yang sama dengan plot3d. Ini dapat menghasilkan grafik fungsi f(x,y), atau permukaan dengan koordinat X,Y,Z dalam matriks, termasuk garis level opsional. Fungsi ini memulai raytracer secara otomatis, dan memuat gambar ke dalam notebook Euler.

Selain pov3d(), ada banyak fungsi yang menghasilkan objek Povray. Fungsi-fungsi ini mengembalikan string, yang berisi kode Povray untuk objek. Untuk menggunakan fungsi ini, mulai file Povray dengan povstart(). Kemudian gunakan writeln(\ldots) untuk menulis objek ke file gambar. Terakhir, akhiri file dengan povend(). Secara default, raytracer akan dimulai, dan PNG akan dimasukkan ke dalam notebook Euler.

Fungsi objek memiliki parameter yang disebut ``look'', yang membutuhkan string dengan kode Povray untuk tekstur dan hasil akhir objek. Fungsi povlook() dapat digunakan untuk menghasilkan string ini. Ini memiliki parameter untuk warna, transparansi, Phong Shading dll.

Lingkup Povray memiliki sistem koordinat lain. Interface ini menerjemahkan semua koordinat ke sistem Povray. Jadi Anda dapat terus berpikir dalam sistem koordinat Euler dengan z menunjuk vertikal ke atas, dan x,y,z sumbu dalam arti tangan kanan.

Anda perlu memuat file povray.

\textgreater load povray;

Pastikan, direktori bin Povray ada di path. Jika tidak, edit variabel berikut sehingga berisi path ke povray yang dapat dieksekusi.

\textgreater defaultpovray=``C:\textbackslash Program Files\textbackslash POV-Ray\textbackslash v3.7\textbackslash bin\textbackslash pvengine.exe''

\begin{verbatim}
C:\Program Files\POV-Ray\v3.7\bin\pvengine.exe
\end{verbatim}

\textbf{Contoh Penggunaan}

Akan diberikan contoh sederhana penggunaan povray pada EMT

Perintah berikut menghasilkan file povray di direktori pengguna dan menjalankan Povray untuk ray tracing file ini.

Jika memulai perintah berikut, GUI Povray akan terbuka, menjalankan file, dan menutup secara otomatis. Karena alasan keamanan, akan ditanya, apakah ingin mengizinkan file exe untuk dijalankan. Agar pertanyaan tersebut tidak muncul lagi bisa dipilih batal.

\textgreater pov3d(``x\textsuperscript{2+y}2'',zoom=4);

\begin{figure}
\centering
\pandocbounded{\includegraphics[keepaspectratio]{images/EMT_Grafik3D_Alfi Nur Azumah-230.png}}
\caption{images/EMT\_Grafik3D\_Alfi\%20Nur\%20Azumah-230.png}
\end{figure}

hasil visualisasi fungsi dapat dibuat menjadi transparan dan menambahkan hasil akhir lainnya.

\textgreater{} pov3d(``(x\textsuperscript{2+y}3)'',axiscolor=green,angle=30°, \ldots{}\\
\textgreater{} look=povlook(yellow,0.2),level=-1:0.5:1,zoom=3);

\begin{figure}
\centering
\pandocbounded{\includegraphics[keepaspectratio]{images/EMT_Grafik3D_Alfi Nur Azumah-231.png}}
\caption{images/EMT\_Grafik3D\_Alfi\%20Nur\%20Azumah-231.png}
\end{figure}

\textgreater pov3d(``((x-1)\textsuperscript{2+(y+1)}2)*((x+1)\textsuperscript{2+y}2)/40'',r=1.5, \ldots{}\\
\textgreater{} angle=120°,level=1/40,dlevel=0.005,light={[}-1,1,1{]},height=45°,n=50, \ldots{}\\
\textgreater{} \textless fscale,zoom=3.8);

\begin{figure}
\centering
\pandocbounded{\includegraphics[keepaspectratio]{images/EMT_Grafik3D_Alfi Nur Azumah-232.png}}
\caption{images/EMT\_Grafik3D\_Alfi\%20Nur\%20Azumah-232.png}
\end{figure}

\textbf{Object Povray}

Contoh-contoh di atas tadi merupakan visualisasi permukaan fungsi dengan menggunakan sintaks pov3d. Untuk menghasilkan objek dalam povray perlu ditulis menjadi file povray.

Untuk menghasilkan output dimulai dengan povstart()

\textgreater load povray; \ldots{}\\
\textgreater{} defaultpovray=``C:\textbackslash Program Files\textbackslash POV-Ray\textbackslash v3.7\textbackslash bin\textbackslash pvengine.exe''

\begin{verbatim}
C:\Program Files\POV-Ray\v3.7\bin\pvengine.exe
\end{verbatim}

\textgreater povstart(zoom=3.5)

\textgreater c1=povcylinder(-povx,povx,1,povlook(orange)); \ldots{}\\
\textgreater{} c2=povcylinder(-povy,povy,1,povlook(yellow)); \ldots{}\\
\textgreater{} c3=povcylinder(-povz,povz,1,povlook(lightblue));

Di atas telah didefinisikan tiga silinder yang disimpan dalam string di Euler. Fungsi povx(), povy(), dll. hanya mengembalikan vektor {[}1,0,0{]} yang dapat digunakan sebagai gantinya.

\textgreater c1

\begin{verbatim}
cylinder { &lt;-1,0,0&gt;, &lt;1,0,0&gt;, 1
 texture { pigment { color rgb &lt;0.941176,0.509804,0.392157&gt; }  } 
 finish { ambient 0.2 } 
 }
\end{verbatim}

Akan ditambahkan tekstur ke objek dengan tiga warna berbeda yaitu orange, yellow, dan lightblue.

Untuk menamahkan tekstur ini dapat digunakan sintaks povlook(), yang mengembalikan string dengan kode Povray yang relevan. Selain menambahkan warna, ditambahkan juga transparansi dan cahaya.

\textgreater povlook(rgb(0.1,0.2,0.3),0.1,0.5)

\begin{verbatim}
 texture { pigment { color rgbf &lt;0.101961,0.2,0.301961,0.1&gt; }  } 
 finish { ambient 0.5 } 
\end{verbatim}

\textgreater writeln(povintersection({[}c1,c2,c3{]}));

\textgreater povend;

\begin{figure}
\centering
\pandocbounded{\includegraphics[keepaspectratio]{images/EMT_Grafik3D_Alfi Nur Azumah-233.png}}
\caption{images/EMT\_Grafik3D\_Alfi\%20Nur\%20Azumah-233.png}
\end{figure}

\textbf{Contoh Lain}

Akan ditampilkan fungsi untuk membuat sebuah donat

\textgreater povstart(angle=0,height=45°); //height untuk menampilkan fungsi dengan suatiu derajat tertentu

\textgreater function povdonat (r1,r2,look=``\,``) :=''torus \{``+r1+'',``+r2+look+''\}``; //fungsi untuk menampilkan sebuah donat

\textgreater writeln(povobject(povdonat(1,0.5),povlook(lightblue,\textgreater phong),xrotate(90°)));

\textgreater povend();

\begin{figure}
\centering
\pandocbounded{\includegraphics[keepaspectratio]{images/EMT_Grafik3D_Alfi Nur Azumah-234.png}}
\caption{images/EMT\_Grafik3D\_Alfi\%20Nur\%20Azumah-234.png}
\end{figure}

\subsection{4.13 Menggambar Grafik Tiga Dimensi alam modus anaglif}\label{menggambar-grafik-tiga-dimensi-alam-modus-anaglif}

Untuk menghasilkan anaglyph untuk kacamata merah/sian, Povray harus berjalan dua kali dari posisi kamera yang berbeda. Ini menghasilkan dua file Povray dan dua file PNG, yang dimuat dengan fungsi loadanaglyph().

Tentu saja, Anda memerlukan kacamata merah/sian untuk melihat contoh berikut dengan benar.

Fungsi pov3d() memiliki sakelar sederhana untuk menghasilkan anaglyphs.

\textgreater pov3d(``-exp(-x\textsuperscript{2-y}2)/2'',r=2,height=45°,\textgreater anaglyph, \ldots{}\\
\textgreater{} center={[}0,0,0.5{]},zoom=3.5);

\begin{figure}
\centering
\pandocbounded{\includegraphics[keepaspectratio]{images/EMT_Grafik3D_Alfi Nur Azumah-235.png}}
\caption{images/EMT\_Grafik3D\_Alfi\%20Nur\%20Azumah-235.png}
\end{figure}

Jika Anda membuat adegan dengan objek, Anda perlu menempatkan generasi adegan ke dalam fungsi, dan menjalankannya dua kali dengan nilai yang berbeda untuk parameter anaglyph.

\textgreater function myscene \ldots{}

\begin{verbatim}
  s=povsphere(povc,1);
  cl=povcylinder(-povz,povz,0.5);
  clx=povobject(cl,rotate=xrotate(90°));
  cly=povobject(cl,rotate=yrotate(90°));
  c=povbox([-1,-1,0],1);
  un=povunion([cl,clx,cly,c]);
  obj=povdifference(s,un,povlook(red));
  writeln(obj);
  writeAxes();
endfunction
\end{verbatim}

Fungsi povanaglyph() melakukan semua ini. Parameternya seperti di povstart() dan povend() digabungkan.

\textgreater povanaglyph(``myscene'',zoom=4.5);

\begin{figure}
\centering
\pandocbounded{\includegraphics[keepaspectratio]{images/EMT_Grafik3D_Alfi Nur Azumah-236.png}}
\caption{images/EMT\_Grafik3D\_Alfi\%20Nur\%20Azumah-236.png}
\end{figure}

\subsection{4.14 Fungsi Implisit menggunakan Povray}\label{fungsi-implisit-menggunakan-povray}

Povray dapat memplot himpunan di mana f(x,y,z)=0, seperti parameter implisit di plot3d. Namun hasilnya terlihat jauh lebih baik.

Sintaks untuk fungsinya sedikit berbeda. Anda tidak dapat menggunakan keluaran ekspresi Maxima atau Euler.

\textgreater load povray;

\textgreater defaultpovray=``C:\textbackslash Program Files\textbackslash POV-Ray\textbackslash v3.7\textbackslash bin\textbackslash pvengine.exe''

\begin{verbatim}
C:\Program Files\POV-Ray\v3.7\bin\pvengine.exe
\end{verbatim}

\textgreater povstart(angle=25°,height=10°);

\textgreater writeln(povsurface(``pow(x,2)+pow(y,2)*pow(z,2)-1'',povlook(blue),povbox(-2,2,``\,``)));

\textgreater povend();

\begin{figure}
\centering
\pandocbounded{\includegraphics[keepaspectratio]{images/EMT_Grafik3D_Alfi Nur Azumah-237.png}}
\caption{images/EMT\_Grafik3D\_Alfi\%20Nur\%20Azumah-237.png}
\end{figure}

\textgreater load povray;

\textgreater defaultpovray=``C:\textbackslash Program Files\textbackslash POV-Ray\textbackslash v3.7\textbackslash bin\textbackslash pvengine.exe''

\begin{verbatim}
C:\Program Files\POV-Ray\v3.7\bin\pvengine.exe
\end{verbatim}

\textgreater povstart(angle=70°,height=50°,zoom=4);

\textgreater writeln(povsurface(``pow(x,2)*y-pow(y,3)-pow(z,2)'',povlook(green))); \ldots{}\\
\textgreater{} writeAxes(); \ldots{}\\
\textgreater{} povend();

\begin{figure}
\centering
\pandocbounded{\includegraphics[keepaspectratio]{images/EMT_Grafik3D_Alfi Nur Azumah-238.png}}
\caption{images/EMT\_Grafik3D\_Alfi\%20Nur\%20Azumah-238.png}
\end{figure}

\textgreater load povray;

\textgreater defaultpovray=``C:\textbackslash Program Files\textbackslash POV-Ray\textbackslash v3.7\textbackslash bin\textbackslash pvengine.exe''

\begin{verbatim}
C:\Program Files\POV-Ray\v3.7\bin\pvengine.exe
\end{verbatim}

\textgreater povstart(angle=70°,height=30°);

\textgreater writeln(povsurface(``pow(x,2)+pow(y,2)*pow(z,2)-1'',povlook(red),povbox(-2,2,``\,``)));

\textgreater povend();

\begin{figure}
\centering
\pandocbounded{\includegraphics[keepaspectratio]{images/EMT_Grafik3D_Alfi Nur Azumah-239.png}}
\caption{images/EMT\_Grafik3D\_Alfi\%20Nur\%20Azumah-239.png}
\end{figure}

\subsection{4.15 Menggambar Titik pada ruang Tiga Dimensi (3D)}\label{menggambar-titik-pada-ruang-tiga-dimensi-3d}

Alih-alih fungsi, kita dapat memplot dengan koordinat. Seperti pada plot3d, kita membutuhkan tiga matriks untuk mendefinisikan objek.

Dalam contoh kita memutar fungsi di sekitar sumbu z.

\textgreater function f(x) := x\^{}3-x+1; \ldots{}\\
\textgreater{} x=-1:0.01:1; t=linspace(0,2pi,8)'; \ldots{}\\
\textgreater{} Z=x; X=cos(t)*f(x); Y=sin(t)*f(x); \ldots{}\\
\textgreater{} pov3d(X,Y,Z,angle=40°,height=20°,axis=0,zoom=4,light={[}10,-5,5{]});

\begin{figure}
\centering
\pandocbounded{\includegraphics[keepaspectratio]{images/EMT_Grafik3D_Alfi Nur Azumah-240.png}}
\caption{images/EMT\_Grafik3D\_Alfi\%20Nur\%20Azumah-240.png}
\end{figure}

Dalam contoh berikut, kami memplot gelombang teredam. Kami menghasilkan gelombang dengan bahasa matriks Euler.

Kami juga menunjukkan, bagaimana objek tambahan dapat ditambahkan ke adegan pov3d. Untuk pembuatan objek, lihat contoh berikut. Perhatikan bahwa plot3d menskalakan plot, sehingga cocok dengan kubus satuan.

\textgreater r=linspace(0,1,80); phi=linspace(0,2pi,80)'; \ldots{}\\
\textgreater{} x=r*cos(phi); y=r*sin(phi); z=exp(-5*r)*cos(8*pi*r)/3; \ldots{}\\
\textgreater{} pov3d(x,y,z,zoom=5,axis=0,add=povsphere({[}0,0,0.5{]},0.1,povlook(green)), \ldots{}\\
\textgreater{} w=500,h=300);

\begin{figure}
\centering
\pandocbounded{\includegraphics[keepaspectratio]{images/EMT_Grafik3D_Alfi Nur Azumah-241.png}}
\caption{images/EMT\_Grafik3D\_Alfi\%20Nur\%20Azumah-241.png}
\end{figure}

Dengan metode bayangan canggih dari Povray, sangat sedikit titik yang dapat menghasilkan permukaan yang sangat halus. Hanya di perbatasan dan dalam bayang-bayang triknya mungkin menjadi jelas.

Untuk ini, kita perlu menambahkan vektor normal di setiap titik matriks.

\textgreater Z \&= x\textsuperscript{2*y}3

\begin{verbatim}
                                 2  3
                                x  y
\end{verbatim}

Persamaan permukaannya adalah {[}x,y,Z{]}. Kami menghitung dua turunan ke x dan y ini dan mengambil produk silang sebagai normal.

\textgreater dx \&= diff({[}x,y,Z{]},x); dy \&= diff({[}x,y,Z{]},y);

Kami mendefinisikan normal sebagai produk silang dari turunan ini, dan mendefinisikan fungsi koordinat.

\textgreater N \&= crossproduct(dx,dy); NX \&= N{[}1{]}; NY \&= N{[}2{]}; NZ \&= N{[}3{]}; N,

\begin{verbatim}
                               3       2  2
                       [- 2 x y , - 3 x  y , 1]
\end{verbatim}

Kami hanya menggunakan 25 poin.

\textgreater x=-1:0.5:1; y=x';

\textgreater pov3d(x,y,Z(x,y),angle=10°, \ldots{}\\
\textgreater{} xv=NX(x,y),yv=NY(x,y),zv=NZ(x,y),\textless shadow);

\begin{figure}
\centering
\pandocbounded{\includegraphics[keepaspectratio]{images/EMT_Grafik3D_Alfi Nur Azumah-242.png}}
\caption{images/EMT\_Grafik3D\_Alfi\%20Nur\%20Azumah-242.png}
\end{figure}

Berikut ini adalah simpul Trefoil yang dilakukan oleh A. Busser di Povray. Ada versi yang ditingkatkan dari ini dalam contoh.

Simpul trefoil

Untuk tampilan yang bagus dengan tidak terlalu banyak titik, kami menambahkan vektor normal di sini. Kami menggunakan Maxima untuk menghitung normal bagi kami. Pertama, ketiga fungsi koordinat sebagai ekspresi simbolik.

\textgreater X \&= ((4+sin(3*y))+cos(x))*cos(2*y); \ldots{}\\
\textgreater{} Y \&= ((4+sin(3*y))+cos(x))*sin(2*y); \ldots{}\\
\textgreater{} Z \&= sin(x)+2*cos(3*y);

Kemudian kedua vektor turunan ke x dan y.

\textgreater dx \&= diff({[}X,Y,Z{]},x); dy \&= diff({[}X,Y,Z{]},y);

Sekarang normal, yang merupakan produk silang dari dua turunan.

\textgreater dn \&= crossproduct(dx,dy);

Kami sekarang mengevaluasi semua ini secara numerik.

\textgreater x:=linspace(-\%pi,\%pi,40); y:=linspace(-\%pi,\%pi,100)';

Vektor normal adalah evaluasi dari ekspresi simbolik dn{[}i{]} untuk i=1,2,3. Sintaks untuk ini adalah \&``expression''(parameters). Ini adalah alternatif dari metode pada contoh sebelumnya, di mana kita mendefinisikan ekspresi simbolik NX, NY, NZ terlebih dahulu.

\textgreater pov3d(X(x,y),Y(x,y),Z(x,y),axis=0,zoom=5,w=450,h=350, \ldots{}\\
\textgreater{} \textless shadow,look=povlook(gray), \ldots{}\\
\textgreater{} xv=\&``dn{[}1{]}''(x,y), yv=\&``dn{[}2{]}''(x,y), zv=\&``dn{[}3{]}''(x,y));

\begin{figure}
\centering
\pandocbounded{\includegraphics[keepaspectratio]{images/EMT_Grafik3D_Alfi Nur Azumah-243.png}}
\caption{images/EMT\_Grafik3D\_Alfi\%20Nur\%20Azumah-243.png}
\end{figure}

Kami juga dapat menghasilkan grid dalam 3D.

\textgreater povstart(zoom=4); \ldots{}\\
\textgreater{} x=-1:0.5:1; r=1-(x+1)\^{}2/6; \ldots{}\\
\textgreater{} t=(0°:30°:360°)'; y=r*cos(t); z=r*sin(t); \ldots{}\\
\textgreater{} writeln(povgrid(x,y,z,d=0.02,dballs=0.05)); \ldots{}\\
\textgreater{} povend();

\begin{figure}
\centering
\pandocbounded{\includegraphics[keepaspectratio]{images/EMT_Grafik3D_Alfi Nur Azumah-244.png}}
\caption{images/EMT\_Grafik3D\_Alfi\%20Nur\%20Azumah-244.png}
\end{figure}

With povgrid(), curves are possible.

\textgreater povstart(center={[}0,0,1{]},zoom=3.6); \ldots{}\\
\textgreater{} t=linspace(0,2,1000); r=exp(-t); \ldots{}\\
\textgreater{} x=cos(2*pi*10*t)*r; y=sin(2*pi*10*t)*r; z=t; \ldots{}\\
\textgreater{} writeln(povgrid(x,y,z,povlook(red))); \ldots{}\\
\textgreater{} writeAxis(0,2,axis=3); \ldots{}\\
\textgreater{} povend();

\begin{figure}
\centering
\pandocbounded{\includegraphics[keepaspectratio]{images/EMT_Grafik3D_Alfi Nur Azumah-245.png}}
\caption{images/EMT\_Grafik3D\_Alfi\%20Nur\%20Azumah-245.png}
\end{figure}

\textbf{Latihan Soal}

\begin{enumerate}
\def\labelenumi{\arabic{enumi}.}
\tightlist
\item
  Buatlah plot 3D dari fungsi
\end{enumerate}

\[f(x,y)=x^3+3y^2\]

dengan zoom 3 dan angle 55 derajat menggunakan povray

\textgreater pov3d(``x\textsuperscript{3+3*y}2'',zoom=3,angle=55°);

\begin{figure}
\centering
\pandocbounded{\includegraphics[keepaspectratio]{images/EMT_Grafik3D_Alfi Nur Azumah-247.png}}
\caption{images/EMT\_Grafik3D\_Alfi\%20Nur\%20Azumah-247.png}
\end{figure}

\begin{enumerate}
\def\labelenumi{\arabic{enumi}.}
\setcounter{enumi}{1}
\tightlist
\item
  Buatlah gabungan 2 silinder dengan fungsi povx() berwarna merah dan povz() berwarna kuning dan zoom 4
\end{enumerate}

\textgreater povstart(zoom=4)

\textgreater c1 = povcylinder (-povx,povx,1,povlook(red));

\textgreater c2 = povcylinder (-povz,povz,1,povlook(yellow));

\textgreater writeln(povintersection({[}c1,c2{]}));

\textgreater povend();

\begin{figure}
\centering
\pandocbounded{\includegraphics[keepaspectratio]{images/EMT_Grafik3D_Alfi Nur Azumah-248.png}}
\caption{images/EMT\_Grafik3D\_Alfi\%20Nur\%20Azumah-248.png}
\end{figure}

\textgreater{}

\begin{enumerate}
\def\labelenumi{\arabic{enumi}.}
\setcounter{enumi}{2}
\tightlist
\item
  Buatlah grafik 3D dari fungsi kuadrat berikut ini dengan parameter tambahan:
\end{enumerate}

\[z=4x^2-2y^2\]

Tampilkan grafik tersebut dengan transparent, dan menggunakan grid dengan resolusi 50, dengan warna biru pada garis di plot tersebut

\textgreater plot3d(``4*x\textsuperscript{2-2*y}2'',\textgreater transparent,grid=50,wirecolor=blue):

\begin{figure}
\centering
\pandocbounded{\includegraphics[keepaspectratio]{images/EMT_Grafik3D_Alfi Nur Azumah-250.png}}
\caption{images/EMT\_Grafik3D\_Alfi\%20Nur\%20Azumah-250.png}
\end{figure}

\backmatter
\end{document}
